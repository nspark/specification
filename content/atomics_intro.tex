An \ac{AMO} is a one-sided communication mechanism that combines memory read,
update, or write operations with atomicity guarantees described in Section~%
\ref{subsec:amo_guarantees}.  Similar to the \ac{RMA} routines, described in
Section~\ref{sec:rma}, the \acp{AMO} are performed only on symmetric objects.
\openshmem defines two types of \ac{AMO} routines:

\begin{itemize}

\item
  The \emph{fetching} routines return the original value of, and optionally
  update, the remote data object in a single atomic operation.  The routines
  return after the data has been fetched from the target \ac{PE} and delivered
  to the calling \ac{PE}.
  The data type of the returned value is the same as the type of
  the remote data object.

  The fetching routines include:
  \FUNC{shmem\_atomic\_\{fetch, compare\_swap, swap\}} and
  \FUNC{shmem\_atomic\_fetch\_\{inc, add, and, or, xor\}}.

\item
  The \emph{non-fetching} routines update the remote data object in a single
  atomic operation.  A call to a non-fetching atomic routine issues the atomic
  operation and may return before the operation executes on the target \ac{PE}.
  The \FUNC{shmem\_quiet}, \FUNC{shmem\_barrier}, or \FUNC{shmem\_barrier\_all}
  routines can be used to force completion for these non-fetching
  atomic routines.

  The non-fetching routines include:
  \FUNC{shmem\_atomic\_\{set, inc, add, and, or, xor\}}.

\end{itemize}

Where appropriate compiler support is available, \openshmem provides
type-generic \ac{AMO} interfaces via \Cstd[11] generic selection.
The type-generic support for the \ac{AMO} routines is as follows:

\begin{itemize}
\item \FUNC{shmem\_atomic\_\{compare\_swap, fetch\_inc, inc, fetch\_add, add\}}
  support the ``standard \ac{AMO} types'' listed in Table~\ref{stdamotypes},
\item \FUNC{shmem\_atomic\_\{fetch, set, swap\}} support
  the ``extended \ac{AMO} types'' listed in Table~\ref{extamotypes}, and
\item \FUNC{shmem\_atomic\_\{fetch\_and, and, fetch\_or, or, fetch\_xor, xor\}}
  support the ``bitwise \ac{AMO} types'' listed in Table~\ref{bitamotypes}.
\end{itemize}

The standard, extended, and bitwise \ac{AMO} types include some of the exact-width
integer types defined in \HEADER{stdint.h} by \Cstd[99]~\S7.18.1.1 and
\Cstd[11]~\S7.20.1.1. When the \Cstd translation environment
does not provide exact-width integer types with \HEADER{stdint.h}, an
\openshmem implemementation is not required to provide support for these types.

\begin{table}[h]
  \begin{center}
    \begin{tabular}{|l|l|}
      \hline
      \TYPE              & \TYPENAME  \\ \hline
      int                & int        \\ \hline
      long               & long       \\ \hline
      long long          & longlong   \\ \hline
      unsigned int       & uint       \\ \hline
      unsigned long      & ulong      \\ \hline
      unsigned long long & ulonglong  \\ \hline
      int32\_t           & int32      \\ \hline
      int64\_t           & int64      \\ \hline
      uint32\_t          & uint32     \\ \hline
      uint64\_t          & uint64     \\ \hline
      size\_t            & size       \\ \hline
      ptrdiff\_t         & ptrdiff    \\ \hline
    \end{tabular}
    \caption{Standard \ac{AMO} Types and Names}
    \label{stdamotypes}
  \end{center}
\end{table}

\begin{table}[h]
  \begin{center}
    \begin{tabular}{|l|l|}
      \hline
      \TYPE              & \TYPENAME  \\ \hline
      float              & float      \\ \hline
      double             & double     \\ \hline
      int                & int        \\ \hline
      long               & long       \\ \hline
      long long          & longlong   \\ \hline
      unsigned int       & uint       \\ \hline
      unsigned long      & ulong      \\ \hline
      unsigned long long & ulonglong  \\ \hline
      int32\_t           & int32      \\ \hline
      int64\_t           & int64      \\ \hline
      uint32\_t          & uint32     \\ \hline
      uint64\_t          & uint64     \\ \hline
      size\_t            & size       \\ \hline
      ptrdiff\_t         & ptrdiff    \\ \hline
    \end{tabular}
    \caption{Extended \ac{AMO} Types and Names}
    \label{extamotypes}
  \end{center}
\end{table}

\begin{table}[h]
  \begin{center}
    \begin{tabular}{|l|l|}
      \hline
      \TYPE              & \TYPENAME  \\ \hline
      unsigned int       & uint       \\ \hline
      unsigned long      & ulong      \\ \hline
      unsigned long long & ulonglong  \\ \hline
      int32\_t           & int32      \\ \hline
      int64\_t           & int64      \\ \hline
      uint32\_t          & uint32     \\ \hline
      uint64\_t          & uint64     \\ \hline
    \end{tabular}
    \caption{Bitwise \ac{AMO} Types and Names}
    \label{bitamotypes}
  \end{center}
\end{table}
