\documentclass[10pt,oneside]{book}

\input{utils/packages}

\definecolor{ListingBG}{rgb}{0.91,0.91,0.91}
\definecolor{shadecolor}{rgb}{0.92,0.92,0.92}

\hyphenation{Open-SHMEM}

\renewcommand{\chaptername}{Chapter} 
\renewcommand{\appendixname}{Annex} 

% Place some penalty for doing the break
% The penalty for a ``\gb'' should be greater than a \hyphenpenalty.
% \hyphenpenalty is 50 in plain.tex.
\def\gb{\penalty10000\hskip 0pt plus 8em\penalty4800\hskip 0pt plus-8em%
\penalty10000}

% This macro enables that all "_" (underscore) characters in the pfd
% file are searchable, and that cut&paste will copy the "_" as underscore. 
% Without the following macro, the \_ is treated in searches and cut&paste
% as a " " (space character). 
% This macro does not modify the behavior of _ in math or in verbatim 
% environments. In verbatim environments, the "_" is always treated
% as a searchable character.
%
\DeclareRobustCommand{\_}{\texttt{\char`\_}} 
% 

\def\colorswapnt{\colorlet{saved}{.}\color{ForestGreen}}
\def\colorswapot{\colorlet{saved}{.}\color{red}}
\def\prevcolor{\color{saved}}

\newcommand{\newtext}[1]{\textcolor{ForestGreen}{#1}}
\newcommand{\oldtext}[1]{\textcolor{magenta}{\sout{#1}}}
\newcommand{\insertDocVersion}{1.4}
\newcommand{\openshmem}[1][]{%
  {Open\-SHMEM\ifthenelse{\equal{#1}{}}{}{~#1}}\xspace}
\newcommand{\HEADER}[1]{\textit{#1}}
\newcommand{\FUNC}[1]{\textit{#1}}
\newcommand{\CTYPE}[1]{\textit{#1}}
\newcommand{\VAR}[1]{\textit{#1}}
\newcommand{\CONST}[1]{\textit{#1}}
\newcommand{\const}[1]{\protect\gb\protect{\textsf{\small #1}}\index{CONST:#1}} % Only library_constants.tex table.
\newcommand{\CorCpp}{\textit{C/C++}\xspace}
\newcommand{\CorCppFor}{\textit{C/C++/Fortran}\xspace}
\newcommand{\Fortran}[1][]{%
  \textit{Fortran\ifthenelse{\equal{#1}{}}{}{~#1}}\xspace}
\newcommand{\Cstd}[1][]{%
  \textit{C\ifthenelse{\equal{#1}{}}{}{#1}}\xspace}
\newcommand{\Cpp}[1][]{%
  \textit{C++\ifthenelse{\equal{#1}{}}{}{#1}}\xspace}
\newcommand{\TYPE}{\emph{TYPE}}
\newcommand{\TYPENAME}{\emph{TYPENAME}}
\newcommand{\SIZE}{\emph{SIZE}}

\newcommand{\source}{\textit{source}}
\newcommand{\dest}{\textit{dest}}
\newcommand{\PUT}{\textit{Put}}
\newcommand{\GET}{\textit{Get}}
\newcommand{\OPR}[1]{\textit{#1}}
\newcommand{\barrier}{\FUNC{SHMEM\_BARRIER}\xspace} % why here an not others?
\newcommand{\barrierall}{\FUNC{SHMEM\_BARRIER\_ALL}\xspace} % why here an not others?
\newcommand{\broadcast}{\FUNC{SHMEM\_BROADCAST}}
\newcommand{\collect}{\FUNC{SHMEM\_COLLECT}}
\newcommand{\fcollect}{\FUNC{SHMEM\_FCOLLECT}}
\newcommand{\reduction}{\textit{Reduction Operations}}
\newcommand{\alltoall}{\FUNC{SHMEM\_ALLTOALL}}
\newcommand{\alltoalls}{\FUNC{SHMEM\_ALLTOALLS}}
\newcommand{\activeset}{\textit{Active~set}\xspace} % why here and not others?
\newcommand{\shmemprefix}{\textit{SHMEM\_}}
\newcommand{\shmemprefixLC}{\textit{shmem\_}}
\newcommand{\shmemprefixC}{\textit{\_SHMEM\_}}
\newcommand{\ith}{${\textit{i}^{\text{\tiny th}}}$}
\newcommand{\jth}{${\textit{j}^{\text{\tiny th}}}$}
\newcommand{\kth}{${\textit{k}^{\text{\tiny th}}}$}
\newcommand{\lth}{${\textit{l}^{\text{\tiny th}}}$}


\begin{acronym}
\acro{RMA}{\emph{Remote Memory Access}}
\acro{RMO}{\emph{Remote Memory Operation}}
\acro{AMO}{\emph{Atomic Memory Operation}}
\acro{PE}{\emph{Processing Element}}
\acrodefplural{PE}[PEs]{\emph{Processing Elements}}
\acro{PGAS}{\emph{Partitioned Global Address Space}}
\acro{API}{\emph{Application Programming Interface}}
\acro{MPI}{\emph{Message Passing Interface}}
\acro{SPMD}{\emph{Single Program Multiple Data}}
\acro{UH}{University of Houston}
\acro{UO}{University of Oregon}
\acro{ORNL}{Oak Ridge National Laboratory}
\acro{LANL}{Los Alamos National Laboratory}
\acro{ESSC}{Extreme Scale Systems Center}
\acro{OSSS}{Open Software System Solutions}
\acro{DoD}{U.S. Department of Defense}
\end{acronym}


%
% This is used to put line numbers on plain pages.  Used in draft.tex
%
\makeatletter

\def\withlinenumbers{\relax
  \def\@evenfoot{\hbox to 0pt{\hss\LineNumberRuler\hskip 1.5pc}\hfil}\relax
  \def\@oddfoot{\hfil\hbox to 0pt{\hskip 1.5pc\LineNumberRuler\hss}}}

\def\LineNumberRuler{\vbox to 0pt{\vss\normalsize \baselineskip13.6pt
    \lineskip 1pt \normallineskip 1pt \def\baselinestretch{1}\relax
    \LNR{1}\LNR{2}\LNR{3}\LNR{4}\LNR{5}\LNR{6}\LNR{7}\LNR{8}\LNR{9}
    \LNR{10}\LNR{11}\LNR{12}\LNR{13}\LNR{14}
        \LNR{15}\LNR{16}\LNR{17}\LNR{18}\LNR{19}
    \LNR{20}\LNR{21}\LNR{22}\LNR{23}\LNR{24}
        \LNR{25}\LNR{26}\LNR{27}\LNR{28}\LNR{29}
    \LNR{30}\LNR{31}\LNR{32}\LNR{33}\LNR{34}\LNR{35}
        \LNR{36}\LNR{37}\LNR{38}\LNR{39}
    \LNR{40}\LNR{41}\LNR{42}\LNR{43}\LNR{44}
        \LNR{45}\LNR{46}\LNR{47}\LNR{48}
    \vskip 31pt}}
\def\LNR#1{\hbox to 1pc{\hfil\tiny#1\hfil}}

\def\ps@plainwithlinenumbers{\let\@mkboth\@gobbletwo
     \def\@oddhead{}
     \def\@oddfoot{\hfil\rm\thepage\hfil
       \hbox to 0pt{\hskip 1.5pc\LineNumberRuler\hss}}
     \def\@evenhead{}
     \def\@evenfoot{\hbox to 0pt{\hss
     \LineNumberRuler\hskip 1.5pc}\rm\hfil\thepage\hfil}}

    % Contents is done with \chapter*{Contents}, so we need to turn off the
    % line numbers in this case.  Easiest to look at def

\newwrite\chappages
\immediate\openout\chappages=chappage.txt
\def\writespace{ }

\def\incontents{0}
\newif\ifcontents
\contentsfalse
\def\chapter{\clearpage \ifcontents\else\thispagestyle{plainwithlinenumbers}\fi
        \write\chappages{Chapter \thechapter\writespace - \the\count0}
        \global\@topnum\z@ \@afterindentfalse \secdef\@chapter\@schapter}

\makeatother

%
% End this is used to put line numbers on plain pages.  Used in draft.tex
%

%
% Use Sans Serif font for sections, etc.
%
\makeatletter
\def\section{\@startsection {section}{1}{\z@}{-3.5ex plus -1ex minus 
-.2ex}{2.3ex plus .2ex}{\Large\sf}}
\def\subsection{\@startsection{subsection}{2}{\z@}{-3.25ex plus -1ex minus 
-.2ex}{1.5ex plus .2ex}{\large\sf}}
\def\subsubsection{\@startsection{subsubsection}{3}{\z@}{-3.25ex plus 
-1ex minus -.2ex}{1.5ex plus .2ex}{\normalsize\sf\bf}}
\def\paragraph{\@startsection {paragraph}{4}{\z@}{3.25ex plus 1ex 
minus .2ex}{-1em}{\normalsize\sf}}
\makeatother
%
% End use Sans Serif font for sections, etc.  S. Otto
%


%
% This section is for example code listings
%
\definecolor{gray}{rgb}{0.92,0.92,0.92}

\lstset{ % set defaults for languages not otherwise defined
  breakatwhitespace=false,         % sets if automatic breaks should only happen at whitespace
  basicstyle=\ttfamily\footnotesize,
  breaklines=true,                 % sets automatic line breaking
  escapeinside={|}{|},          % if you want to add LaTeX within your code
  extendedchars=true,              % lets you use non-ASCII characters; for 8-bits 
                                   % encodings only, does not work with UTF-8
  keepspaces=true,                 % keeps spaces in text, useful for keeping indentation of code 
                                   % (possibly needs columns=flexible)
  morekeywords={*,...},            % if you want to add more keywords to the set
  showspaces=false,                % show spaces everywhere adding particular underscores; 
                                   % it overrides 'showstringspaces'
  showstringspaces=false,          % underline spaces within strings only
  showtabs=false,                  % show tabs within strings adding particular underscores
}

\def\StandardListing {
  \lstset {
    breakatwhitespace=false,         % sets if automatic breaks should only happen at whitespace
    basicstyle=\ttfamily\footnotesize,
    breaklines=true,                 % sets automatic line breaking
    escapeinside={\%*}{*)},          % if you want to add LaTeX within your code
    extendedchars=true,              % lets you use non-ASCII characters; for 8-bits 
                                     % encodings only, does not work with UTF-8
    keepspaces=true,                 % keeps spaces in text, useful for keeping
                                     % indentation of code (possibly needs columns=flexible)
    morekeywords={*,...},            % if you want to add more keywords to the set
    showspaces=false,                % show spaces everywhere adding particular underscores; 
                                     % it overrides 'showstringspaces'
    showstringspaces=false,          % underline spaces within strings only
    showtabs=false,                  % show tabs within strings adding particular underscores
    backgroundcolor=\color{gray}, 
  }
}

\def\ProgramNumberedListing {
  \StandardListing
  \lstset {
    numbers=left,
    numberstyle=\footnotesize
  }
}

\newcommand{\numberedlisting}[2] {
  \ProgramNumberedListing
  \lstinputlisting[#1]{#2}
  \StandardListing
}

\newcommand{\outputlisting}[2] {
\begin{minipage}{\linewidth}
\vspace{0.1in}
  \lstinputlisting[#1]{#2}
  \StandardListing
\vspace{0.1in}
\end{minipage}
}

\lstdefinelanguage{OSH+C}[]{C}{
  classoffset=1,
  morekeywords={
    size_t, ptrdiff_t,
    SHMEM_BCAST_SYNC_SIZE, SHMEM_SYNC_VALUE,
    start_pes,
    my_pe, _my_pe, shmem_my_pe,
    num_pes, _num_pes, shmem_n_pes,
    shmem_int_p, shmem_short_p, shmem_long_p,
    shmem_int_put, shmem_short_put, shmem_long_put,
    shmem_barrier_all, shmem_barrier,
    shmalloc,  shfree, shrealloc,
    shmem_broadcast32, shmem_broadcast64,
    shmem_short_inc, shmem_int_inc, shmem_long_inc,
    shmem_short_add, shmem_int_add, shmem_long_add,
    shmem_short_finc, shmem_int_finc, shmem_long_finc,
    shmem_short_fadd, shmem_int_fadd, shmem_long_fadd,
    shmem_set_lock, shmem_test_lock, shmem_clear_lock,
    shmem_long_sum_to_all,
    shmem_complexd_sum_to_all,
    shmem_cmp_t
  },
  keywordstyle=\color{black}\textbf,
  classoffset=0,
  sensitive=true
}

\lstdefinelanguage{OSH2+C}[]{OSH+C}{
  classoffset=1,
  morekeywords={
    shmem_init,
    shmem_finalize,
    shmem_malloc,
    shmem_my_pe,
    shmem_error,
    shmem_global_exit,
  },
  keywordstyle=\color{black}\textbf,
  classoffset=0,
  sensitive=true
}

\lstdefinelanguage{OSH+F}[]{Fortran}{
  classoffset=1,
  morekeywords={
    SHMEM_BCAST_SYNC_SIZE, SHMEM_SYNC_VALUE,
    start_pes,
    my_pe, shmem_my_pe,
    num_pes, shmem_n_pes,
    shmem_int_p, shmem_short_p, shmem_long_p,
    shmem_int_put, shmem_short_put, shmem_long_put,
    shmem_barrier_all, shmem_barrier,
    shpalloc,  shpdeallc, shpclmove,
    shmem_broadcast32, shmem_broadcast64,
    shmem_broadcast4, shmem_broadcast8,
    shmem_short_inc, shmem_int_inc, shmem_long_inc,
    shmem_short_add, shmem_int_add, shmem_long_add,
    shmem_short_finc, shmem_int_finc, shmem_long_finc,
    shmem_short_fadd, shmem_int_fadd, shmem_long_fadd,
    shmem_set_lock, shmem_test_lock, shmem_clear_lock,
    shmem_long_sum_to_all,
  },
  keywordstyle=\color{black}\textbf,
  classoffset=0,
  sensitive=false
}

\lstdefinelanguage{OSH2+F}[]{OSH+F}{
  classoffset=1,
  morekeywords={
    shmem_init,
    shmem_finalize,
    shmem_malloc,
    shmem_my_pe,
    shmem_error,
    shmem_global_exit,
  },
  keywordstyle=\color{black}\textbf,
  classoffset=0,
  sensitive=true
}

%
% End this section is for example code listings
%

%
% Deprecation Helpers
%

\newcommand{\strikeline}[1][red]{{\color{#1}\raisebox{.5ex}{\rule{1em}{.4pt}}}}
\newcommand{\stretchline}[1][red]{\xrfill[.5ex]{.4pt}[#1]}
\newcommand{\DeprecationStart}[1][red]{{\color{#1} deprecation start} \mbox{}}
\newcommand{\DeprecationEnd}[1][red]{{\color{#1} deprecation end} \mbox{}}

\newcommand{\StartDeprecateBlock}{
  {\strikeline\mbox{} \DeprecationStart \stretchline\mbox{}}}
\newcommand{\EndDeprecateBlock}{%
  \mbox{}\stretchline\mbox{} \DeprecationEnd \strikeline}

\newenvironment{DeprecateBlock}{%
  \par \StartDeprecateBlock \par}{\par \EndDeprecateBlock \par}

\newcommand{\StartInlineDeprecate}{%
  \strikeline\mbox{} \DeprecationStart \strikeline \mbox{}}
\newcommand{\EndInlineDeprecate}{%
  \strikeline\mbox{} \DeprecationEnd \strikeline}
\newenvironment{DeprecateInline}{\StartInlineDeprecate}{\EndInlineDeprecate}

%
% Library API description template commands
%

\newcommand{\deprecationstart}{\color{red} \raisebox{.5ex}{\rule{1em}{.4pt}}
  deprecation start \xrfill[.5ex]{.4pt}[red] \mbox{}}
\newcommand{\deprecationend}{\mbox{}\xrfill[.5ex]{.4pt}[red]\mbox{} \color{red}
  deprecation end \raisebox{.5ex}{\rule{1em}{.4pt}}}

\newenvironment{deprecate}{\deprecationstart \\}{\\ \deprecationend}

\newcommand{\apisummary}[1]{
    #1
\hfill
}

\newenvironment{apidefinition}{
\begin{description}
\item[SYNOPSIS] \hfill \\ \\ 
\vspace{-2em}
}
{
\end{description}
}

\lstnewenvironment{Cpp11synopsis}
{
  \textbf{C++11:}
  \lstset{language={C++}, backgroundcolor=\color{gray}, lineskip=2pt,
  morekeywords={size_t, TYPE, noreturn}, aboveskip=0pt, belowskip=0pt}}{}

\lstnewenvironment{C11synopsis}
{ 
  \textbf{C11:} 
  \lstset{language={C}, backgroundcolor=\color{gray}, lineskip=2pt,
  morekeywords={size_t, ptrdiff_t, TYPE, _Noreturn, shmem_cmp_t},
  aboveskip=0pt, belowskip=0pt}}{}

\lstnewenvironment{CsynopsisCol}
{ 
  \lstset{language={C}, backgroundcolor=\color{gray}, lineskip=2pt,
  morekeywords={size_t, ptrdiff_t, TYPE, TYPENAME, SIZE, shmem_cmp_t},
  aboveskip=0pt, belowskip=0pt}}{}


\lstnewenvironment{Csynopsis}
{ 
  \textbf{C/C++:} 
  \lstset{language={C}, backgroundcolor=\color{gray}, lineskip=2pt,
  morekeywords={size_t, ptrdiff_t, TYPE, TYPENAME, SIZE, shmem_cmp_t},
  aboveskip=0pt, belowskip=0pt}}{}

\lstnewenvironment{CsynopsisST}
{ 
  \textbf{C/C++:} 
  \color{red}  
  {\lstset{language={C}, backgroundcolor=\color{gray}, lineskip=2pt,
  morekeywords={size_t}, aboveskip=0pt, belowskip=0pt}}
  }
  {}
  
\lstnewenvironment{Fsynopsis}
{ \deprecationstart \\
  \textbf{FORTRAN:}
  \lstset{language={Fortran}, backgroundcolor=\color{gray}, lineskip=3pt,
  deletekeywords=[2]{STATUS},
  deletekeywords=[3]{LOG}, aboveskip=0pt,
  belowskip=0pt}}
{ \deprecationend }

\newenvironment{apiarguments}{
\newcommand{\apiargument}[3]{
\begin{tabular}{p{2cm} p{2cm} p{10cm}}
\textbf{##1} & \textit{##2} & {##3} \\ 
\end{tabular}
}
\hfill
\item[DESCRIPTION] \hfill 

\begin{description}
\item[Arguments] \hfill \\
}
{
\hfill
\end{description}
}

\newcommand{\apidescription}[1]{
\begin{description}
\vspace{-1em}
\item[API description] \hfill \\ 
    #1
\hfill
}

\newcommand{\apidesctable}[4] {\hfill \\ #1 \\ \\
    \begin{tabular}{p{5cm} p{9cm}}
       \hline
       #2 & #3 \\
       \hline \tabularnewline
       \end{tabular}\\
        #4
}  

\newcommand{\apireturnvalues}[1]{
\hfill 
\item[Return Values] \hfill \\
    #1
\\
\hfill
}

\newcommand{\apitablerow}[2]{
 \begin{tabular}{p{5cm} p{9cm}}
 #1 & #2 \tabularnewline
  \end{tabular}\\
}

\newcommand{\apinotes}[1]{
\item[Notes] \hfill \\
    #1
\hfill \\
\end{description}
}

\newcommand{\apiimpnotes}[1]{
\begin{description}
\item[Note to implementors] \hfill \\
    #1
\hfill \\
\end{description}
}

\newenvironment{apiexamples}{
\newcommand{\apicexample}[3]{
    ##1
    \lstinputlisting[language={C}, tabsize=2,
      basicstyle=\ttfamily\footnotesize,
      morekeywords={size_t, ptrdiff_t, shmem_cmp_t}]{##2}
    ##3 }
\newcommand{\apifexample}[3]{
    ##1
    \lstinputlisting[language={Fortran}, tabsize=2,
    basicstyle=\ttfamily\footnotesize, deletekeywords={TARGET}]{##2}
    ##3 }
\vspace{-2pt}
\item[EXAMPLES] \hfill \\
\vspace{-2pt}
}
{
}

%
% End library API description template commands
%


\begin{document}

\input{content/frontmatter}



\section{The OpenSHMEM Effort}\label{subsec:openshmem_effort}
\input{content/the_openshmem_effort}

\section{Programming Model Overview}\label{subsec:programming_model}
\openshmem implements \ac{PGAS} by defining remotely accessible data objects as
mechanisms to share information among \openshmem processes or \acp{PE} and
private data objects that are accessible by the \ac{PE} itself. The \ac{API}
allows communication and synchronization operations on both private (local to
the PE initiating the operation) and remotely accessible data objects. The key
feature of \openshmem is that data transfer operations are
\textit{\textbf{one-sided}} in nature. This means that a local \ac{PE} executing
a data transfer routine does not require the participation of the remote \ac{PE}
to complete the routine. This allows for overlap between communication and
computation to hide data transfer latencies, which makes  \openshmem ideal for
unstructured, small/medium size data communication patterns. The \openshmem
library routines have the potential to provide a low-latency, high-bandwidth
communication \ac{API} for use in highly parallelized scalable programs.  

The \openshmem interfaces can be used to implement \ac{SPMD} style programs.
It provides interfaces to start the \openshmem \acp{PE} in parallel, and
communication and synchronization interfaces to access remotely accessible data
objects across \acp{PE}. These interfaces can be leveraged to divide a problem
into multiple sub-problems that can be solved independently or with coordination
using the communication and synchronization interfaces.  The \openshmem
specification defines library calls, constants, variables, and language bindings
for \Cstd and \Fortran.%
\footnote{As of \openshmem[1.4], the \Fortran interface has been deprecated
  and should be expected to be removed in a future release.}.
The \Cpp interface is currently the same as that
for \Cstd. Unlike UPC, \Fortran[2008], Titanium, X10 and Chapel, which are all
PGAS languages, \openshmem relies on the user to use the library calls  to
implement the correct semantics of its programming model.

An overview of the \openshmem routines is described below:

\begin{enumerate}

\item \textbf{Library Setup and Query}
\begin{enumerate}
  \item \OPR{Initialization}: The \openshmem library environment is initialized. 
  \item \OPR{Query}: The local \ac{PE} may get the number of \acp{PE} running
      the same program and its unique integer identifier. 
  \item \OPR{Accessibility}: The local \ac{PE} can find out if a remote \ac{PE} is
      executing the same binary, or if a particular symmetric data object can be
      accessed by a remote \ac{PE}, or may obtain a pointer to a symmetric data
      object on the specified remote \ac{PE} on shared memory systems.
\end{enumerate}

\item \textbf{Symmetric Data Object Management}
\begin{enumerate}
  \item \OPR{Allocation}: All executing \acp{PE} must participate in the
      allocation of a symmetric data object with identical arguments.
  \item  \OPR{Deallocation}: All executing \acp{PE} must participate in the
      deallocation of the same symmetric data object with identical arguments.
  \item  \OPR{Reallocation}: All executing \acp{PE} must participate in the
      reallocation of the same symmetric data object with identical arguments.
\end{enumerate}

\item \textbf{Remote Memory Access}
\begin{enumerate}
    \item \PUT: The local \ac{PE} specifies the \source{} data object (private
        or symmetric) that is copied to the symmetric data object on the remote
        \ac{PE}. 
  \item \GET: The local \ac{PE} specifies the symmetric data object on the remote
      \ac{PE} that is copied to a data object (private or symmetric) on the local
      \ac{PE}. 
\end{enumerate}

\item \textbf{Atomics}
\begin{enumerate}
    \item \OPR{Swap}: The \ac{PE} initiating the swap gets the old value of a
        symmetric data object from a remote \ac{PE} and copies a new value to
        that symmetric data object on the remote \ac{PE}.
  \item \OPR{Increment}: The \ac{PE} initiating the increment adds 1 to the
      symmetric data object on the remote \ac{PE}.
  \item \OPR{Add}: The \ac{PE} initiating the add specifies the value to be added
      to the symmetric data object on the remote \ac{PE}.
  \item \OPR{Compare and Swap}: The \ac{PE} initiating the swap gets the old value
      of the symmetric data object based on a value to be compared and copies a
      new value to the symmetric data object on the remote \ac{PE}.
  \item \OPR{Fetch and Increment}: The \ac{PE} initiating the increment adds 1 to
      the symmetric data object on the remote \ac{PE} and returns with the old
      value.
  \item \OPR{Fetch and Add}: The \ac{PE} initiating the add specifies the value to
      be added to the symmetric data object on the remote \ac{PE} and returns with
      the old value.
\end{enumerate}

\item \textbf{Synchronization and Ordering}
\begin{enumerate}
  \item \OPR{Fence}: The \ac{PE} calling fence ensures ordering of   
  \PUT, AMO, and memory store operations
  to symmetric data objects with respect to a specific
      destination \ac{PE}. 
  \item \OPR{Quiet}: The \ac{PE} calling quiet ensures completion of remote access
      operations and stores to symmetric data objects. 
  \item \OPR{Barrier}: All or some \acp{PE} collectively synchronize and ensure
      completion of all remote and local updates prior to any \ac{PE} returning
      from the call.
\end{enumerate}

\item \textbf{Collective Communication}
\begin{enumerate}
  \item \OPR{Broadcast}: The \textit{root} \ac{PE} specifies a symmetric data
      object to be copied to a symmetric data object on one or more remote
      \acp{PE} (not including itself). 
  \item \OPR{Collection}: All \acp{PE} participating in the routine get the result
      of concatenated symmetric objects contributed by each of the \acp{PE} in
      another symmetric data object.
  \item \OPR{Reduction}: All \acp{PE} participating in the routine get the result
      of an associative binary routine over elements of the specified symmetric
      data object on another symmetric data object. 
\end{enumerate}

\item \textbf{Mutual Exclusion}
\begin{enumerate}
  \item \OPR{Set Lock}: The \ac{PE} acquires exclusive access to the region
      bounded by the symmetric \textit{lock} variable.
  \item \OPR{Test Lock}: The \ac{PE} tests the symmetric \textit{lock} variable
      for availability.
  \item \OPR{Clear Lock}: The \ac{PE} which has previously acquired the
      \textit{lock} releases it.
\end{enumerate}

\item \textbf{Data Cache Control \textit{(deprecated)}}
\begin{enumerate}
  \item Implementation of mechanisms to exploit the capabilities of hardware cache
      if available.
\end{enumerate}
\end{enumerate}


\section{Memory Model}\label{subsec:memory_model}
\begin{figure}[h]
\includegraphics[width=0.95\textwidth]{figures/mem_model}
\caption{\openshmem Memory Model}
\label{fig:mem_model}
\end{figure}
%
An \openshmem program consists of data objects that are private to each \ac{PE}
and data  objects that are remotely accessible by all \acp{PE}. Private data
objects are stored in the local memory of each \ac{PE} and can only be accessed
by the \ac{PE} itself; these data objects cannot be accessed by other \acp{PE}
via \openshmem routines. Private data objects follow the memory model of
\Cstd. Remotely accessible objects, however, can be accessed by
remote \acp{PE} using \openshmem routines.  Remotely accessible data objects are
called \emph{Symmetric Data Objects}.  Each symmetric data object has a
corresponding object with the same name, type, and size on all \acp{PE} where that object is
accessible via the \openshmem \ac{API}\footnote{For efficiency reasons,
the same offset (from an arbitrary memory address) for symmetric data
objects might be used on all \acp{PE}. Further discussion about symmetric heap
layout and implementation efficiency can be found in Section~\ref{sec:memory_management}}.
(For the definition of what is accessible, see the
descriptions for \FUNC{shmem\_pe\_accessible} and \FUNC{shmem\_addr\_accessible}
in Sections~\ref{subsec:shmem_pe_accessible} and
\ref{subsec:shmem_addr_accessible}.) In \openshmem the following kinds of
data objects are symmetric:
%
\begin{itemize}
\item Global and static \Cstd and \Cpp variables. These data objects must
  not be defined in a dynamic shared object (DSO).
\item \Cstd and \Cpp data allocated by \openshmem memory management routines
  (Section~\ref{sec:memory_management})
\end{itemize}

\openshmem dynamic memory allocation routines (e.g.,
\FUNC{shmem\_malloc}) allow collective allocation of \emph{Symmetric Data
Objects} on a special memory region called the \emph{Symmetric Heap}. The
Symmetric Heap is created during the execution of a program at a memory location
determined by the implementation. The Symmetric Heap may reside in different
memory regions on different \acp{PE}.
Figure~\ref{fig:mem_model} shows an example \openshmem
memory layout, illustrating the location of remotely accessible symmetric
objects and private data objects.  As shown, symmetric data objects can be
located either in the symmetric heap or in the global/static memory section of
each \ac{PE}.

\subsection{Pointers to Symmetric Objects}\label{subsec:pointers_to_symmetric_objects}

Symmetric data objects are referenced in \openshmem operations through the
local pointer to the desired remotely accessible object.  The address contained
in this pointer is referred to as a {\em symmetric address}.  Every symmetric
address is also a {\em local address} that is valid for direct memory access;
however, not all local addresses are symmetric.  Manipulation of symmetric
addresses passed to \openshmem routines---including pointer arithmetic,
array indexing, and access of structure or union members---are permitted as long as
the resulting local pointer remains within the same symmetric allocation or
object.  Symmetric addresses are only valid at the \ac{PE} where they were
generated; using a symmetric address generated by a different \ac{PE} for
direct memory access or as an argument to an \openshmem routine results
in undefined behavior.

Symmetric addresses provided to typed and type-generic \openshmem interfaces
must be naturally aligned based on their type and any requirements of the
underlying architecture.  Symmetric addresses provided to fixed-size \openshmem
interfaces (e.g., \FUNC{shmem\_put32}) must also be aligned to the given
size.  Symmetric objects provided to fixed-size \openshmem interfaces
must have storage size equal to the bit-width of the given
operation\footnote{The bit-width of a byte is implementation-defined in \Cstd.  The
\CONST{CHAR\_BIT} constant in \HEADER{limits.h} can be used to portably
calculate the bit-width of a \Cstd object.}.  Because \CorCpp{} structures may
contain implementation-defined padding, the fixed-size interfaces should not be
used with \CorCpp{} structures.
The ``mem'' interfaces (e.g., \FUNC{shmem\_putmem}) have no alignment
requirements.

The \FUNC{shmem\_ptr} routine allows the programmer to query a {\em local
address} to a remotely accessible data object at a specified \ac{PE}.  The
resulting pointer is valid for direct memory access; however, providing this
address as an argument of an \openshmem routine that requires a symmetric
address results in undefined behavior.

\subsection{Atomicity Guarantees}\label{subsec:amo_guarantees}

\openshmem contains a number of routines that perform atomic operations on
symmetric data objects, which are defined in Section~\ref{sec:amo}.
The atomic routines
guarantee that concurrent accesses by any of these routines to the same
location, using the same datatype (specified in Tables~\ref{stdamotypes} and
\ref{extamotypes}), and using communication contexts (see Section~\ref{sec:ctx})
in the same atomicity domain will be exclusive.
Exclusivity is also guaranteed when the target \ac{PE} performs a wait or test
operation on the same location and with the same datatype as one or more atomic
operations.

An \openshmem \emph{atomicity domain} is a set of communication
contexts whose associated teams (see Section~\ref{subsec:team}) are
all split by (possibly recursive) calls to a
\FUNC{shmem\_team\_split\_*} routine from a common predefined team.
\openshmem defines two such predefined teams, \LibHandleRef{SHMEM\_TEAM\_WORLD}
and \LibHandleRef{SHMEM\_TEAM\_SHARED} (see Section~\ref{subsec:library_handles}).%
\footnote{
  Although all \acp{PE} in \LibHandleRef{SHMEM\_TEAM\_SHARED} are also
  in \LibHandleRef{SHMEM\_TEAM\_WORLD}, and a \ac{PE}'s number can be
  translated from its \LibHandleRef{SHMEM\_TEAM\_SHARED} to
  \LibHandleRef{SHMEM\_TEAM\_WORLD}, the
  \LibHandleRef{SHMEM\_TEAM\_SHARED} team is defined as not having
  been created by a call to a \FUNC{shmem\_team\_split\_*} routine on
  \LibHandleRef{SHMEM\_TEAM\_WORLD}.
  Therefore, the two teams are distinct predefined teams forming
  separate atomicity domains.
}

\openshmem atomic operations do not guarantee exclusivity in the following
scenarios, all of which result in undefined behavior.
\begin{enumerate}
    \item \label{amo-scenario/1}
        When concurrent accesses to the same location are performed using
        \openshmem atomic operations using communication contexts in
        different atomicity domains.
    \item \label{amo-scenario/2}
        When concurrent accesses to the same location are performed using
        \openshmem atomic operations using different datatypes.
    \item \label{amo-scenario/3}
        When atomic and non-atomic \openshmem operations are used to access
        the same location concurrently.
    \item \label{amo-scenario/4}
        When \openshmem atomic operations and non-\openshmem operations (e.g.,
        load and store operations) are used to access the same location
        concurrently.
\end{enumerate}

\SourceExample{./example_code/amo_scenario_1.c}{
  The following \CorCpp example illustrates scenario 1.
  In this example, different atomicity domains are used to access
  the same location, resulting in undefined behavior.
  The undefined behavior can be resolved by using communication
  contexts in the same atomicity domain in all concurrent operations.
}

\SourceExample{./example_code/amo_scenario_2.c}{
  The following \CorCpp example illustrates scenario 2.  In this example,
  different datatypes are used to access the same location concurrently,
  resulting in undefined behavior.  The undefined behavior can be resolved by
  using the same datatype in all concurrent operations.  For example, the
  32-bit value can be left-shifted and a 64-bit atomic OR operation can be
  used.
}

\SourceExample{./example_code/amo_scenario_3.c}{
  The following \CorCpp example illustrates scenario 3.  In this example,
  atomic increment operations are concurrent with a non-atomic reduction
  operation, resulting in undefined behavior.  The undefined behavior can be
  resolved by inserting a barrier operation before the reduction.  The
  barrier ensures that all local and remote \acp{AMO} have completed before the
  reduction operation accesses $x$.
}

\SourceExample{./example_code/amo_scenario_4.c}{
  The following \CorCpp example illustrates scenario 4.  In this example, an
  \openshmem atomic increment operation is concurrent with a local increment
  operation, resulting in undefined behavior.  The undefined behavior can be
  resolved by replacing the local increment operation with an \openshmem
  atomic increment.
}



\section{Execution Model}\label{subsec:execution_model}
\input{content/execution_model}

\section{Language Bindings and Conformance}\label{subsec:bindings}
\input{content/language_bindings_and_conformance}

\section{Library Constants}\label{subsec:library_constants}
The constants that start with SHMEM\_* are for both \Fortran
and \CorCpp, and they are compile-time constants. 
All constants that start with
\_SHMEM\_* are deprecated and provided for backwards compatibility.
\newline
\newline
\begin{tabular}{|p{0.4\textwidth}|p{0.5\textwidth}|}
\hline
\textbf{Constant} & \textbf{Description}
\tabularnewline
\hline 
\hline
%new
\vspace{3mm}
\vtop{\hbox{\CorCppFor:}
\hbox{\hspace*{12mm} \const{SHMEM\_SYNC\_SIZE}}}
& Length of a work array that can be used with any SHMEM collective
communication operation. The value of this constant is implementation
specific. Refer to the individual \hyperref[subsec:coll]{Collective Routines} for more information
about the usage of this constant. Work arrays sized for specific operations may
consume less memory.\tabularnewline
%new
\hline 
\vspace{3mm}
\vtop{\hbox{\CorCppFor:} 
\hbox{\hspace*{12mm} \const{SHMEM\_BCAST\_SYNC\_SIZE}}} 
& 
Length of the \VAR{pSync} arrays needed for broadcast routines. The value
of this constant is implementation specific. Refer to the
\hyperref[subsec:shmem_broadcast]{Broadcast Routines} section under
\hyperref[sec:openshmem_library_api]{Library Routines} for more information
about the usage of this constant. \tabularnewline
\hline 
\vspace{3mm}
\vtop{\hbox{\CorCppFor:} 
\hbox{\hspace*{12mm} \const{SHMEM\_SYNC\_VALUE}}} 
& 
The value used to initialize the elements of \VAR{pSync} arrays. The
value of this constant is implementation specific.\tabularnewline
\hline
\vspace{3mm}
\vtop{\hbox{\CorCppFor:} 
\hbox{\hspace*{12mm} \const{SHMEM\_REDUCE\_SYNC\_SIZE}}}
& 
Length of the work arrays needed for reduction routines. The value
of this constant is implementation specific. Refer to the
\hyperref[subsec:shmem_reductions]{Reduction Routines} section under
\hyperref[sec:openshmem_library_api]{Library Routines} for more information
about the usage of this constant.\tabularnewline
\hline
\vspace{3mm}
\vtop{\hbox{\CorCppFor:} 
\hbox{\hspace*{12mm} \const{SHMEM\_BARRIER\_SYNC\_SIZE}}} 
& 
Length of the work array needed for barrier routines. The value
of this constant is implementation specific. Refer to the
\hyperref[subsec:shmem_barrier]{Barrier Synchronization Routines} section under
\hyperref[sec:openshmem_library_api]{Library Routines}
for more information about the usage of this constant.\tabularnewline
\hline
\vspace{3mm}
\vtop{\hbox{\CorCppFor:}
\hbox{\hspace*{12mm} \const{SHMEM\_COLLECT\_SYNC\_SIZE}}} 
& 
Length of the work array needed for collect routines. The value
of this constant is implementation specific. Refer to the
\hyperref[subsec:shmem_collect]{Collect Routines} section under
\hyperref[sec:openshmem_library_api]{Library Routines} for more information
about the usage of this constant.\tabularnewline
\hline
\vspace{3mm}
\vtop{\hbox{\CorCppFor:}
\hbox{\hspace*{12mm} \const{SHMEM\_ALLTOALL\_SYNC\_SIZE}}} 
& 
Length of the work array needed for \FUNC{shmem\_alltoall}
routines. The value of this constant is implementation
specific. Refer to the \hyperref[subsec:shmem_alltoall]{Alltoall
routines} sections under \hyperref[sec:openshmem_library_api]{Library Routines}
for more information about the usage of this constant.\tabularnewline
\hline
\end{tabular}

\begin{tabular}{|p{0.4\textwidth}|p{0.5\textwidth}|}
\hline
\vspace{3mm}
\vtop{\hbox{\CorCppFor:}
\hbox{\hspace*{12mm} \const{SHMEM\_ALLTOALLS\_SYNC\_SIZE}}} 
& 
Length of the work array needed for \FUNC{shmem\_alltoalls}
routines. The value of this constant is implementation
specific. Refer to the \hyperref[subsec:shmem_alltoalls]{Alltoalls
routines} sections under \hyperref[sec:openshmem_library_api]{Library Routines}
for more information about the usage of this constant.\tabularnewline
\hline
\vspace{3mm}
\vtop{\hbox{\CorCppFor:} 
\hbox{\hspace*{12mm} \const{SHMEM\_REDUCE\_MIN\_WRKDATA\_SIZE}}} 
& Minimum length of work arrays used in various collective routines.\tabularnewline
\hline
\vspace{3mm}
%\color{red}
%\vtop{\hbox{} 
%\hbox{\hspace*{12mm} \const{}} 
%\hbox{} 
%\hbox{\hspace*{12mm} \const{}}} 
%& \color{red}
%Ticket \#107 \tabularnewline
\vtop{\hbox{\CorCppFor:} 
\hbox{\hspace*{12mm} \const{SHMEM\_MAJOR\_VERSION}}}
& 
Integer representing the major version of \openshmem standard in use. \tabularnewline
\hline
\vspace{3mm}
\vtop{\hbox{\CorCppFor:} 
\hbox{\hspace*{12mm} \const{SHMEM\_MINOR\_VERSION}}}
& 
Integer representing the minor version of \openshmem standard in use. \tabularnewline
\hline
\vspace{3mm}
\vtop{\hbox{\CorCppFor:} 
\hbox{\hspace*{12mm} \const{SHMEM\_MAX\_NAME\_LEN}}}
&
Integer representing the maximum length of \const{SHMEM\_VENDOR\_STRING}. \tabularnewline
\hline
\vspace{3mm}
\vtop{\hbox{\CorCppFor:} 
\hbox{\hspace*{12mm} \const{SHMEM\_VENDOR\_STRING}}} 
&
String representing vendor defined information of size at most
\const{SHMEM\_MAX\_NAME\_LEN}.
In \CorCpp{}, the string is terminated by a null character.  In \Fortran, the
string of size less than \const{SHMEM\_MAX\_NAME\_LEN} is padded with blank
characters up to size \const{SHMEM\_MAX\_NAME\_LEN}. \tabularnewline
\hline

\end{tabular}
\color{black}


\section{Library Handles}\label{subsec:library_handles}
\input{content/library_handles}

\section{Environment Variables }\label{subsec:environment_variables}
\input{content/environment_variables}

\clearpage



\section{OpenSHMEM Library \acs{API}}\label{sec:openshmem_library_api}

\subsection{Library Setup, Exit, and Query Routines}
The library setup and query interfaces that initialize and monitor the parallel
environment of the \acp{PE}.

\subsubsection{\textbf{SHMEM\_INIT}}\label{subsec:shmem_init}
\input{content/shmem_init}

\subsubsection{\textbf{SHMEM\_MY\_PE}}\label{subsec:shmem_my_pe}
\input{content/shmem_my_pe}

\subsubsection{\textbf{SHMEM\_N\_PES}}\label{subsec:shmem_n_pes}
\input{content/shmem_n_pes}

\subsubsection{\textbf{SHMEM\_FINALIZE}}\label{subsec:shmem_finalize}
\input{content/shmem_finalize}

\subsubsection{\textbf{SHMEM\_INITIALIZED}}\label{subsec:shmem_initialized}
\apisummary{
  Indicates whether the \openshmem library has been initialized.
}

\begin{apidefinition}

\begin{Csynopsis}
int @\FuncDecl{shmem\_initialized}@(void);
\end{Csynopsis}

\begin{apiarguments}
  \apiargument{None}{}{}
\end{apiarguments}

\apidescription{
  The \FUNC{shmem\_initialized} routine returns a value indicating
  whether the \openshmem library has been initialized (i.e, a call to
  either \FUNC{shmem\_init} or \FUNC{shmem\_init\_thread} has
  completed successfully).
  This routine may be called at any point in an \openshmem program,
  including before \FUNC{shmem\_init[\_thread]} and after
  \FUNC{shmem\_finalize}.
  This routine may be called by any thread of execution in the
  \ac{PE}, independent of the level of thread support provided by the
  implementation.
}

\apireturnvalues{
  Returns \CONST{1} if the \openshmem library has been initialized;
  otherwise, returns \CONST{0}.
}

\end{apidefinition}


\subsubsection{\textbf{SHMEM\_FINALIZED}}\label{subsec:shmem_finalized}
\apisummary{
  Indicates whether the \openshmem library has been finalized.
}

\begin{apidefinition}

\begin{Csynopsis}
int @\FuncDecl{shmem\_finalized}@(void);
\end{Csynopsis}

\begin{apiarguments}
  \apiargument{None}{}{}
\end{apiarguments}

\apidescription{
  The \FUNC{shmem\_finalized} routine returns a value indicating
  whether the \openshmem library has been finalized (i.e, a call to
  \FUNC{shmem\_finalize} has completed successfully).
  This routine may be called at any point in an \openshmem program,
  including before \FUNC{shmem\_init[\_thread]} and after
  \FUNC{shmem\_finalize}.
  This routine may be called by any thread of execution in the
  \ac{PE}, independent of the level of thread support provided by the
  implementation.
}

\apireturnvalues{
  Returns \CONST{1} if the \openshmem library has been finalized;
  otherwise, returns \CONST{0}.
}

\end{apidefinition}


\subsubsection{\textbf{SHMEM\_GLOBAL\_EXIT}}\label{subsec:shmem_global_exit}
\input{content/shmem_global_exit}

\subsubsection{\textbf{SHMEM\_PE\_ACCESSIBLE}}\label{subsec:shmem_pe_accessible}
\input{content/shmem_pe_accessible}

\subsubsection{\textbf{SHMEM\_ADDR\_ACCESSIBLE}}\label{subsec:shmem_addr_accessible}
\input{content/shmem_addr_accessible}

\subsubsection{\textbf{SHMEM\_PTR}}\label{subsec:shmem_ptr}
\apisummary{
    Returns a pointer to a data object on a specified \ac{PE}.
}

\begin{apidefinition}

\begin{Csynopsis}
void *shmem_ptr(const void *dest, int pe);
\end{Csynopsis}

\begin{Fsynopsis}
POINTER (PTR, POINTEE)
INTEGER pe
PTR = SHMEM_PTR(dest, pe)
\end{Fsynopsis}


\begin{apiarguments}
\apiargument{IN}{dest}{The symmetric data object to be referenced.}
\apiargument{IN}{pe}{An integer that indicates the \ac{PE} number on which \dest{} is to
		 be accessed.  When using \Fortran, it must be a  default
		 integer value.}
\end{apiarguments}

\apidescription{
    \FUNC{shmem\_ptr} returns an address that may be used to directly reference
    \dest{} on the specified \ac{PE}.  This address can be assigned to a pointer.
    After that, ordinary loads and stores to this remote address may be performed.
    
    When a sequence of loads (gets) and stores (puts) to a data object on a
    remote \ac{PE} does not match the access pattern provided in an \openshmem data
    transfer routine like \FUNC{shmem\_put32} or \FUNC{shmem\_real\_iget}, the
    \FUNC{shmem\_ptr} routine can provide an efficient means to accomplish the
    communication.
}

\apireturnvalues{
    The return value is a non-NULL address of the \dest{} data object when it is 
    accessible using memory loads and stores in addition to \openshmem operations.
    Otherwise, a NULL address is returned.
}

\apinotes{
    When calling \FUNC{shmem\_ptr}, \dest{} is the address of the referenced
    symmetric data object on the calling \ac{PE}.
}

\begin{apiexamples}

\apifexample
    { This  \Fortran  program calls \FUNC{shmem\_ptr} and then \ac{PE} 0 writes to
    the \VAR{BIGD} array on \ac{PE} 1: }
    {./example_code/shmem_ptr_example.f90 }
    {}
    
\apicexample
    {This is the equivalent program written in \Cstd[11]:}
    {./example_code/shmem_ptr_example.c}
    {}

\end{apiexamples}

\end{apidefinition}


\subsubsection{\textbf{SHMEM\_INFO\_GET\_VERSION}}\label{subsec:shmem_info_get_version}
\apisummary{
    Returns the major and minor version of the library implementation.
}

\begin{apidefinition}

\begin{Csynopsis}
void shmem_info_get_version(int *major, int *minor);
\end{Csynopsis}

\begin{Fsynopsis}
INTEGER MAJOR, MINOR
SHMEM_INFO_GET_VERSION(MAJOR, MINOR)   
\end{Fsynopsis}

\begin{apiarguments}
    \apiargument{OUT}{major}{The major version of the \openshmem standard in use.}
    \apiargument{OUT}{minor}{The minor version of the \openshmem standard in use.}
\end{apiarguments}

\apidescription{
    This routine returns the major and minor version of the \openshmem standard
    in use.  For a given library implementation, the major and minor version
    returned by these calls are consistent with the library constants
    \CONST{SHMEM\_MAJOR\_VERSION} and \CONST{SHMEM\_MINOR\_VERSION}.
}

\apireturnvalues{
    None.
}

\apinotes{
    None. 
}

\end{apidefinition}


\subsubsection{\textbf{SHMEM\_INFO\_GET\_NAME}}\label{subsec:shmem_info_get_name}
\apisummary{
    This routine returns the vendor defined character string.
}

\begin{apidefinition}

\begin{Csynopsis}
void shmem_info_get_name(char *name);
\end{Csynopsis}

\begin{Fsynopsis}
CHARACTER *(*)NAME
SHMEM_INFO_GET_NAME(NAME)   
\end{Fsynopsis}

\begin{apiarguments}
    \apiargument{OUT}{name}{The vendor defined string.}
\end{apiarguments}

\apidescription{
    This routine returns the vendor defined character string of size defined by
    the library constant \CONST{SHMEM\_MAX\_NAME\_LEN}. The program calling
    this function prepares the \VAR{name} memory buffer of at least size
    \CONST{SHMEM\_MAX\_NAME\_LEN}. The implementation copies the vendor defined
    string of size at most \CONST{SHMEM\_MAX\_NAME\_LEN} to \VAR{name}. In
    \CorCpp{}, the string is terminated by a null character.  In \Fortran,
    the string of size less than \CONST{SHMEM\_MAX\_NAME\_LEN} is padded with
    blank characters up to size \CONST{SHMEM\_MAX\_NAME\_LEN}. If the
    \VAR{name} memory buffer is provided with size less than
    \CONST{SHMEM\_MAX\_NAME\_LEN}, behavior is undefined. For a given library
    implementation, the vendor string returned is consistent with the library
    constant \CONST{SHMEM\_VENDOR\_STRING}.
}

\apireturnvalues{ 
    None. 
}

\apinotes{ 
    None. 
}

\end{apidefinition}


\subsubsection{\textbf{START\_PES}}\label{subsec:start_pes}
\input{content/start_pes}

\subsection{Thread Support}
\label{subsec:thread_support}
\input{content/threads_intro.tex}

\subsubsection{\textbf{SHMEM\_INIT\_THREAD}}
\label{subsec:shmem_init_thread}
\input{content/shmem_init_thread}

\subsubsection{\textbf{SHMEM\_QUERY\_THREAD}}
\label{subsec:shmem_query_thread}
\input{content/shmem_query_thread}


\subsection{Memory Management Routines}
\label{sec:memory_management}

\openshmem provides a set of \acp{API} for managing the symmetric heap. The
\acp{API} allow one to dynamically allocate, deallocate, reallocate and align
symmetric data objects in the symmetric heap.

\subsubsection{\textbf{SHMEM\_MALLOC, SHMEM\_FREE, SHMEM\_REALLOC, SHMEM\_ALIGN}}\label{subsec:shfree}
\apisummary{
    Symmetric heap memory management routines.
}

\begin{apidefinition}

\begin{Csynopsis}
void *shmem_malloc(size_t size);
void shmem_free(void *ptr);
void *shmem_realloc(void *ptr, size_t size);
void *shmem_align(size_t alignment, size_t size);
\end{Csynopsis}

\begin{apiarguments}
    \apiargument{IN}{size}{The size, in bytes, of a block to be
        allocated from the symmetric heap. This argument is of type \VAR{size\_t}}
    \apiargument{IN}{ptr}{Points to a block within the symmetric heap.}
    \apiargument{IN}{alignment}{Byte alignment of the block allocated from the
        symmetric heap.}
\end{apiarguments}


\apidescription{
    The \FUNC{shmem\_malloc} routine returns a pointer to a block of at least
    \VAR{size} bytes suitably aligned for any use.  This space is allocated from the
    symmetric heap (in contrast to \FUNC{malloc}, which allocates from the private
    heap).
    
    The \FUNC{shmem\_align} routine allocates a block in the symmetric heap that has
    a byte alignment specified by the alignment argument.
    
    The \FUNC{shmem\_free} routine causes the block to which \VAR{ptr} points to be
    deallocated, that is, made available for further allocation.  If \VAR{ptr} is a
    null pointer, no action occurs. 
           
    The \FUNC{shmem\_realloc} routine changes the size of the block to which
    \VAR{ptr} points to the size (in bytes) specified by \VAR{size}.  The contents
    of the block are unchanged up to the lesser of the new and old sizes. If the new
    size is larger, the newly allocated portion of the block is
    uninitialized.  If \VAR{ptr} is a \CONST{NULL} pointer, the
    \FUNC{shmem\_realloc} routine behaves like the \FUNC{shmem\_malloc} routine for
    the specified size.  If \VAR{size} is \CONST{0} and \VAR{ptr} is not a
    \CONST{NULL} pointer, the block to which it points is freed. If the space cannot
    be allocated, the block to which \VAR{ptr} points is unchanged.
    
    The \FUNC{shmem\_malloc}, \FUNC{shmem\_align}, \FUNC{shmem\_free}, and \FUNC{shmem\_realloc} routines
    are provided  so that multiple \acp{PE} in a program can allocate symmetric,
    remotely accessible memory blocks.  These memory blocks can then be used with
    \openshmem communication routines.  Each of these routines include at least one
    call to a procedure that is semantically equivalent to \FUNC{shmem\_barrier\_all}:
    \FUNC{shmem\_malloc} and \FUNC{shmem\_align} call a
    barrier on exit; \FUNC{shmem\_free} calls a barrier on entry; and
    \FUNC{shmem\_realloc} may call barriers on both entry and exit, depending on
    whether an existing allocation is modified and whether new memory is allocated.
    This ensures that all
    \acp{PE} participate in the memory allocation, and that the memory on other
    \acp{PE} can be used as  soon as the local \ac{PE} returns.  The user is
    responsible for calling these routines with identical argument(s) on all
    \acp{PE}; if differing \VAR{size} arguments are used, the behavior of the call
    and any subsequent \openshmem calls becomes undefined.
}

\apireturnvalues{
    The \FUNC{shmem\_malloc} routine returns a pointer to the allocated space;
    otherwise, it returns a \CONST{NULL} pointer.
    
    The \FUNC{shmem\_free} routine returns no value.
    
    The \FUNC{shmem\_realloc} routine returns a pointer to the allocated space
    (which may have moved); otherwise, it returns a null pointer.
    
    The \FUNC{shmem\_align} routine returns an aligned pointer to the allocated
    space; otherwise, it returns a \CONST{NULL} pointer.
}

\apinotes{ 
    As of Specification 1.2 the use of \FUNC{shmalloc}, \FUNC{shmemalign},
    \FUNC{shfree},  and \FUNC{shrealloc} has been deprecated. Although OpenSHMEM
    libraries are required to support the calls, program users are encouraged to use
    \FUNC{shmem\_malloc}, \FUNC{shmem\_align}, \FUNC{shmem\_free}, and
    \FUNC{shmem\_realloc} instead.  The behavior and signature  of the routines
    remains unchanged from the deprecated versions.
    					 
    The total size of the symmetric heap is determined at job startup.  One can
    adjust the size of the heap using the \CONST{SHMEM\_SYMMETRIC\_SIZE} environment
    variable (where available).	
    
    The \FUNC{shmem\_malloc}, \FUNC{shmem\_free}, and \FUNC{shmem\_realloc} routines
    differ from the private heap allocation routines in that all \acp{PE} in a
    program must call them (a barrier is used to ensure this).
}		

\apiimpnotes{
    The symmetric heap allocation routines always return a pointer to corresponding
    symmetric objects across all PEs. The \openshmem specification does not
    require that the virtual addresses are equal across all \acp{PE}. Nevertheless,
    the implementation must avoid costly address translation operations in the
    communication path, including order $N$ (where $N$ is the number of \acp{PE})
    memory translation tables.  In order to avoid address translations, the
    implementation may re-map the allocated block of memory based on agreed virtual
    address.  Additionally, some operating systems provide an option to disable
    virtual address randomization, which enables predictable allocation of virtual
    memory addresses.
}

\end{apidefinition}


\subsubsection{\textbf{SHMEM\_MALLOC\_WITH\_HINTS}}\label{subsec:shmmallochint}
\input{content/shmem_malloc_hints.tex}

\subsubsection{\textbf{SHMEM\_CALLOC}}\label{subsec:shmem_calloc}
\input{content/shmem_calloc.tex}



\subsection{Team Management Routines}\label{subsec:team}
\input{content/teams_intro.tex}

\subsubsection{\textbf{SHMEM\_TEAM\_MY\_PE}}\label{subsec:shmem_team_my_pe}
\input{content/shmem_team_my_pe.tex}

\subsubsection{\textbf{SHMEM\_TEAM\_N\_PES}}\label{subsec:shmem_team_n_pes}
\input{content/shmem_team_n_pes.tex}

\subsubsection{\textbf{SHMEM\_TEAM\_CONFIG\_T}}
\label{subsec:shmem_team_config_t}
\input{content/shmem_team_config_t.tex}

\subsubsection{\textbf{SHMEM\_TEAM\_GET\_CONFIG}}\label{subsec:shmem_team_get_config}
\input{content/shmem_team_get_config.tex}

\subsubsection{\textbf{SHMEM\_TEAM\_TRANSLATE\_PE}}\label{subsec:shmem_team_translate_pe}
\input{content/shmem_team_translate_pe.tex}

\subsubsection{\textbf{SHMEM\_TEAM\_SPLIT\_STRIDED}}\label{subsec:shmem_team_split_strided}
\input{content/shmem_team_split_strided.tex}

\subsubsection{\textbf{SHMEM\_TEAM\_SPLIT\_2D}}\label{subsec:shmem_team_split_2d}
\input{content/shmem_team_split_2d.tex}

\subsubsection{\textbf{SHMEM\_TEAM\_DESTROY}}\label{subsec:shmem_team_destroy}
\input{content/shmem_team_destroy.tex}



\subsection{Communication Management Routines}\label{sec:ctx}
\input{content/context_intro.tex}

\subsubsection{\textbf{SHMEM\_CTX\_CREATE}}
\label{subsec:shmem_ctx_create}
\input{content/shmem_ctx_create.tex}

\subsubsection{\textbf{SHMEM\_TEAM\_CREATE\_CTX}}
\label{subsec:shmem_team_create_ctx}
\input{content/shmem_team_create_ctx.tex}

\subsubsection{\textbf{SHMEM\_CTX\_DESTROY}}
\label{subsec:shmem_ctx_destroy}
\input{content/shmem_ctx_destroy.tex}

\subsubsection{\textbf{SHMEM\_CTX\_GET\_TEAM}}
\label{subsec:shmem_ctx_get_team}
\input{content/shmem_ctx_get_team.tex}


\subsection{Remote Memory Access Routines}\label{sec:rma}
The \ac{RMA} routines described in this section are one-sided communication
mechanisms of the \openshmem \ac{API}. While using these mechanisms, the user
is required to provide parameters only on the calling side. A characteristic of
one-sided communication is that it decouples communication from the
synchronization. One-sided communication mechanisms transfer the data but do not
synchronize the sender of the data with the receiver of the data. 

\openshmem \ac{RMA} routines are all performed on the symmetric objects.  The
initiator \ac{PE} of the call is designated as \source{}, and the \ac{PE} in
which memory is accessed is designated as \dest{}. In the case of the remote
update routine, \PUT{}, the origin is the \source{} \ac{PE} and the destination
\ac{PE} is the \dest{} PE. In the case of the remote read routine, \GET{}, the
origin is the \dest{} \ac{PE} and the destination is the \source{} \ac{PE}.

Where appropriate compiler support is available, \openshmem provides type-generic 
one-sided communication interfaces via \Cstd[11] generic selection
(\Cstd[11]~\S6.5.1.1\footnote{Formally, the \Cstd[11] specification is ISO/IEC 9899:2011(E).})
for block, scalar, and block-strided put and get communication. 
Such type-generic routines are supported for the ``standard \ac{RMA} types''
listed in Table \ref{stdrmatypes}.

The standard \ac{RMA} types include the exact-width integer types defined in
\HEADER{stdint.h} by \Cstd[99]%
\footnote{Formally, the \Cstd[99] specification is ISO/IEC~9899:1999(E).}%
~\S7.18.1.1 and \Cstd[11]~\S7.20.1.1. When the \Cstd translation environment
does not provide exact-width integer types with \HEADER{stdint.h}, an
\openshmem implemementation is not required to provide support for these types.

\begin{table}[h]
  \begin{center}
    \begin{tabular}{|l|l|}
      \hline
      \TYPE              & \TYPENAME  \\ \hline
      float              & float      \\ \hline
      double             & double     \\ \hline
      long double        & longdouble \\ \hline
      char               & char       \\ \hline
      signed char        & schar      \\ \hline
      short              & short      \\ \hline
      int                & int        \\ \hline
      long               & long       \\ \hline
      long long          & longlong   \\ \hline
      unsigned char      & uchar      \\ \hline
      unsigned short     & ushort     \\ \hline
      unsigned int       & uint       \\ \hline
      unsigned long      & ulong      \\ \hline
      unsigned long long & ulonglong  \\ \hline
      int8\_t            & int8       \\ \hline
      int16\_t           & int16      \\ \hline
      int32\_t           & int32      \\ \hline
      int64\_t           & int64      \\ \hline
      uint8\_t           & uint8      \\ \hline
      uint16\_t          & uint16     \\ \hline
      uint32\_t          & uint32     \\ \hline
      uint64\_t          & uint64     \\ \hline
      size\_t            & size       \\ \hline
      ptrdiff\_t         & ptrdiff    \\ \hline
    \end{tabular}
    \caption{Standard \ac{RMA} Types and Names}
    \label{stdrmatypes}
  \end{center} 
\end{table}


\subsubsection{Blocking Remote Memory Access Routines}\label{subsec:rma}
\subsubsubsection{\textbf{SHMEM\_PUT}}\label{subsec:shmem_put}
\apisummary{
    The  put routines  provide  a method for copying data from a contiguous local
    data object to a data object on a specified \ac{PE}.
}

\begin{apidefinition}

\begin{C11synopsis}
void shmem_put(TYPE *dest, const TYPE *source, size_t nelems, int pe);
\end{C11synopsis}
where \TYPE{} is one of the standard \ac{RMA} types specified by Table \ref{stdrmatypes}.

\begin{Csynopsis}
void shmem_<TYPENAME>_put(TYPE *dest, const TYPE *source, size_t nelems, int pe);
\end{Csynopsis}
where \TYPE{} is one of the standard \ac{RMA} types and has a corresponding \TYPENAME{} specified by Table \ref{stdrmatypes}.

\begin{CsynopsisCol}
void shmem_put<SIZE>(void *dest, const void *source, size_t nelems, int pe);
\end{CsynopsisCol}
where \SIZE{} is one of \CONST{8, 16, 32, 64, 128}.

\begin{CsynopsisCol}
void shmem_putmem(void *dest, const void *source, size_t nelems, int pe);
\end{CsynopsisCol}

\begin{Fsynopsis}
CALL SHMEM_CHARACTER_PUT(dest, source, nelems, pe)
CALL SHMEM_COMPLEX_PUT(dest, source, nelems, pe)
CALL SHMEM_DOUBLE_PUT(dest, source, nelems, pe)
CALL SHMEM_INTEGER_PUT(dest, source, nelems, pe)
CALL SHMEM_LOGICAL_PUT(dest, source, nelems, pe)
CALL SHMEM_PUT4(dest, source, nelems, pe)
CALL SHMEM_PUT8(dest, source, nelems, pe)
CALL SHMEM_PUT32(dest, source, nelems, pe)
CALL SHMEM_PUT64(dest, source, nelems, pe)
CALL SHMEM_PUT128(dest, source, nelems, pe)
CALL SHMEM_PUTMEM(dest, source, nelems, pe)
CALL SHMEM_REAL_PUT(dest, source, nelems, pe)
\end{Fsynopsis}

\begin{apiarguments}
    \apiargument{IN}{dest}{Data object to be updated on the remote \ac{PE}. This
    data object must be remotely accessible.}
    \apiargument{IN}{source}{Data object containing the data to be copied.}
    \apiargument{IN}{nelems}{Number of elements in the \VAR{dest} and \VAR{source}
    arrays. \VAR{nelems} must be of type \VAR{size\_t} for \Cstd. When using
    \Fortran, it must be a constant, variable, or array element of default
    integer type.}
    \apiargument{IN}{pe}{\ac{PE} number of the remote \ac{PE}. \VAR{pe} must be
    of type integer. When using \Fortran, it must be a constant, variable,
    or array element of default integer type.}
\end{apiarguments}

\apidescription{
    The routines return after the data has been copied out of the \source{} array
    on the local \ac{PE}.  The delivery of data words into the data object on the
    destination \ac{PE} may occur in any order.  Furthermore, two successive put
    routines may deliver data out of order unless a call to \FUNC{shmem\_fence} is
    introduced between the two calls.   
 }

\apidesctable{
    The \dest{} and \source{} data objects must conform to certain typing
    constraints, which are as follows:}
    {Routine}{Data type of \VAR{dest} and \VAR{source}}
    \apitablerow{shmem\_putmem}{\Fortran: Any noncharacter type. \Cstd: Any
        data  type.  nelems is scaled in bytes.}
    \apitablerow{shmem\_put4, shmem\_put32}{Any noncharacter type
        that has a storage size equal to \CONST{32} bits.}
    \apitablerow{shmem\_put8}{\Cstd: Any noncharacter type that
        has a storage size equal to \CONST{8} bits.}
    \apitablerow{}{\Fortran: Any noncharacter type that
        has a storage size equal to \CONST{64} bits.}
    \apitablerow{shmem\_put64}{Any noncharacter type that
        has a storage size equal to \CONST{64} bits.}
    \apitablerow{shmem\_put128}{Any noncharacter type that has a
        storage size equal to \CONST{128} bits.}
    \apitablerow{SHMEM\_CHARACTER\_PUT}{Elements of type character.  \VAR{nelems}
    is  the number  of	 characters to transfer. The actual character lengths of
    the \source{} and \dest{} variables are ignored. }
    \apitablerow{SHMEM\_COMPLEX\_PUT}{Elements of type complex of default size.}
    \apitablerow{SHMEM\_DOUBLE\_PUT}{Elements of type double precision. }
    \apitablerow{SHMEM\_INTEGER\_PUT}{Elements of type integer.}
    \apitablerow{SHMEM\_LOGICAL\_PUT}{Elements of type logical.}
    \apitablerow{SHMEM\_REAL\_PUT}{Elements of type real.}

\apireturnvalues{
    None.
}
\apinotes{
    When using \Fortran, data types must be of default size.  For example,
    a real variable must be declared as \CONST{REAL},  \CONST{REAL*4},  or
    \CONST{REAL(KIND=KIND(1.0))}. The Fortran API routine \FUNC{SHMEM\_PUT} has
    been deprecated, and either \FUNC{SHMEM\_PUT8} or \FUNC{SHMEM\_PUT64} should
    be used in its place.
}

\begin{apiexamples}

\apicexample
    { The following \FUNC{shmem\_put} example is for \Cstd[11] programs:}
    {./example_code/shmem_put_example.c}
    {} 
\end{apiexamples}

\end{apidefinition}


\subsubsubsection{\textbf{SHMEM\_P}}\label{subsec:shmem_p}
\apisummary{
    Copies one data item to a remote \ac{PE}.
}

\begin{apidefinition}

\begin{C11synopsis}
void shmem_p(TYPE *dest, TYPE value, int pe);
\end{C11synopsis}
where \TYPE{} is one of the standard \ac{RMA} types specified by Table \ref{stdrmatypes}.

\begin{Csynopsis}
void shmem_<TYPENAME>_p(TYPE *dest, TYPE value, int pe);
\end{Csynopsis}
where \TYPE{} is one of the standard \ac{RMA} types and has a corresponding \TYPENAME{} specified by Table \ref{stdrmatypes}.

\begin{apiarguments}
    \apiargument{IN}{dest}{The remotely accessible array element or scalar data object
    which will receive the data on the remote \ac{PE}.}
    \apiargument{IN}{value}{The value to be transferred to \VAR{dest} on the
    remote \ac{PE}.}
    \apiargument{IN}{pe}{The number of the remote \ac{PE}.}
\end{apiarguments}

\apidescription{
    These routines provide a very low latency put capability for single elements of
    most basic types.
    
    As with \FUNC{shmem\_put}, these routines start the remote transfer and may
    return before the data is delivered to the remote \ac{PE}.  Use
    \FUNC{shmem\_quiet} to force completion of all remote \PUT{} transfers.
}

\apireturnvalues{
    None.
}

\apinotes{
    None.
}

\begin{apiexamples}

    \apicexample
    {The following example uses \FUNC{shmem\_p} in a \Cstd[11] program.}
    {./example_code/shmem_p_example.c}
    {}

\end{apiexamples}

\end{apidefinition}


\subsubsubsection{\textbf{SHMEM\_IPUT}}\label{subsec:shmem_iput}
\apisummary{
    Copies strided data to a specified \ac{PE}.
}

\begin{apidefinition}

\begin{C11synopsis}
void shmem_iput(TYPE *dest, const TYPE *source, ptrdiff_t dst, ptrdiff_t sst, size_t nelems, int pe);
\end{C11synopsis}
where \TYPE{} is one of the standard \ac{RMA} types specified by Table \ref{stdrmatypes}.

\begin{Csynopsis}
void shmem_<TYPENAME>_iput(TYPE *dest, const TYPE *source, ptrdiff_t dst, ptrdiff_t sst, size_t nelems, int pe);
\end{Csynopsis}
where \TYPE{} is one of the standard \ac{RMA} types and has a corresponding \TYPENAME{} specified by Table \ref{stdrmatypes}.

\begin{CsynopsisCol}
void shmem_iput<SIZE>(void *dest, const void *source, ptrdiff_t dst, ptrdiff_t sst, size_t nelems, int pe);
\end{CsynopsisCol}
where \SIZE{} is one of \CONST{8, 16, 32, 64, 128}.

\begin{Fsynopsis}
INTEGER dst, sst, nelems, pe
CALL SHMEM_COMPLEX_IPUT(dest, source, dst, sst, nelems, pe)
CALL SHMEM_DOUBLE_IPUT(dest, source, dst, sst, nelems, pe)
CALL SHMEM_INTEGER_IPUT(dest, source, dst, sst, nelems, pe)
CALL SHMEM_IPUT4(dest, source, dst, sst, nelems, pe)
CALL SHMEM_IPUT8(dest, source, dst, sst, nelems, pe)
CALL SHMEM_IPUT32(dest, source, dst, sst, nelems, pe)
CALL SHMEM_IPUT64(dest, source, dst, sst, nelems, pe)
CALL SHMEM_IPUT128(dest, source, dst, sst, nelems, pe)
CALL SHMEM_LOGICAL_IPUT(dest, source, dst, sst, nelems, pe)
CALL SHMEM_REAL_IPUT(dest, source, dst, sst, nelems, pe)
\end{Fsynopsis}

\begin{apiarguments}
    \apiargument{OUT}{dest}{Array to be updated on the remote \ac{PE}. This data
        object  must be remotely accessible.}
    \apiargument{IN}{source}{Array containing the data to be copied.}
    \apiargument{IN}{dst}{The stride between consecutive elements of the \dest{}
        array.  The stride is scaled by the element size of the \dest{} array.  A
        value of \CONST{1} indicates contiguous data.  \VAR{dst} must be of type
        \CTYPE{ptrdiff\_t}.  When using \Fortran, it must be a default integer value.}
    \apiargument{IN}{sst}{The  stride between consecutive elements of the
        \source{} array.  The stride is scaled by the element size of the \source{}
        array.  A  value of \CONST{1} indicates contiguous data.  \VAR{sst} must be
        of type \CTYPE{ptrdiff\_t}.  When using \Fortran, it must be a
        default integer value.}
    \apiargument{IN}{nelems}{Number of elements in the \dest{} and \source{}
        arrays.  \VAR{nelems} must be of type \VAR{size\_t} for \Cstd.  When
        using \Fortran, it must be  a constant, variable, or array element of
        default integer type.}
    \apiargument{IN}{pe}{\ac{PE} number of the remote \ac{PE}.  \VAR{pe} must be
        of type integer.   When using  \Fortran, it must be a constant,
        variable, or array element of default integer type.}
\end{apiarguments}


\apidescription{
    The \FUNC{iput} routines provide a method  for  copying strided data
    elements (specified by \VAR{sst}) of an array from a \source{} array on the
    local \ac{PE} to locations specified by stride \VAR{dst} on a \dest{} array
    on specified remote \ac{PE}. Both strides, \VAR{dst} and \VAR{sst}, must be
    greater than or equal to \CONST{1}. The routines return when the data has
    been copied out of the \VAR{source} array on the local \ac{PE} but not
    necessarily before the data has been delivered to the remote data object.
}

\apidesctable{
    The \dest{} and \source{} data objects must conform to typing constraints,
    which are as follows:
}{Routine}{Data type of \VAR{dest} and \VAR{source}}
    \apitablerow{shmem\_iput4, shmem\_iput32}{Any noncharacter type
        that has a storage size equal to \CONST{32} bits.}
    \apitablerow{shmem\_iput8}{\Cstd: Any noncharacter type that
        has a storage size equal to \CONST{8} bits.}
    \apitablerow{}{\Fortran: Any noncharacter type that
        has a storage size equal to \CONST{64} bits.}
    \apitablerow{shmem\_iput64}{Any noncharacter type that
        has a storage size equal to \CONST{64} bits.}
    \apitablerow{shmem\_iput128}{Any noncharacter type that has a
        storage size equal to \CONST{128} bits.}
    \apitablerow{SHMEM\_COMPLEX\_IPUT}{Elements of type complex of default size.}
    \apitablerow{SHMEM\_DOUBLE\_IPUT}{Elements of type double precision.}
    \apitablerow{SHMEM\_INTEGER\_IPUT}{Elements of type integer.}
    \apitablerow{SHMEM\_LOGICAL\_IPUT}{Elements of type logical.}
    \apitablerow{SHMEM\_REAL\_IPUT}{Elements of type real.}

\apireturnvalues{
    None.
}

\apinotes{
    When using \Fortran, data types must be of default size.  For example, a
    real variable must be declared as  \CONST{REAL}, \CONST{REAL*4} or
    \CONST{REAL(KIND=KIND(1.0))}.
    See Section \ref{subsec:memory_model} for a definition of the term
    remotely accessible.
}

\begin{apiexamples}

\apicexample
    {Consider the following \FUNC{shmem\_iput}  example  for C11 programs.}
    {./example_code/shmem_iput_example.c}
    {}
\end{apiexamples}

\end{apidefinition}


\subsubsubsection{\textbf{SHMEM\_GET}}\label{subsec:shmem_get}
\input{content/shmem_get.tex}

\subsubsubsection{\textbf{SHMEM\_G}}\label{subsec:shmem_g}
\apisummary{
    Copies one data item from a remote \ac{PE}
}

\begin{apidefinition}

\begin{C11synopsis}
TYPE shmem_g(const TYPE *source, int pe);
\end{C11synopsis}
where \TYPE{} is one of the standard \ac{RMA} types specified by Table \ref{stdrmatypes}.

\begin{Csynopsis}
TYPE shmem_<TYPENAME>_g(const TYPE *source, int pe);
\end{Csynopsis}
where \TYPE{} is one of the standard \ac{RMA} types and has a corresponding \TYPENAME{} specified by Table \ref{stdrmatypes}.

\begin{apiarguments}
    \apiargument{IN}{source}{The remotely accessible array element or scalar data object.}
    \apiargument{IN}{pe}{The number of the remote \ac{PE} on which \VAR{source} resides.}
\end{apiarguments}

\apidescription{
  These routines provide a very low latency get capability for single elements
  of most basic types. 
}

\apireturnvalues{
    Returns a single element of type specified in the synopsis.
}

\apinotes{
    None.
}

\begin{apiexamples}

\apicexample
    {The following \FUNC{shmem\_g} example is for \Cstd[11] programs:}
    {./example_code/shmem_g_example.c}
    {}
\end{apiexamples}

\end{apidefinition}


\subsubsubsection{\textbf{SHMEM\_IGET}}\label{subsec:shmem_iget}
\apisummary{
    Copies strided data from a specified \ac{PE}.
}

\begin{apidefinition}

\begin{C11synopsis}
void shmem_iget(TYPE *dest, const TYPE *source, ptrdiff_t dst, ptrdiff_t sst, size_t nelems, int pe);
\end{C11synopsis}
where \TYPE{} is one of the standard \ac{RMA} types specified by Table \ref{stdrmatypes}.

\begin{Csynopsis}
void shmem_<TYPENAME>_iget(TYPE *dest, const TYPE *source, ptrdiff_t dst, ptrdiff_t sst, size_t nelems, int pe);
\end{Csynopsis}
where \TYPE{} is one of the standard \ac{RMA} types and has a corresponding \TYPENAME{} specified by Table \ref{stdrmatypes}.

\begin{CsynopsisCol}
void shmem_iget<SIZE>(void *dest, const void *source, ptrdiff_t dst, ptrdiff_t sst, size_t  nelems, int pe);
\end{CsynopsisCol}
where \SIZE{} is one of \CONST{8, 16, 32, 64, 128}.

\begin{Fsynopsis}
INTEGER dst, sst, nelems, pe
CALL SHMEM_COMPLEX_IGET(dest, source, dst, sst, nelems, pe)
CALL SHMEM_DOUBLE_IGET(dest, source, dst, sst, nelems, pe)
CALL SHMEM_IGET4(dest, source, dst, sst, nelems, pe)
CALL SHMEM_IGET8(dest, source, dst, sst, nelems, pe)
CALL SHMEM_IGET32(dest, source, dst, sst, nelems, pe)
CALL SHMEM_IGET64(dest, source, dst, sst, nelems, pe)
CALL SHMEM_IGET128(dest, source, dst, sst, nelems, pe)
CALL SHMEM_INTEGER_IGET(dest, source, dst, sst, nelems, pe)
CALL SHMEM_LOGICAL_IGET(dest, source, dst, sst, nelems, pe)
CALL SHMEM_REAL_IGET(dest, source, dst, sst, nelems, pe)
\end{Fsynopsis}

\begin{apiarguments}
    \apiargument{OUT}{dest}{Array to be updated on the local \ac{PE}. }
    \apiargument{IN}{source}{Array containing the data to be copied on the remote \ac{PE}.}
    \apiargument{IN}{dst}{The stride between consecutive elements of the \dest{}
        array.  The stride is scaled by the element size of the \dest{} array.
        A  value of \CONST{1} indicates contiguous data. \VAR{dst} must be of
        type \CTYPE{ptrdiff\_t}.  When using  \Fortran,  it  must
        be a default integer value.}
    \apiargument{IN}{sst}{The stride between consecutive elements of the
        \source{} array.  The stride is scaled by the element size of the \source{}
        array.  A  value of \CONST{1} indicates contiguous data.  \VAR{sst} must be
        of type \CTYPE{ptrdiff\_t}.  When using  \Fortran,  it  must
        be a default integer value.}
    \apiargument{IN}{nelems}{Number of elements in the \dest{} and \source{}
        arrays.  \VAR{nelems} must be of type \VAR{size\_t} for \Cstd. When
        using \Fortran, it must be  a constant, variable, or array element of
        default integer type.}
    \apiargument{IN}{pe}{\ac{PE} number of the remote \ac{PE}.  \VAR{pe} must be
        of type integer. When using  \Fortran, it must be a constant,
        variable, or array element of default integer type.}
\end{apiarguments}

\apidescription{
    The \FUNC{iget} routines provide a method for copying strided data elements from
    a symmetric array from a specified remote \ac{PE} to strided locations on a
    local array.  The routines return when the data has been copied into the local
    \VAR{dest} array.
}

\apidesctable{
    The \VAR{dest} and \VAR{source} data objects must conform to typing
    constraints, which are as follows:}
    {Routine}{Data type of \VAR{dest} and \VAR{source}}
    \apitablerow{shmem\_iget4, shmem\_iget32}{Any noncharacter type
        that has a storage size equal to \CONST{32} bits.}
    \apitablerow{shmem\_iget8}{\Cstd: Any noncharacter type that
        has a storage size equal to \CONST{8} bits.}
    \apitablerow{}{\Fortran: Any noncharacter type that
        has a storage size equal to \CONST{64} bits.}
    \apitablerow{shmem\_iget64}{Any noncharacter type that
        has a storage size equal to \CONST{64} bits.}
    \apitablerow{shmem\_iget128}{Any noncharacter type that has a
        storage size equal to \CONST{128} bits.}
    \apitablerow{SHMEM\_COMPLEX\_IGET}{Elements of type complex of default size.}
    \apitablerow{SHMEM\_DOUBLE\_IGET}{\Fortran: Elements of type double precision.}
    \apitablerow{SHMEM\_INTEGER\_IGET}{Elements of type integer.}
    \apitablerow{SHMEM\_LOGICAL\_IGET}{Elements of type logical.}
    \apitablerow{SHMEM\_REAL\_IGET}{Elements of type real.}

\apireturnvalues{
    None.
}

\apinotes{
    When using \Fortran, data types must be of default size. For example, a
    real variable must be declared as \CONST{REAL}, \CONST{REAL*4}, or
    \CONST{REAL(KIND=KIND(1.0))}. 
}

\begin{apiexamples}

\apifexample
    {The following example uses \FUNC{shmem\_logical\_iget}  in a \Fortran
    program.} 
    {./example_code/shmem_iget_example.f90}
    {}

\end{apiexamples}

\end{apidefinition}


\subsubsection{Nonblocking Remote Memory Access Routines}\label{subsec:rma_nbi}

\subsubsubsection{\textbf{SHMEM\_PUT\_NBI}}\label{subsec:shmem_put_nbi}
\apisummary{
    The nonblocking put routines provide a method for copying data
    from a contiguous local data object to a data object on a specified \ac{PE}. 
}

\begin{apidefinition}

\begin{C11synopsis}
void shmem_put_nbi(TYPE *dest, const TYPE *source, size_t nelems, int pe);
\end{C11synopsis}
where \TYPE{} is one of the standard \ac{RMA} types specified by Table \ref{stdrmatypes}.

\begin{Csynopsis}
void shmem_<TYPENAME>_put_nbi(TYPE *dest, const TYPE *source, size_t nelems, int pe);
\end{Csynopsis}
where \TYPE{} is one of the standard \ac{RMA} types and has a corresponding \TYPENAME{} specified by Table \ref{stdrmatypes}.

\begin{CsynopsisCol}
void shmem_put<SIZE>_nbi(void *dest, const void *source, size_t nelems, int pe);
\end{CsynopsisCol}
where \SIZE{} is one of \CONST{8, 16, 32, 64, 128}.

\begin{CsynopsisCol}
void shmem_putmem_nbi(void *dest, const void *source, size_t nelems, int pe);
\end{CsynopsisCol}

\begin{Fsynopsis}
CALL SHMEM_CHARACTER_PUT_NBI(dest, source, nelems, pe)
CALL SHMEM_COMPLEX_PUT_NBI(dest, source, nelems, pe)
CALL SHMEM_DOUBLE_PUT_NBI(dest, source, nelems, pe)
CALL SHMEM_INTEGER_PUT_NBI(dest, source, nelems, pe)
CALL SHMEM_LOGICAL_PUT_NBI(dest, source, nelems, pe)
CALL SHMEM_PUT4_NBI(dest, source, nelems, pe)
CALL SHMEM_PUT8_NBI(dest, source, nelems, pe)
CALL SHMEM_PUT32_NBI(dest, source, nelems, pe)
CALL SHMEM_PUT64_NBI(dest, source, nelems, pe)
CALL SHMEM_PUT128_NBI(dest, source, nelems, pe)
CALL SHMEM_PUTMEM_NBI(dest, source, nelems, pe)
CALL SHMEM_REAL_PUT_NBI(dest, source, nelems, pe)
\end{Fsynopsis}

\begin{apiarguments}
    \apiargument{IN}{dest}{Data object to be updated on the remote \ac{PE}. This
    data object must be remotely accessible.}
    \apiargument{IN}{source}{Data object containing the data to be copied.}
    \apiargument{IN}{nelems}{Number of elements in the \VAR{dest} and \VAR{source}
    arrays. \VAR{nelems} must be of type \VAR{size\_t} for \Cstd. When using
    \Fortran, it must be a constant, variable, or array element of default
    integer type.}
    \apiargument{IN}{pe}{\ac{PE} number of the remote \ac{PE}. \VAR{pe} must be
    of type integer. When using \Fortran, it must be a constant, variable,
    or array element of default integer type.}
\end{apiarguments}

\apidescription{
    The routines return after posting the operation.  The operation is considered 
    complete after a subsequent call to \FUNC{shmem\_quiet}. 
    At the completion of \FUNC{shmem\_quiet}, the data has been copied into the \dest{} array
    on the destination \ac{PE}.
    The delivery of data words into the data object on the
    destination \ac{PE} may occur in any order.
    Furthermore, two successive put
    routines may deliver data out of order unless a call to \FUNC{shmem\_fence} is
    introduced between the two calls.   
 }

\apidesctable{
    The \dest{} and \source{} data objects must conform to certain typing
    constraints, which are as follows:}
    {Routine}{Data type of \VAR{dest} and \VAR{source}}
    \apitablerow{shmem\_putmem\_nbi}{\Fortran: Any noncharacter type. \Cstd:
        Any  data  type.  nelems is scaled in bytes.}
    \apitablerow{shmem\_put4\_nbi, shmem\_put32\_nbi}{Any noncharacter type
        that has a storage size equal to \CONST{32} bits.}
    \apitablerow{shmem\_put8\_nbi}{\Cstd: Any noncharacter type that
        has a storage size equal to \CONST{8} bits.}
    \apitablerow{}{\Fortran: Any noncharacter type that
        has a storage size equal to \CONST{64} bits.}
    \apitablerow{shmem\_put64\_nbi}{Any noncharacter type that
        has a storage size equal to \CONST{64} bits.}
    \apitablerow{shmem\_put128\_nbi}{Any noncharacter type that has a
        storage size equal to \CONST{128} bits.}
    \apitablerow{SHMEM\_CHARACTER\_PUT\_NBI}{Elements of type character.  \VAR{nelems}
    is  the number  of	 characters to transfer. The actual character lengths of
    the \source{} and \dest{} variables are ignored. }
    \apitablerow{SHMEM\_COMPLEX\_PUT\_NBI}{Elements of type complex of default size.}
    \apitablerow{SHMEM\_DOUBLE\_PUT\_NBI}{Elements of type double precision. }
    \apitablerow{SHMEM\_INTEGER\_PUT\_NBI}{Elements of type integer.}
    \apitablerow{SHMEM\_LOGICAL\_PUT\_NBI}{Elements of type logical.}
    \apitablerow{SHMEM\_REAL\_PUT\_NBI}{Elements of type real.}

\apireturnvalues{
    None.
}
\apinotes{ None.}

\end{apidefinition}


\subsubsubsection{\textbf{SHMEM\_GET\_NBI}}\label{subsec:shmem_get_nbi}
\input{content/shmem_get_nbi.tex}



\subsection{Atomic Memory Operations}\label{sec:amo}
An \ac{AMO} is a one-sided communication mechanism that combines memory read,
update, or write operations with atomicity guarantees described in Section~%
\ref{subsec:amo_guarantees}.  Similar to the \ac{RMA} routines, described in
Section~\ref{sec:rma}, the \acp{AMO} are performed only on symmetric objects.
\openshmem defines two types of \ac{AMO} routines:

\begin{itemize}

\item
  The \emph{fetching} routines return the original value of, and optionally
  update, the remote data object in a single atomic operation.  The routines
  return after the data has been fetched from the target \ac{PE} and delivered
  to the calling \ac{PE}.
  The data type of the returned value is the same as the type of
  the remote data object.

  The fetching routines include:
  \FUNC{shmem\_atomic\_\{fetch, compare\_swap, swap\}} and
  \FUNC{shmem\_atomic\_fetch\_\{inc, add, and, or, xor\}}.

\item
  The \emph{non-fetching} routines update the remote data object in a single
  atomic operation.  A call to a non-fetching atomic routine issues the atomic
  operation and may return before the operation executes on the target \ac{PE}.
  The \FUNC{shmem\_quiet}, \FUNC{shmem\_barrier}, or \FUNC{shmem\_barrier\_all}
  routines can be used to force completion for these non-fetching
  atomic routines.

  The non-fetching routines include:
  \FUNC{shmem\_atomic\_\{set, inc, add, and, or, xor\}}.

\end{itemize}

Where appropriate compiler support is available, \openshmem provides
type-generic \ac{AMO} interfaces via \Cstd[11] generic selection.
The type-generic support for the \ac{AMO} routines is as follows:

\begin{itemize}
\item \FUNC{shmem\_atomic\_\{compare\_swap, fetch\_inc, inc, fetch\_add, add\}}
  support the ``standard \ac{AMO} types'' listed in Table~\ref{stdamotypes},
\item \FUNC{shmem\_atomic\_\{fetch, set, swap\}} support
  the ``extended \ac{AMO} types'' listed in Table~\ref{extamotypes}, and
\item \FUNC{shmem\_atomic\_\{fetch\_and, and, fetch\_or, or, fetch\_xor, xor\}}
  support the ``bitwise \ac{AMO} types'' listed in Table~\ref{bitamotypes}.
\end{itemize}

The standard, extended, and bitwise \ac{AMO} types include some of the exact-width
integer types defined in \HEADER{stdint.h} by \Cstd[99]~\S7.18.1.1 and
\Cstd[11]~\S7.20.1.1. When the \Cstd translation environment
does not provide exact-width integer types with \HEADER{stdint.h}, an
\openshmem implemementation is not required to provide support for these types.

\begin{table}[h]
  \begin{center}
    \begin{tabular}{|l|l|}
      \hline
      \TYPE              & \TYPENAME  \\ \hline
      int                & int        \\ \hline
      long               & long       \\ \hline
      long long          & longlong   \\ \hline
      unsigned int       & uint       \\ \hline
      unsigned long      & ulong      \\ \hline
      unsigned long long & ulonglong  \\ \hline
      int32\_t           & int32      \\ \hline
      int64\_t           & int64      \\ \hline
      uint32\_t          & uint32     \\ \hline
      uint64\_t          & uint64     \\ \hline
      size\_t            & size       \\ \hline
      ptrdiff\_t         & ptrdiff    \\ \hline
    \end{tabular}
    \caption{Standard \ac{AMO} Types and Names}
    \label{stdamotypes}
  \end{center}
\end{table}

\begin{table}[h]
  \begin{center}
    \begin{tabular}{|l|l|}
      \hline
      \TYPE              & \TYPENAME  \\ \hline
      float              & float      \\ \hline
      double             & double     \\ \hline
      int                & int        \\ \hline
      long               & long       \\ \hline
      long long          & longlong   \\ \hline
      unsigned int       & uint       \\ \hline
      unsigned long      & ulong      \\ \hline
      unsigned long long & ulonglong  \\ \hline
      int32\_t           & int32      \\ \hline
      int64\_t           & int64      \\ \hline
      uint32\_t          & uint32     \\ \hline
      uint64\_t          & uint64     \\ \hline
      size\_t            & size       \\ \hline
      ptrdiff\_t         & ptrdiff    \\ \hline
    \end{tabular}
    \caption{Extended \ac{AMO} Types and Names}
    \label{extamotypes}
  \end{center}
\end{table}

\begin{table}[h]
  \begin{center}
    \begin{tabular}{|l|l|}
      \hline
      \TYPE              & \TYPENAME  \\ \hline
      unsigned int       & uint       \\ \hline
      unsigned long      & ulong      \\ \hline
      unsigned long long & ulonglong  \\ \hline
      int32\_t           & int32      \\ \hline
      int64\_t           & int64      \\ \hline
      uint32\_t          & uint32     \\ \hline
      uint64\_t          & uint64     \\ \hline
    \end{tabular}
    \caption{Bitwise \ac{AMO} Types and Names}
    \label{bitamotypes}
  \end{center}
\end{table}


\subsubsection{Blocking Atomic Memory Operations}\label{subsec:amo}

\subsubsubsection{\textbf{SHMEM\_ATOMIC\_FETCH}}
\label{subsec:shmem_atomic_fetch}
\input{content/shmem_atomic_fetch.tex}

\subsubsubsection{\textbf{SHMEM\_ATOMIC\_SET}}
\label{subsec:shmem_atomic_set}
\input{content/shmem_atomic_set.tex}

\subsubsubsection{\textbf{SHMEM\_ATOMIC\_COMPARE\_SWAP}}
\label{subsec:shmem_atomic_compare_swap}
\apisummary{
    Performs an atomic conditional swap on a remote data object.
}

\begin{apidefinition}

\begin{C11synopsis}
TYPE shmem_atomic_compare_swap(TYPE *dest, TYPE cond, TYPE value, int pe);
\end{C11synopsis}
where \TYPE{} is one of the standard \ac{AMO} types specified by
Table~\ref{stdamotypes}.

\begin{Csynopsis}
TYPE shmem_<TYPENAME>_atomic_compare_swap(TYPE *dest, TYPE cond, TYPE value, int pe);
\end{Csynopsis}
where \TYPE{} is one of the standard \ac{AMO} types and has a corresponding
\TYPENAME{} specified by Table~\ref{stdamotypes}.

\begin{Fsynopsis}
INTEGER pe
INTEGER*4 SHMEM_INT4_CSWAP,  cond_i4, value_i4, ires_i4
ires_i4 = SHMEM_INT4_CSWAP(dest, cond_i4, value_i4, pe)
INTEGER*8 SHMEM_INT8_CSWAP,  cond_i8, value_i8, ires_i8
ires_i8 = SHMEM_INT8_CSWAP(dest, cond_i8, value_i8, pe)
\end{Fsynopsis}

\begin{apiarguments}
    \apiargument{IN}{dest}{The remotely accessible integer data object to be
        updated on the remote \ac{PE}. }
    \apiargument{IN}{cond}{\VAR{cond} is compared to the remote \VAR{dest}
        value. If \VAR{cond} and the remote \VAR{dest} are equal, then \VAR{value}
        is swapped into the remote \VAR{dest}. Otherwise, the remote \VAR{dest} is
        unchanged.  In either case, the old value of the remote \VAR{dest} is
        returned as the routine return value. \VAR{cond} must be of the same data
        type as \VAR{dest}.}
    \apiargument{IN}{value}{The value to be atomically written to the remote
        \ac{PE}. \VAR{value} must be the same data type as \VAR{dest}.}
    \apiargument{IN}{pe}{An integer that indicates the \ac{PE} number upon which
        \VAR{dest} is to be updated. When using \Fortran, it must be a default
        integer value.}
\end{apiarguments}

\apidescription{
    The conditional swap routines conditionally update a \VAR{dest} data object on
    the specified \ac{PE} and return the prior contents of the data object in one
    atomic operation.
}
\apidesctable{
    When using \Fortran, \VAR{dest}, \VAR{cond}, and \VAR{value} must be of the following type:
}{Routine}{Data type of \VAR{dest}, \VAR{cond}, and \VAR{value}}

\apitablerow{SHMEM\_INT4\_CSWAP}{\CONST{4}-byte integer.}
\apitablerow{SHMEM\_INT8\_CSWAP}{\CONST{8}-byte integer.}


\apireturnvalues{
    The contents that had been in the \VAR{dest} data object on the remote
    \ac{PE} prior to the conditional swap. Data type is the same as the
    \VAR{dest} data type.
}

\apinotes{
    As of \openshmem[1.4], \FUNC{shmem\_cswap} has been deprecated.
    Its behavior and call signature are identical to the replacement
    interface, \FUNC{shmem\_atomic\_compare\_swap}.
}

\begin{apiexamples}

\apicexample
    {The following call ensures that the first \ac{PE} to execute the
    conditional swap will successfully write its \ac{PE} number to
    \VAR{race\_winner} on \ac{PE} \CONST{0}.}
    {./example_code/shmem_atomic_compare_swap_example.c}
    {}

\end{apiexamples}

\end{apidefinition}


\subsubsubsection{\textbf{SHMEM\_ATOMIC\_SWAP}}
\label{subsec:shmem_atomic_swap}
\apisummary{
    Performs an atomic swap to a remote data object.
}

\begin{apidefinition}

\begin{C11synopsis}
TYPE shmem_atomic_swap(TYPE *dest, TYPE value, int pe);
\end{C11synopsis}
where \TYPE{} is one of the extended \ac{AMO} types specified by Table \ref{extamotypes}.

\begin{Csynopsis}
TYPE shmem_<TYPENAME>_atomic_swap(TYPE *dest, TYPE value, int pe);
\end{Csynopsis}
where \TYPE{} is one of the extended \ac{AMO} types and has a corresponding \TYPENAME{} specified by Table \ref{extamotypes}.

\begin{Fsynopsis}
INTEGER SHMEM_SWAP, value, pe
ires = SHMEM_SWAP(dest, value, pe)
INTEGER*4 SHMEM_INT4_SWAP, value_i4, ires_i4
ires_i4 = SHMEM_INT4_SWAP(dest, value_i4, pe)
INTEGER*8 SHMEM_INT8_SWAP, value_i8, ires_i8
ires_i8 = SHMEM_INT8_SWAP(dest, value_i8, pe)
REAL*4 SHMEM_REAL4_SWAP, value_r4, res_r4
res_r4 = SHMEM_REAL4_SWAP(dest, value_r4, pe)
REAL*8 SHMEM_REAL8_SWAP, value_r8, res_r8
res_r8 = SHMEM_REAL8_SWAP(dest, value_r8, pe)
\end{Fsynopsis}

\begin{apiarguments}
    \apiargument{IN}{dest}{The  remotely accessible integer data object to be
        updated on the remote \ac{PE}.	 When using \CorCpp, the type of
        \dest{} should match that  implied in the SYNOPSIS section.}
    \apiargument{IN}{value}{The value to be atomically written to the remote
        \ac{PE}. \VAR{value}  is the same type as \dest.}
    \apiargument{IN}{pe}{ An integer that indicates the \ac{PE} number on which
        \dest{} is to be updated. When using \Fortran, it must be a default
        integer value.}
\end{apiarguments}

\apidescription{
    \FUNC{shmem\_atomic\_swap} performs an atomic swap operation.
    It writes \VAR{value} into \dest{} on \ac{PE} and returns the previous
    contents of \dest{} as an atomic operation.
}

\apidesctable{
  When using \Fortran, \VAR{dest} and \VAR{value} must be of the following type:
}{Routine}{Data type of \VAR{dest} and \VAR{value}}

\apitablerow{SHMEM\_SWAP}{Integer of default kind}
\apitablerow{SHMEM\_INT4\_SWAP}{\CONST{4}-byte integer}
\apitablerow{SHMEM\_INT8\_SWAP}{\CONST{8}-byte integer}
\apitablerow{SHMEM\_REAL4\_SWAP}{\CONST{4}-byte real}
\apitablerow{SHMEM\_REAL8\_SWAP}{\CONST{8}-byte real}

\apireturnvalues{
       The content that had been at the \dest{} address on the remote \ac{PE}
       prior to the swap is returned.
}

\apinotes{
    As of \openshmem[1.4], \FUNC{shmem\_swap} has been deprecated.
    Its behavior and call signature are identical to the replacement
    interface, \FUNC{shmem\_atomic\_swap}.
}

\begin{apiexamples}

\apicexample
    {The example below swaps values between odd numbered \acp{PE} and
    their right (modulo) neighbor and outputs the result of swap.}
    {./example_code/shmem_atomic_swap_example.c}
    {}

\end{apiexamples}

\end{apidefinition}


\subsubsubsection{\textbf{SHMEM\_ATOMIC\_FETCH\_INC}}
\label{subsec:shmem_atomic_fetch_inc}
\input{content/shmem_atomic_fetch_inc.tex}

\subsubsubsection{\textbf{SHMEM\_ATOMIC\_INC}}
\label{subsec:shmem_atomic_inc}
\input{content/shmem_atomic_inc.tex}

\subsubsubsection{\textbf{SHMEM\_ATOMIC\_FETCH\_ADD}}
\label{subsec:shmem_atomic_fetch_add}
\apisummary{
    Performs an atomic fetch-and-add operation on a remote data object.
}

\begin{apidefinition}

\begin{C11synopsis}
TYPE shmem_atomic_fetch_add(TYPE *dest, TYPE value, int pe);
\end{C11synopsis}
where \TYPE{} is one of the standard \ac{AMO} types specified by
Table~\ref{stdamotypes}.

\begin{Csynopsis}
TYPE shmem_<TYPENAME>_atomic_fetch_add(TYPE *dest, TYPE value, int pe);
\end{Csynopsis}
where \TYPE{} is one of the standard \ac{AMO} types and has a corresponding
\TYPENAME{} specified by Table~\ref{stdamotypes}.

\begin{Fsynopsis}
INTEGER pe
INTEGER*4 SHMEM_INT4_FADD, ires_i4, value_i4
ires_i4 = SHMEM_INT4_FADD(dest, value_i4, pe)
INTEGER*8 SHMEM_INT8_FADD, ires_i8, value_i8
ires_i8 = SHMEM_INT8_FADD(dest, value_i8, pe)
\end{Fsynopsis}

\begin{apiarguments}

\apiargument{IN}{dest}{The remotely accessible integer data object to be updated on
    the remote \ac{PE}. The type of \VAR{dest} should match that implied in the
    SYNOPSIS section.}
\apiargument{IN}{value}{The value to be atomically added to \VAR{dest}.  The
    type of \VAR{value} should match that implied in the SYNOPSIS section.}
\apiargument{IN}{pe}{An integer that indicates the \ac{PE} number on which
    \VAR{dest} is to be updated.  When using \Fortran, it must be a default
    integer value.}

\end{apiarguments}

\apidescription{
    \FUNC{shmem\_atomic\_fetch\_add} routines perform an atomic fetch-and-add operation.  An
    atomic fetch-and-add operation fetches the old \VAR{dest} and adds \VAR{value}
    to \VAR{dest} without the possibility of another atomic operation on the
    \VAR{dest} between the time of the fetch and the update.  These routines add
    \VAR{value} to \VAR{dest} on \VAR{pe} and return the previous contents of
    \VAR{dest} as an atomic operation.
}

\apidesctable{
    When using \Fortran, \VAR{dest} and \VAR{value} must be of the following type:
}{Routine}{Data type of \VAR{dest} and \VAR{value}}

\apitablerow{SHMEM\_INT4\_FADD}{\CONST{4}-byte integer}
\apitablerow{SHMEM\_INT8\_FADD}{\CONST{8}-byte integer}


\apireturnvalues{
    The contents that had been at the \VAR{dest} address on the remote \ac{PE}
    prior to the atomic addition operation.  The data type of the return value is
    the same as the \VAR{dest}.
}

\apinotes{
    As of \openshmem[1.4], \FUNC{shmem\_fadd} has been deprecated.
    Its behavior and call signature are identical to the replacement
    interface, \FUNC{shmem\_atomic\_fetch\_add}.
}

\begin{apiexamples}

\apicexample
        {The following \FUNC{shmem\_atomic\_fetch\_add} example is for
        \Cstd[11] programs:}
        {./example_code/shmem_atomic_fetch_add_example.c}
        {}

\end{apiexamples}

\end{apidefinition}


\subsubsubsection{\textbf{SHMEM\_ATOMIC\_ADD}}
\label{subsec:shmem_atomic_add}
\apisummary{
    Performs an atomic add operation on a remote symmetric data object.
}

\begin{apidefinition}

\begin{C11synopsis}
void shmem_atomic_add(TYPE *dest, TYPE value, int pe);
\end{C11synopsis}
where \TYPE{} is one of the standard \ac{AMO} types specified by
Table~\ref{stdamotypes}.

\begin{Csynopsis}
void shmem_<TYPENAME>_atomic_add(TYPE *dest, TYPE value, int pe);
\end{Csynopsis}
where \TYPE{} is one of the standard \ac{AMO} types and has a corresponding
\TYPENAME{} specified by Table~\ref{stdamotypes}.

\begin{Fsynopsis}
INTEGER pe
INTEGER*4  value_i4
CALL SHMEM_INT4_ADD(dest, value_i4, pe)
INTEGER*8 value_i8
CALL SHMEM_INT8_ADD(dest, value_i8, pe)
\end{Fsynopsis}

\begin{apiarguments}
    \apiargument{IN}{dest}{The remotely accessible integer data object to be
        updated  on the remote \ac{PE}.  When using \CorCpp, the type of
        \dest{} should match that implied in the SYNOPSIS section.}
    \apiargument{IN}{value}{The value to be atomically added to \dest. When using \CorCpp, the type of \VAR{value} should match that  implied  in
        the SYNOPSIS  section.  When using \Fortran, it must be of type
        integer with an element size of \dest.}
    \apiargument{IN}{pe}{An integer that indicates the \ac{PE} number upon which
        \dest{} is to be updated.  When using \Fortran, it must be a default
        integer value.}
\end{apiarguments}

\apidescription{
    The \FUNC{shmem\_atomic\_add} routine performs an atomic add operation. It adds
    \VAR{value} to \dest{} on \ac{PE} \VAR{pe} and atomically updates the \dest{}
    without returning the value.
 }

\apidesctable{
    When using \Fortran, \VAR{dest} and \VAR{value} must be of the following type:
}{Routine}{Data type of \VAR{dest} and \VAR{value}}

\apitablerow{SHMEM\_INT4\_ADD}{\CONST{4}-byte integer}
\apitablerow{SHMEM\_INT8\_ADD}{\CONST{8}-byte integer}

\apireturnvalues{
    None.
}

\apinotes{
    As of \openshmem[1.4], \FUNC{shmem\_add} has been deprecated.
    Its behavior and call signature are identical to the replacement
    interface, \FUNC{shmem\_atomic\_add}.
}

\begin{apiexamples}

\apicexample
    {}
    {./example_code/shmem_atomic_add_example.c}
    {}

\end{apiexamples}

\end{apidefinition}


\subsubsubsection{\textbf{SHMEM\_ATOMIC\_FETCH\_AND}}
\label{subsec:shmem_atomic_fetch_and}
\apisummary{
  Atomically perform a fetching bitwise AND operation on a remote data object.
}

\begin{apidefinition}

\begin{C11synopsis}
TYPE shmem_atomic_fetch_and(TYPE *dest, TYPE value, int pe);
\end{C11synopsis}
where \TYPE{} is one of the bitwise \ac{AMO} types specified by
Table~\ref{bitamotypes}.

\begin{Csynopsis}
TYPE shmem_<TYPENAME>_atomic_fetch_and(TYPE *dest, TYPE value, int pe);
\end{Csynopsis}
where \TYPE{} is one of the bitwise \ac{AMO} types and has a corresponding
\TYPENAME{} specified by Table~\ref{bitamotypes}.

\begin{apiarguments}

  \apiargument{IN}{dest}{A pointer to the remotely accessible data object to
    be updated.}
  \apiargument{IN}{value}{The operand to the bitwise AND operation.}
  \apiargument{IN}{pe}{An integer value for the \ac{PE} on which \VAR{dest}
    is to be updated.}

\end{apiarguments}

\apidescription{
  \FUNC{shmem\_atomic\_fetch\_and} atomically performs a fetching bitwise AND
  on the remotely accessible data object pointed to by \VAR{dest} at PE
  \VAR{pe} with the operand \VAR{value}.
}

\apireturnvalues{
  The value pointed to by \VAR{dest} on PE \VAR{pe} immediately before the
  operation is performed.
}

\apinotes{
  None.
}

\end{apidefinition}


\subsubsubsection{\textbf{SHMEM\_ATOMIC\_AND}}
\label{subsec:shmem_atomic_and}
\apisummary{
  Atomically perform a non-fetching bitwise AND operation on a
  remote data object.
}

\begin{apidefinition}

\begin{C11synopsis}
void shmem_atomic_and(TYPE *dest, TYPE value, int pe);
\end{C11synopsis}
where \TYPE{} is one of the bitwise \ac{AMO} types specified by
Table~\ref{bitamotypes}.

\begin{Csynopsis}
void shmem_<TYPENAME>_atomic_and(TYPE *dest, TYPE value, int pe);
\end{Csynopsis}
where \TYPE{} is one of the bitwise \ac{AMO} types and has a corresponding
\TYPENAME{} specified by Table~\ref{bitamotypes}.

\begin{apiarguments}

  \apiargument{IN}{dest}{A pointer to the remotely accessible data object to
    be updated.}
  \apiargument{IN}{value}{The operand to the bitwise AND operation.}
  \apiargument{IN}{pe}{An integer value for the \ac{PE} on which \VAR{dest}
    is to be updated.}

\end{apiarguments}

\apidescription{
  \FUNC{shmem\_atomic\_and} atomically performs a non-fetching bitwise AND
  on the remotely accessible data object pointed to by \VAR{dest} at PE
  \VAR{pe} with the operand \VAR{value}.
}

\apireturnvalues{
  None.
}

\apinotes{
  None.
}

\end{apidefinition}


\subsubsubsection{\textbf{SHMEM\_ATOMIC\_FETCH\_OR}}
\label{subsec:shmem_atomic_fetch_or}
\apisummary{
  Atomically perform a fetching bitwise OR operation on a remote data object.
}

\begin{apidefinition}

\begin{C11synopsis}
TYPE shmem_atomic_fetch_or(TYPE *dest, TYPE value, int pe);
\end{C11synopsis}
where \TYPE{} is one of the bitwise \ac{AMO} types specified by
Table~\ref{bitamotypes}.

\begin{Csynopsis}
TYPE shmem_<TYPENAME>_atomic_fetch_or(TYPE *dest, TYPE value, int pe);
\end{Csynopsis}
where \TYPE{} is one of the bitwise \ac{AMO} types and has a corresponding
\TYPENAME{} specified by Table~\ref{bitamotypes}.

\begin{apiarguments}

  \apiargument{IN}{dest}{A pointer to the remotely accessible data object to
    be updated.}
  \apiargument{IN}{value}{The operand to the bitwise OR operation.}
  \apiargument{IN}{pe}{An integer value for the \ac{PE} on which \VAR{dest}
    is to be updated.}

\end{apiarguments}

\apidescription{
  \FUNC{shmem\_atomic\_fetch\_or} atomically performs a fetching bitwise OR
  on the remotely accessible data object pointed to by \VAR{dest} at PE
  \VAR{pe} with the operand \VAR{value}.
}

\apireturnvalues{
  The value pointed to by \VAR{dest} on PE \VAR{pe} immediately before the
  operation is performed.
}

\apinotes{
  None.
}

\end{apidefinition}


\subsubsubsection{\textbf{SHMEM\_ATOMIC\_OR}}
\label{subsec:shmem_atomic_or}
\apisummary{
  Atomically perform a non-fetching bitwise OR operation on a
  remote data object.
}

\begin{apidefinition}

\begin{C11synopsis}
void shmem_atomic_or(TYPE *dest, TYPE value, int pe);
\end{C11synopsis}
where \TYPE{} is one of the bitwise \ac{AMO} types specified by
Table~\ref{bitamotypes}.

\begin{Csynopsis}
void shmem_<TYPENAME>_atomic_or(TYPE *dest, TYPE value, int pe);
\end{Csynopsis}
where \TYPE{} is one of the bitwise \ac{AMO} types and has a corresponding
\TYPENAME{} specified by Table~\ref{bitamotypes}.

\begin{apiarguments}

  \apiargument{IN}{dest}{A pointer to the remotely accessible data object to
    be updated.}
  \apiargument{IN}{value}{The operand to the bitwise OR operation.}
  \apiargument{IN}{pe}{An integer value for the \ac{PE} on which \VAR{dest}
    is to be updated.}

\end{apiarguments}

\apidescription{
  \FUNC{shmem\_atomic\_or} atomically performs a non-fetching bitwise OR
  on the remotely accessible data object pointed to by \VAR{dest} at PE
  \VAR{pe} with the operand \VAR{value}.
}

\apireturnvalues{
  None.
}

\apinotes{
  None.
}

\end{apidefinition}


\subsubsubsection{\textbf{SHMEM\_ATOMIC\_FETCH\_XOR}}
\label{subsec:shmem_atomic_fetch_xor}
\apisummary{
  Atomically perform a fetching bitwise exclusive OR (XOR) operation on a
  remote data object.
}

\begin{apidefinition}

\begin{C11synopsis}
TYPE shmem_atomic_fetch_xor(TYPE *dest, TYPE value, int pe);
\end{C11synopsis}
where \TYPE{} is one of the bitwise \ac{AMO} types specified by
Table~\ref{bitamotypes}.

\begin{Csynopsis}
TYPE shmem_<TYPENAME>_atomic_fetch_xor(TYPE *dest, TYPE value, int pe);
\end{Csynopsis}
where \TYPE{} is one of the bitwise \ac{AMO} types and has a corresponding
\TYPENAME{} specified by Table~\ref{bitamotypes}.

\begin{apiarguments}

  \apiargument{IN}{dest}{A pointer to the remotely accessible data object to
    be updated.}
  \apiargument{IN}{value}{The operand to the bitwise XOR operation.}
  \apiargument{IN}{pe}{An integer value for the \ac{PE} on which \VAR{dest}
    is to be updated.}

\end{apiarguments}

\apidescription{
  \FUNC{shmem\_atomic\_fetch\_xor} atomically performs a fetching bitwise XOR
  on the remotely accessible data object pointed to by \VAR{dest} at PE
  \VAR{pe} with the operand \VAR{value}.
}

\apireturnvalues{
  The value pointed to by \VAR{dest} on PE \VAR{pe} immediately before the
  operation is performed.
}

\apinotes{
  None.
}

\end{apidefinition}


\subsubsubsection{\textbf{SHMEM\_ATOMIC\_XOR}}
\label{subsec:shmem_atomic_xor}
\apisummary{
  Atomically perform a non-fetching bitwise exclusive OR (XOR) operation on a
  remote data object.
}

\begin{apidefinition}

\begin{C11synopsis}
void shmem_atomic_xor(TYPE *dest, TYPE value, int pe);
\end{C11synopsis}
where \TYPE{} is one of the bitwise \ac{AMO} types specified by
Table~\ref{bitamotypes}.

\begin{Csynopsis}
void shmem_<TYPENAME>_atomic_xor(TYPE *dest, TYPE value, int pe);
\end{Csynopsis}
where \TYPE{} is one of the bitwise \ac{AMO} types and has a corresponding
\TYPENAME{} specified by Table~\ref{bitamotypes}.

\begin{apiarguments}

  \apiargument{IN}{dest}{A pointer to the remotely accessible data object to
    be updated.}
  \apiargument{IN}{value}{The operand to the bitwise XOR operation.}
  \apiargument{IN}{pe}{An integer value for the \ac{PE} on which \VAR{dest}
    is to be updated.}

\end{apiarguments}

\apidescription{
  \FUNC{shmem\_atomic\_xor} atomically performs a non-fetching bitwise XOR
  on the remotely accessible data object pointed to by \VAR{dest} at PE
  \VAR{pe} with the operand \VAR{value}.
}

\apireturnvalues{
  None.
}

\apinotes{
  None.
}

\end{apidefinition}


\subsubsection{Nonblocking Atomic Memory Operations}\label{subsec:amo_nbi}

\subsubsubsection{\textbf{SHMEM\_ATOMIC\_FETCH\_NBI}}
\label{subsec:shmem_atomic_fetch_nbi}
\input{content/shmem_atomic_fetch_nbi.tex}

\subsubsubsection{\textbf{SHMEM\_ATOMIC\_COMPARE\_SWAP\_NBI}}
\label{subsec:shmem_atomic_compare_swap_nbi}
\input{content/shmem_atomic_compare_swap_nbi.tex}

\subsubsubsection{\textbf{SHMEM\_ATOMIC\_SWAP\_NBI}}
\label{subsec:shmem_atomic_swap_nbi}
\input{content/shmem_atomic_swap_nbi.tex}

\subsubsubsection{\textbf{SHMEM\_ATOMIC\_FETCH\_INC\_NBI}}
\label{subsec:shmem_atomic_fetch_inc_nbi}
\input{content/shmem_atomic_fetch_inc_nbi.tex}

\subsubsubsection{\textbf{SHMEM\_ATOMIC\_FETCH\_ADD\_NBI}}
\label{subsec:shmem_atomic_fetch_add_nbi}
\input{content/shmem_atomic_fetch_add_nbi.tex}

\subsubsubsection{\textbf{SHMEM\_ATOMIC\_FETCH\_AND\_NBI}}
\label{subsec:shmem_atomic_fetch_and_nbi}
\input{content/shmem_atomic_fetch_and_nbi.tex}

\subsubsubsection{\textbf{SHMEM\_ATOMIC\_FETCH\_OR\_NBI}}
\label{subsec:shmem_atomic_fetch_or_nbi}
\input{content/shmem_atomic_fetch_or_nbi.tex}

\subsubsubsection{\textbf{SHMEM\_ATOMIC\_FETCH\_XOR\_NBI}}
\label{subsec:shmem_atomic_fetch_xor_nbi}
\input{content/shmem_atomic_fetch_xor_nbi.tex}



\subsection{Signaling Operations}\label{sec:shmem_signal}
This section specifies the OpenSHMEM support for \OPR{put-with-signal},
nonblocking \OPR{put-with-signal}, and \OPR{signal-fetch} routines. The
put-with-signal routines provide a method for copying data from a contiguous
local data object to a data object on a specified \ac{PE} and subsequently
updating a remote flag to signal completion. The signal-fetch routine provides
support for fetching a signal update operation.

\openshmem \OPR{put-with-signal} routines specified in this section have two
variants. In one of the variants, the context handle, \VAR{ctx}, is explicitly
passed as an argument. In this variant, the operation is performed on the
specified context. If the context handle \VAR{ctx} does not correspond to a
valid context, the behavior is undefined. In the other variant, the context
handle is not explicitly passed and thus, the operations are performed on the
default context.

\subsubsection{Atomicity Guarantees for Signaling Operations}
\label{subsec:signal_atomicity}
All signaling operations put-with-signal, nonblocking put-with-signal, and
signal-fetch are performed on a signal data object, a remotely accessible
symmetric object of type \VAR{uint64\_t}. A signal operator in the
put-with-signal routine is a \openshmem library constant that determines the
type of update to be performed as a signal on the signal data object.

All signaling operations on the signal data object completes as if performed
atomically with respect to the following:
\begin{itemize}
    \item other blocking or nonblocking variant of the put-with-signal routine
    that updates the signal data object using the same signal update operator;
    \item signal-fetch routine that fetches the signal data object; and
    \item any point-to-point synchronization routine that accesses the signal
    data object.
\end{itemize}

\subsubsection{Available Signal Operators}
\label{subsec:signal_operator}

With the atomicity guarantees as described in
Section~\ref{subsec:signal_atomicity}, the following options can be used as a
signal operator.

    \apitablerow{\LibConstRef{SHMEM\_SIGNAL\_SET}}{An update to signal data
    object is an atomic set operation. It writes an unsigned 64-bit value as a
    signal into the signal data object on a remote \VAR{PE} as an atomic
    operation.}

    \apitablerow{\LibConstRef{SHMEM\_SIGNAL\_ADD}}{An update to signal data
    object is an atomic add operation. It adds an unsigned 64-bit value as a
    signal into the signal data object on a remote \VAR{PE} as an atomic
    operation.}


\subsubsection{\textbf{SHMEM\_PUT\_SIGNAL}}\label{subsec:shmem_put_signal}
\input{content/shmem_put_signal.tex}

\subsubsection{\textbf{SHMEM\_PUT\_SIGNAL\_NBI}}\label{subsec:shmem_put_signal_nbi}
\input{content/shmem_put_signal_nbi.tex}

\subsubsection{\textbf{SHMEM\_SIGNAL\_FETCH}}\label{subsec:shmem_signal_fetch}
\input{content/shmem_signal_fetch.tex}



\subsection{Collective Routines}\label{subsec:coll}
\input{content/collective_intro.tex}

\subsubsection{\textbf{SHMEM\_BARRIER\_ALL}}\label{subsec:shmem_barrier_all}
\apisummary{
    Registers the arrival of a \ac{PE} at a barrier and suspends \ac{PE} execution
    until all other \acp{PE} arrive at the barrier and all local and remote memory
    updates are completed.
}

\begin{apidefinition}

\begin{Csynopsis}
void shmem_barrier_all(void);
\end{Csynopsis}

\begin{Fsynopsis}
CALL SHMEM_BARRIER_ALL
\end{Fsynopsis}

\begin{apiarguments}

    \apiargument{None.}{}{} 

\end{apiarguments}

\apidescription{   
    The \FUNC{shmem\_barrier\_all} routine registers the arrival of a \ac{PE} at
    a barrier. Barriers are a fast mechanism for synchronizing all \acp{PE} at
    once.  This routine causes a \ac{PE} to suspend execution until all \acp{PE}
    have called \FUNC{shmem\_barrier\_all}.  This routine must be used with
    \acp{PE} started by \FUNC{shmem\_init}.

    Prior to synchronizing with other \acp{PE}, \FUNC{shmem\_barrier\_all}
    ensures completion of all previously issued memory stores and remote memory
    updates issued via \openshmem \acp{AMO} and \ac{RMA} routine calls  such
    as \FUNC{shmem\_int\_add}, \FUNC{shmem\_put32}, 
    \FUNC{shmem\_put\_nbi}, and \FUNC{shmem\_get\_nbi}.
}

\apireturnvalues{
    None.
}

\apinotes{
    The \FUNC{shmem\_barrier\_all} routine can be used to
    portably ensure that memory access operations observe remote updates in the order
    enforced by initiator PEs.
}

\begin{apiexamples}

\apicexample
    { The following \FUNC{shmem\_barrier\_all} example is for C11 programs:}
    {./example_code/shmem_barrierall_example.c}
    {} 

\end{apiexamples}

\end{apidefinition}


\subsubsection{\textbf{SHMEM\_BARRIER}}\label{subsec:shmem_barrier}
\input{content/shmem_barrier.tex}

\subsubsection{\textbf{SHMEM\_SYNC}}\label{subsec:shmem_sync}
\input{content/shmem_sync.tex}

\subsubsection{\textbf{SHMEM\_SYNC\_ALL}}\label{subsec:shmem_sync_all}
\input{content/shmem_sync_all.tex}

\subsubsection{\textbf{SHMEM\_ALLTOALL}}\label{subsec:shmem_alltoall}
\input{content/shmem_alltoall.tex}

\subsubsection{\textbf{SHMEM\_ALLTOALLS}}\label{subsec:shmem_alltoalls}
\apisummary{
    shmem\_alltoalls is a collective routine where each \ac{PE} exchanges a fixed amount of strided data with all other
    \acp{PE} in the \activeset.
}

\begin{apidefinition}

\begin{Csynopsis}
void shmem_alltoalls32(void *dest, const void *source, ptrdiff_t dst, ptrdiff_t sst, size_t nelems, int PE_start, int logPE_stride, int PE_size, long *pSync);
void shmem_alltoalls64(void *dest, const void *source, ptrdiff_t dst, ptrdiff_t sst, size_t nelems, int PE_start, int logPE_stride, int PE_size, long *pSync);
\end{Csynopsis}

\begin{Fsynopsis}
INTEGER pSync(SHMEM_ALLTOALLS_SYNC_SIZE)
INTEGER dst, sst, PE_start, logPE_stride, PE_size
INTEGER nelems 
CALL SHMEM_ALLTOALLS32(dest, source, dst, sst, nelems, PE_start, logPE_stride, PE_size, pSync)
CALL SHMEM_ALLTOALLS64(dest, source, dst, sst, nelems, PE_start, logPE_stride, PE_size, pSync)
\end{Fsynopsis}

\begin{apiarguments}

\apiargument{OUT}{dest}{A symmetric data object large enough to receive 
    the combined total of \VAR{nelems} elements from each \ac{PE} in the
    \activeset.}
\apiargument{IN}{source}{A symmetric data object that contains \VAR{nelems} 
    elements of data for each \ac{PE} in the \activeset{}, ordered according to 
    destination \ac{PE}.}
\apiargument{IN}{dst}{The stride between consecutive elements of the \dest{}
    data object.  The stride is scaled by the element size.  A
    value of \CONST{1} indicates contiguous data.  \VAR{dst} must be of type
    \CTYPE{ptrdiff\_t}.  When using \Fortran, it must be a default integer
    value.}
\apiargument{IN}{sst}{The  stride between consecutive elements of the
    \source{} data object.  The stride is scaled by the element size.
    A value of \CONST{1} indicates contiguous data.  \VAR{sst} must be
    of type \CTYPE{ptrdiff\_t}.  When using \Fortran, it must be a
    default integer value.}
\apiargument{IN}{nelems}{The number of elements to exchange for each \ac{PE}.
    \VAR{nelems} must be of type size\_t for \CorCpp.  When using
    \Fortran, it must be a default integer value.}
\apiargument{IN}{PE\_start}{The lowest \ac{PE} number of the \activeset{} of
    \acp{PE}.  \VAR{PE\_start} must be of type integer.  When using \Fortran,
    it must be a default integer value.}
\apiargument{IN}{logPE\_stride}{The log (base 2) of the stride between
    consecutive \ac{PE} numbers in the \activeset.  \VAR{logPE\_stride} must be of
    type integer.  When using \Fortran, it must be a default integer value.}
\apiargument{IN}{PE\_size}{The number of \acp{PE} in the \activeset.
    \VAR{PE\_size} must be of type integer.  When using \Fortran, it must
    be a default integer value.}
\apiargument{IN}{pSync}{A symmetric work array. In \CorCpp, \VAR{pSync} must be
    of type long and size \CONST{SHMEM\_ALLTOALLS\_SYNC\_SIZE}. In \Fortran,
    \VAR{pSync} must be of type integer and size
    \CONST{SHMEM\_ALLTOALLS\_SYNC\_SIZE}.  When using \Fortran, it must be a
    default integer value. Every element of this array must be initialized with
    the value \CONST{SHMEM\_SYNC\_VALUE} before any of the \acp{PE} in the
    \activeset{} enter the routine.}
    
\end{apiarguments}

\apidescription{
    The \FUNC{shmem\_alltoalls} routines are collective routines. Each \ac{PE}
    in the \activeset{} exchanges \VAR{nelems} strided data elements of size
    32 bits (for \FUNC{shmem\_alltoalls32}) or 64 bits (for \FUNC{shmem\_alltoalls64})
    with all other \acp{PE} in the set. Both strides, \VAR{dst} and \VAR{sst}, must be greater
    than or equal to \CONST{1}.
    Given a \ac{PE} \VAR{i} that is the \kth PE in the active set and a \ac{PE}
    \VAR{j} that is the \lth \ac{PE} in the active set,
    \ac{PE} \VAR{i} sends the \VAR{sst}*\lth block of the \VAR{source} data object to
    the \VAR{dst}*\kth block of the \VAR{dest} data object on
    \ac{PE} \VAR{j}.

    As with all \openshmem collective routines, these routines assume
    that only \acp{PE} in the \activeset{} call the routine.  If a \ac{PE} not
    in the \activeset{} calls an \openshmem collective routine, undefined
    behavior results.

    The values of arguments \VAR{dst}, \VAR{sst}, \VAR{nelems}, \VAR{PE\_start},
    \VAR{logPE\_stride}, and \VAR{PE\_size} must be equal on all \acp{PE} in the
    \activeset. The same \VAR{dest} and \VAR{source} data objects, and the same
    \VAR{pSync} work array must be passed to all \acp{PE} in the \activeset.
    
    Before any \ac{PE} calls to a \FUNC{shmem\_alltoalls} routine, the following
    conditions must exist (synchronization via a barrier or some other method is
    often needed to ensure this): The \VAR{pSync} array on all \acp{PE} in the
    \activeset{} is not still in use from a prior call to a
    \FUNC{shmem\_alltoalls} routine.  The \VAR{dest} data object on
    all \acp{PE} in the \activeset{} is ready to accept the
    \FUNC{shmem\_alltoalls} data.
    
    Upon return from a \FUNC{shmem\_alltoalls} routine, the following is true for
    the local PE: Its \VAR{dest} symmetric data object is completely updated and
    the data has been copied out of the \VAR{source} data object.
    The values in the \VAR{pSync} array are restored to the original values.
} 

\apidesctable{
The  \dest{}  and \source{} data  objects must conform to certain typing
constraints, which are as follows:
}{Routine}{Data type of \VAR{dest} and \VAR{source}}

\apitablerow{shmem\_alltoalls64}{\CONST{64} bits aligned.}
\apitablerow{shmem\_alltoalls32}{\CONST{32} bits aligned.}

\apireturnvalues{
    None.
}

\apinotes{
    This routine restores \VAR{pSync} to its original contents.  Multiple calls
    to \openshmem\ routines that use the same \VAR{pSync} array do not require
    that \VAR{pSync} be reinitialized after the first call.
    The user must ensure that the \VAR{pSync} array is not being updated by any
    \ac{PE} in the \activeset{} while any of the \acp{PE} participates in
    processing of an \openshmem\ \FUNC{shmem\_alltoalls} routine. Be careful to
    avoid these situations: If the \VAR{pSync} array is initialized at run time,
    some type of synchronization is needed to ensure that all \acp{PE} in the
    \activeset{} have initialized \VAR{pSync} before any of them enter an
    \openshmem\ routine called with the \VAR{pSync} synchronization array.  A
    \VAR{pSync} array may be reused on a subsequent \openshmem\
    \FUNC{shmem\_alltoalls} routine only if none of the \acp{PE} in the
    \activeset{} are still processing a prior \openshmem\ \FUNC{shmem\_alltoalls}
    routine call that used the same \VAR{pSync} array.  In general, this can be
    ensured only by doing some type of synchronization.        
}

\begin{apiexamples}

\apicexample
    {This example shows a \FUNC{shmem\_alltoalls64} on two long elements among
    all \acp{PE}.}
    {./example_code/shmem_alltoalls_example.c}
    {}

\end{apiexamples}

\end{apidefinition}


\subsubsection{\textbf{SHMEM\_BROADCAST}}\label{subsec:shmem_broadcast}
\input{content/shmem_broadcast.tex}

\subsubsection{\textbf{SHMEM\_COLLECT, SHMEM\_FCOLLECT}}\label{subsec:shmem_collect}
\apisummary{
    Concatenates blocks of data from multiple \acp{PE} to an array in every
    \ac{PE}.
}

\begin{apidefinition}

\begin{Csynopsis}
void shmem_collect32(void *dest, const void *source, size_t nelems, int PE_start, int logPE_stride, int PE_size, long *pSync);
void shmem_collect64(void *dest, const void *source, size_t nelems, int PE_start, int logPE_stride, int PE_size, long *pSync);
void shmem_fcollect32(void *dest, const void *source, size_t nelems, int PE_start, int logPE_stride, int PE_size, long *pSync);
void shmem_fcollect64(void *dest, const void *source, size_t nelems, int PE_start, int logPE_stride, int PE_size, long *pSync);
\end{Csynopsis}

\begin{Fsynopsis}
INTEGER nelems
INTEGER PE_start, logPE_stride, PE_size
INTEGER pSync(SHMEM_COLLECT_SYNC_SIZE)
CALL SHMEM_COLLECT4(dest, source, nelems, PE_start, logPE_stride, PE_size, pSync)
CALL SHMEM_COLLECT8(dest, source, nelems, PE_start, logPE_stride, PE_size, pSync)
CALL SHMEM_COLLECT32(dest, source, nelems, PE_start, logPE_stride, PE_size, pSync)
CALL SHMEM_COLLECT64(dest, source, nelems, PE_start, logPE_stride, PE_size, pSync)
CALL SHMEM_FCOLLECT4(dest, source, nelems, PE_start, logPE_stride, PE_size, pSync)
CALL SHMEM_FCOLLECT8(dest, source, nelems, PE_start, logPE_stride, PE_size, pSync)
CALL SHMEM_FCOLLECT32(dest, source, nelems, PE_start, logPE_stride, PE_size, pSync)
CALL SHMEM_FCOLLECT64(dest, source, nelems, PE_start, logPE_stride, PE_size, pSync)
\end{Fsynopsis}

\begin{apiarguments}

\apiargument{OUT}{dest}{A symmetric array. The \dest{} argument must be large enough
    to accept the concatenation of the \source{} arrays on all \acp{PE}.  The data
    types are as follows: For \FUNC{shmem\_collect8}, \FUNC{shmem\_collect64},
    \FUNC{shmem\_fcollect8}, and \FUNC{shmem\_fcollect64}, any data type with an
    element size of 64 bits.  \Fortran derived types, \Fortran character type,
    and \CorCpp{}  structures  are not permitted.  For \FUNC{shmem\_collect4},
    \FUNC{shmem\_collect32}, \FUNC{shmem\_fcollect4}, and \FUNC{shmem\_fcollect32},
    any data type with an element size of \CONST{32} bits.  \Fortran derived
    types, \Fortran character type, and \CorCpp{} structures are not permitted.}
\apiargument{IN}{source}{A symmetric data object that can be of any type permissible
    for the \dest{} argument.}
\apiargument{IN}{nelems}{The number of elements in the \source{} array. \VAR{nelems}
    must be of type \VAR{size\_t} for \Cstd. When using \Fortran, it must be
    a default integer value.}
\apiargument{IN}{PE\_start}{The lowest \ac{PE} number of the \activeset{} of
    \acp{PE}.  \VAR{PE\_start} must be of type integer.  When using \Fortran,
    it must be a default integer value.}
\apiargument{IN}{logPE\_stride}{The log (base \CONST{2}) of the stride between
    consecutive \ac{PE} numbers in the \activeset. \VAR{logPE\_stride} must be of
    type integer.  When using \Fortran, it must be a default integer value.}
\apiargument{IN}{PE\_size}{The number of \acp{PE} in the \activeset. \VAR{PE\_size}
    must be of type integer.  When using  \Fortran, it must be a default
    integer value.}
\apiargument{IN}{pSync}{A symmetric  work array.  In \CorCpp, \VAR{pSync} must be of
    type long and size \CONST{SHMEM\_COLLECT\_SYNC\_SIZE}.  In \Fortran,
    \VAR{pSync} must be of type integer and size \CONST{SHMEM\_COLLECT\_SYNC\_SIZE}.
    When using \Fortran, it must be a default integer value.  Every element of
    this array must be initialized with the value \CONST{SHMEM\_SYNC\_VALUE} in
    \CorCpp{} or \CONST{SHMEM\_SYNC\_VALUE} in \Fortran before any of the \acp{PE}
    in the \activeset{} enter \FUNC{shmem\_collect} or \FUNC{shmem\_fcollect}.}

\end{apiarguments}

\apidescription{   
    \openshmem \FUNC{collect} and \FUNC{fcollect} routines concatenate \VAR{nelems}
    \CONST{64}-bit or \CONST{32}-bit data items from the \source{} array into the
    \dest{} array, over the set of \acp{PE} defined by \VAR{PE\_start},
    \VAR{log2PE\_stride}, and \VAR{PE\_size}, in processor number order. The
    resultant \dest{} array contains the contribution from \ac{PE} \VAR{PE\_start}
    first, then the contribution from \ac{PE} \VAR{PE\_start} + \VAR{PE\_stride}
    second, and so on. The collected result is written to the \dest{} array for all
    \acp{PE} in the \activeset.
    
    The \FUNC{fcollect} routines require that \VAR{nelems} be the same value in all
    participating \acp{PE}, while the \FUNC{collect} routines allow \VAR{nelems} to
    vary from \ac{PE} to \ac{PE}.
    
    As with all \openshmem collective routines, each of these routines assumes that
    only \acp{PE} in the \activeset{} call the routine. If a \ac{PE} not in the
    \activeset{} and calls this collective routine, the behavior is undefined.
    
    The values of arguments \VAR{PE\_start}, \VAR{logPE\_stride}, and \VAR{PE\_size}
    must be equal on all \acp{PE} in the \activeset. The same \dest{} and \source{}
    arrays and the same \VAR{pSync} work array must be passed to all \acp{PE} in the
    \activeset.
    
    Upon return from a collective routine, the following are true for the local
    \ac{PE}: The \dest{} array is updated and the \source{} array may be safely reused. 
    The values in the \VAR{pSync} array are
    restored to the original values.
}

\apireturnvalues{
    None.
}

\apinotes{
    All \openshmem collective routines reset the values in \VAR{pSync} before they
    return, so a particular \VAR{pSync} buffer need only be initialized the first
    time it is used.
    
    The user must ensure that the \VAR{pSync} array is not being updated on any \ac{PE}
    in the \activeset{} while any of the \acp{PE} participate in processing of an
    \openshmem collective routine.  Be careful to avoid these situations: If the
    \VAR{pSync} array is initialized at run time, some type of synchronization is
    needed to ensure that all \acp{PE} in the working set have initialized
    \VAR{pSync} before any of them  enter an \openshmem routine called with the
    \VAR{pSync} synchronization array.  A \VAR{pSync} array can be reused on a
    subsequent \openshmem collective routine only if none of the \acp{PE} in the
    \activeset{}  are still processing a  prior \openshmem collective routine call
    that used the same \VAR{pSync} array.  In general, this may be ensured only by
    doing some type of synchronization.  
    
    The collective routines operate on active \ac{PE} sets that have a
    non-power-of-two \VAR{PE\_size} with some performance degradation.  They operate
    with no performance degradation when \VAR{nelems} is a non-power-of-two value.
}

\begin{apiexamples}

\apicexample
    {The following \FUNC{shmem\_collect} example is for \CorCpp{} programs:}
    {./example_code/shmem_collect_example.c}
    {}

\apifexample
    {The following \FUNC{SHMEM\_COLLECT} example is for \Fortran programs:}
    {./example_code/shmem_collect_example.f90}
    {}

\end{apiexamples}

\end{apidefinition}


\subsubsection{\textbf{SHMEM\_REDUCTIONS}}\label{subsec:shmem_reductions}
\apisummary{
    Performs arithmetic and logical operations across a set of \acp{PE}.
}

\begin{apidefinition}

\textbf{AND} \newline
Performs a bitwise AND function across a set of processing elements (\acp{PE}).\newline
\begin{Csynopsis}
void shmem_short_and_to_all(short *dest, const short *source, int nreduce, int PE_start, int logPE_stride, int PE_size, short *pWrk, long *pSync);
void shmem_int_and_to_all(int *dest, const int *source, int nreduce, int PE_start, int logPE_stride, int PE_size, int *pWrk, long *pSync);
void shmem_long_and_to_all(long *dest, const long *source, int nreduce, int PE_start, int logPE_stride, int PE_size, long *pWrk, long *pSync);
void shmem_longlong_and_to_all(long long *dest, const long long *source, int nreduce, int PE_start, int logPE_stride, int PE_size, long long *pWrk, long *pSync);
\end{Csynopsis}

\begin{Fsynopsis}
CALL SHMEM_INT4_AND_TO_ALL(dest, source, nreduce, PE_start, logPE_stride, PE_size, pWrk, pSync)
CALL SHMEM_INT8_AND_TO_ALL(dest, source, nreduce, PE_start, logPE_stride, PE_size, pWrk, pSync)
\end{Fsynopsis}

\bigskip
\textbf{MAX} \newline
Performs a maximum function reduction across a set of processing elements (\acp{PE}).\newline
\begin{Csynopsis}
void shmem_short_max_to_all(short *dest, const short *source, int nreduce, int PE_start, int logPE_stride, int PE_size, short *pWrk, long *pSync);
void shmem_int_max_to_all(int *dest, const int *source, int nreduce, int PE_start, int logPE_stride, int PE_size, int *pWrk, long *pSync);
void shmem_double_max_to_all(double *dest, const double *source, int nreduce, int PE_start, int logPE_stride, int PE_size, double *pWrk, long *pSync);
void shmem_float_max_to_all(float *dest, const float *source, int nreduce, int PE_start, int logPE_stride, int PE_size, float *pWrk, long *pSync);
void shmem_long_max_to_all(long *dest, const long *source, int nreduce, int PE_start, int logPE_stride, int PE_size, long *pWrk, long *pSync);
void shmem_longdouble_max_to_all(long double *dest, const long double *source, int nreduce, int PE_start, int logPE_stride, int PE_size, long double *pWrk, long *pSync);
void shmem_longlong_max_to_all(long long *dest, const long long *source, int nreduce, int PE_start, int logPE_stride, int PE_size, long long *pWrk, long *pSync);
\end{Csynopsis}

\begin{Fsynopsis}
CALL SHMEM_INT4_MAX_TO_ALL(dest, source, nreduce, PE_start, logPE_stride, PE_size, pWrk, pSync)
CALL SHMEM_INT8_MAX_TO_ALL(dest, source, nreduce, PE_start, logPE_stride, PE_size, pWrk, pSync)
CALL SHMEM_REAL4_MAX_TO_ALL(dest, source, nreduce, PE_start, logPE_stride, PE_size, pWrk, pSync)
CALL SHMEM_REAL8_MAX_TO_ALL(dest, source, nreduce, PE_start, logPE_stride, PE_size, pWrk, pSync)
CALL SHMEM_REAL16_MAX_TO_ALL(dest, source, nreduce, PE_start, logPE_stride, PE_size, pWrk, pSync)
\end{Fsynopsis}

\bigskip
\textbf{MIN} \newline
Performs a minimum function reduction across a set of processing elements (\acp{PE}).\newline
\begin{Csynopsis}
void shmem_short_min_to_all(short *dest, const short *source, int nreduce, int PE_start, int logPE_stride, int PE_size, short *pWrk, long *pSync);
void shmem_int_min_to_all(int *dest, const int *source, int nreduce, int PE_start, int logPE_stride, int PE_size, int *pWrk, long *pSync);
void shmem_double_min_to_all(double *dest, const double *source, int nreduce, int PE_start, int logPE_stride, int PE_size, double *pWrk, long *pSync);
void shmem_float_min_to_all(float *dest, const float *source, int nreduce, int PE_start, int logPE_stride, int PE_size, float *pWrk, long *pSync);
void shmem_long_min_to_all(long *dest, const long *source, int nreduce, int PE_start, int logPE_stride, int PE_size, long *pWrk, long *pSync);
void shmem_longdouble_min_to_all(long double *dest, const long double *source, int nreduce, int PE_start, int logPE_stride, int PE_size, long double *pWrk, long *pSync);
void shmem_longlong_min_to_all(long long *dest, const long long *source, int nreduce, int PE_start, int logPE_stride, int PE_size, long long *pWrk, long *pSync);
\end{Csynopsis}

\begin{Fsynopsis}
CALL SHMEM_INT4_MIN_TO_ALL(dest, source, nreduce, PE_start, logPE_stride, PE_size, pWrk, pSync)
CALL SHMEM_INT8_MIN_TO_ALL(dest, source, nreduce, PE_start, logPE_stride, PE_size, pWrk, pSync)
CALL SHMEM_REAL4_MIN_TO_ALL(dest, source, nreduce, PE_start, logPE_stride, PE_size, pWrk, pSync)
CALL SHMEM_REAL8_MIN_TO_ALL(dest, source, nreduce, PE_start, logPE_stride, PE_size, pWrk, pSync)
CALL SHMEM_REAL16_MIN_TO_ALL(dest, source, nreduce, PE_start, logPE_stride, PE_size, pWrk, pSync)
\end{Fsynopsis}

\bigskip
\textbf{SUM} \newline
Performs a sum reduction across a set of processing elements (\acp{PE}).\newline
\begin{Csynopsis}
void shmem_complexd_sum_to_all(double _Complex *dest, const double _Complex *source, int nreduce, int PE_start, int logPE_stride, int PE_size, double _Complex *pWrk, long |\mbox{*pSync);}|
void shmem_complexf_sum_to_all(float _Complex *dest, const float _Complex *source, int nreduce, int PE_start, int logPE_stride, int PE_size, float _Complex *pWrk, long *pSync);
void shmem_short_sum_to_all(short *dest, const short *source, int nreduce, int PE_start, int logPE_stride, int PE_size, short *pWrk, long *pSync);
void shmem_int_sum_to_all(int *dest, const int *source, int nreduce, int PE_start, int logPE_stride, int PE_size, int *pWrk, long *pSync);
void shmem_double_sum_to_all(double *dest, const double *source, int nreduce, int PE_start, int logPE_stride, int PE_size, double *pWrk, long *pSync);
void shmem_float_sum_to_all(float *dest, const float *source, int nreduce, int PE_start, int logPE_stride, int PE_size, float *pWrk, long *pSync);
void shmem_long_sum_to_all(long *dest, const long *source, int nreduce, int PE_start, int logPE_stride,int PE_size, long *pWrk, long *pSync);
void shmem_longdouble_sum_to_all(long double *dest, const long double *source, int nreduce, int PE_start, int logPE_stride, int PE_size, long double *pWrk, long *pSync);
void shmem_longlong_sum_to_all(long long *dest, const long long *source, int nreduce, int PE_start, int logPE_stride, int PE_size, long long *pWrk, long *pSync);
\end{Csynopsis}

\begin{Fsynopsis}
CALL SHMEM_COMP4_SUM_TO_ALL(dest, source, nreduce, PE_start, logPE_stride, PE_size, pWrk, pSync)
CALL SHMEM_COMP8_SUM_TO_ALL(dest, source, nreduce, PE_start, logPE_stride, PE_size, pWrk, pSync)
CALL SHMEM_INT4_SUM_TO_ALL(dest, source, nreduce, PE_start, logPE_stride, PE_size, pWrk, pSync)
CALL SHMEM_INT8_SUM_TO_ALL(dest, source, nreduce, PE_start, logPE_stride, PE_size, pWrk, pSync)
CALL SHMEM_REAL4_SUM_TO_ALL(dest, source, nreduce, PE_start, logPE_stride, PE_size, pWrk, pSync)
CALL SHMEM_REAL8_SUM_TO_ALL(dest, source, nreduce, PE_start, logPE_stride, PE_size, pWrk, pSync)
CALL SHMEM_REAL16_SUM_TO_ALL(dest, source, nreduce, PE_start, logPE_stride, PE_size, pWrk, pSync)
\end{Fsynopsis}

\bigskip
\textbf{PROD} \newline
Performs a product reduction across a set of processing elements (\acp{PE}).\newline
\begin{Csynopsis}
void shmem_complexd_prod_to_all(double _Complex *dest, const double _Complex *source, int nreduce, int PE_start, int logPE_stride, int PE_size, double _Complex *pWrk, long |\mbox{*pSync);}|
void shmem_complexf_prod_to_all(float _Complex *dest, const float _Complex *source, int |\mbox{nreduce,}| int PE_start, int logPE_stride, int PE_size, float _Complex *pWrk, long *pSync);
void shmem_short_prod_to_all(short *dest, const short *source, int nreduce, int PE_start, int logPE_stride, int PE_size, short *pWrk, long *pSync);
void shmem_int_prod_to_all(int *dest, const int *source, int nreduce, int PE_start, int logPE_stride, int PE_size, int *pWrk, long *pSync);
void shmem_double_prod_to_all(double *dest, const double *source, int nreduce, int PE_start, int logPE_stride, int PE_size, double *pWrk, long *pSync);
void shmem_float_prod_to_all(float *dest, const float *source, int nreduce, int PE_start, int logPE_stride, int PE_size, float *pWrk, long *pSync);
void shmem_long_prod_to_all(long *dest, const long *source, int nreduce, int PE_start, int logPE_stride, int PE_size, long *pWrk, long *pSync);
void shmem_longdouble_prod_to_all(long double *dest, const long double *source, int nreduce, int PE_start, int logPE_stride, int PE_size, long double *pWrk, long *pSync);
void shmem_longlong_prod_to_all(long long *dest, const long long *source, int nreduce, int PE_start, int logPE_stride, int PE_size, long long *pWrk, long *pSync);
\end{Csynopsis}

\begin{Fsynopsis}
CALL SHMEM_COMP4_PROD_TO_ALL(dest, source, nreduce, PE_start, logPE_stride, PE_size, pWrk, pSync)
CALL SHMEM_COMP8_PROD_TO_ALL(dest, source, nreduce, PE_start, logPE_stride, PE_size, pWrk, pSync)
CALL SHMEM_INT4_PROD_TO_ALL(dest, source, nreduce, PE_start, logPE_stride, PE_size, pWrk, pSync)
CALL SHMEM_INT8_PROD_TO_ALL(dest, source, nreduce, PE_start, logPE_stride, PE_size, pWrk, pSync)
CALL SHMEM_REAL4_PROD_TO_ALL(dest, source, nreduce, PE_start, logPE_stride, PE_size, pWrk, pSync)
CALL SHMEM_REAL8_PROD_TO_ALL(dest, source, nreduce, PE_start, logPE_stride, PE_size, pWrk, pSync)
CALL SHMEM_REAL16_PROD_TO_ALL(dest, source, nreduce, PE_start, logPE_stride, PE_size, pWrk, pSync)
\end{Fsynopsis}

\bigskip
\textbf{OR} \newline
Performs  a  bitwise  OR  function reduction across a set of processing elements (\acp{PE}).\newline
\begin{Csynopsis}
void shmem_short_or_to_all(short *dest, const short *source, int nreduce, int PE_start, int logPE_stride, int PE_size, short *pWrk, long *pSync);
void shmem_int_or_to_all(int *dest, const int *source, int nreduce, int PE_start, int logPE_stride, int PE_size, int *pWrk, long *pSync);
void shmem_long_or_to_all(long *dest, const long *source, int nreduce, int PE_start, int logPE_stride, int PE_size, long *pWrk, long *pSync);
void shmem_longlong_or_to_all(long long *dest, const long long *source, int nreduce, int PE_start, int logPE_stride, int PE_size, long long *pWrk, long *pSync);
\end{Csynopsis}

\begin{Fsynopsis}
CALL SHMEM_INT4_OR_TO_ALL(dest, source, nreduce, PE_start, logPE_stride, PE_size, pWrk, |\mbox{pSync)}|
CALL SHMEM_INT8_OR_TO_ALL(dest, source, nreduce, PE_start, logPE_stride, PE_size, pWrk, |\mbox{pSync)}|	
\end{Fsynopsis}

\bigskip
\textbf{XOR}\newline
Performs  a  bitwise  EXCLUSIVE OR reduction across a set of processing elements (\acp{PE}).\newline
\begin{Csynopsis}
void shmem_short_xor_to_all(short *dest, const short *source, int nreduce, int PE_start, int logPE_stride, int PE_size, short *pWrk, long *pSync);
void shmem_int_xor_to_all(int *dest, const int *source, int nreduce, int PE_start, int logPE_stride, int PE_size, int *pWrk, long *pSync);
void shmem_long_xor_to_all(long *dest, const long *source, int nreduce, int PE_start, int logPE_stride, int PE_size, long *pWrk, long *pSync);
void shmem_longlong_xor_to_all(long long *dest, const long long *source, int nreduce, int PE_start, int logPE_stride, int PE_size, long long *pWrk, long *pSync);
\end{Csynopsis}

\begin{Fsynopsis}
CALL SHMEM_INT4_XOR_TO_ALL(dest, source, nreduce, PE_start, logPE_stride, PE_size, pWrk, pSync)
CALL SHMEM_INT8_XOR_TO_ALL(dest, source, nreduce, PE_start, logPE_stride, PE_size, pWrk, pSync)
\end{Fsynopsis}

\begin{apiarguments}

\apiargument{IN}{dest}{A symmetric array, of length \VAR{nreduce} elements, to
    receive the result of the reduction routines.  The data type of \dest{} varies
    with the version of the reduction routine being called.  When calling from
    \CorCpp, refer to the SYNOPSIS section for data type information.}
\apiargument{IN}{source}{ A symmetric array, of length \VAR{nreduce} elements, that
    contains one element for each separate reduction routine.  The \source{}
    argument must have the same data type as \dest.}
\apiargument{IN}{nreduce}{The number of elements in the \dest{} and \source{}
    arrays.  \VAR{nreduce} must be of type integer.  When using \Fortran, it
    must be a default integer value.}
\apiargument{IN}{PE\_start}{The lowest \ac{PE} number of the \activeset{} of
    \acp{PE}.  \VAR{PE\_start} must be of type integer.  When using \Fortran,
    it must be a default integer value.}
\apiargument{IN}{logPE\_stride}{The log (base 2) of the stride between consecutive
    \ac{PE} numbers in the \activeset.  \VAR{logPE\_stride} must be of type integer.
    When using \Fortran, it must be a default integer value.}
\apiargument{IN}{PE\_size}{The number of \acp{PE} in the \activeset.
    \VAR{PE\_size} must be of type integer.  When using \Fortran, it must be a
    default integer value.}
\apiargument{IN}{pWrk}{A symmetric work array. The \VAR{pWrk} argument must have the
    same data type as \dest. In \CorCpp, this contains max(\VAR{nreduce}/2 + 1,
    \CONST{SHMEM\_REDUCE\_MIN\_WRKDATA\_SIZE}) elements. In \Fortran, this
    contains max(\VAR{nreduce}/2 + 1, \CONST{SHMEM\_REDUCE\_MIN\_WRKDATA\_SIZE})
    elements.}
\apiargument{IN}{pSync}{A symmetric work array. In \CorCpp, \VAR{pSync} must be of
    type long and size \CONST{SHMEM\_REDUCE\_SYNC\_SIZE}. In \Fortran, \VAR{pSync}
    must be of type integer and size \CONST{SHMEM\_REDUCE\_SYNC\_SIZE}.  When
    using \Fortran, it must be a default integer value. Every element of this array
    must be initialized with the value \CONST{SHMEM\_SYNC\_VALUE} (in \CorCpp) or
    \CONST{SHMEM\_SYNC\_VALUE} (in \Fortran) before any of the \acp{PE} in the
    \activeset{} enter the reduction routine.}
    
\end{apiarguments}

\apidescription{
    \openshmem reduction routines compute one or more reductions across symmetric
    arrays on multiple \acp{PE}.  A reduction performs an associative binary routine
    across a set of values.	 
    
    The \VAR{nreduce} argument determines the number of separate reductions to
    perform.  The \source{} array on all \acp{PE} in the \activeset{} provides one
    element for each reduction.  The results of the reductions are placed in the
    \dest{} array on all \acp{PE} in the \activeset.  The \activeset{} is defined
    by the \VAR{PE\_start}, \VAR{logPE\_stride}, \VAR{PE\_size} triplet.
    
    The \source{} and \dest{} arrays may be the same array, but they may not be
    overlapping arrays.
    
    As with all \openshmem collective routines, each of these routines assumes
    that only \acp{PE} in the \activeset{} call the routine.  If a \ac{PE} not in
    the \activeset{} calls an \openshmem collective routine, undefined behavior
    results.
    
    The values of arguments \VAR{nreduce}, \VAR{PE\_start}, \VAR{logPE\_stride}, and
    \VAR{PE\_size} must be equal on all \acp{PE} in the \activeset. The same \dest{}
    and \source{} arrays, and the same \VAR{pWrk} and \VAR{pSync} work arrays, must
    be passed to all \acp{PE} in the \activeset.
    %FIXME: Reword 'the following conditions must be met.'
    Before any \ac{PE} calls a reduction routine, the
    following conditions must be met (synchronization via a \OPR{barrier} or some other
    method is often needed to ensure this): The \VAR{pWrk} and \VAR{pSync} arrays
    on all \acp{PE} in the \activeset{} are not still in use from a prior call to a
    collective \openshmem routine.  The \dest{} array on all \acp{PE} in the
    \activeset{} is ready to accept the results of the \OPR{reduction}.
    
    Upon return from a reduction routine, the following are true for the local
    \ac{PE}: The \dest{} array is updated and the \source{} array may be safely reused.  
    The values in the \VAR{pSync} array are
    restored to the original values.

    The sum and product reduction routines include complex-typed interfaces
    for the \Cstd API only.
    When the \Cstd translation environment does not support complex types%
    \footnote{That is, under \Cstd language standards prior to \Cstd[99] or
      under \Cstd[11] when \CONST{\_\_STDC\_NO\_COMPLEX\_\_} is defined to 1},
    an \openshmem implementation is not required to provide support for
    these complex-typed interfaces.
}

\apidesctable{ 
    When calling from \Fortran, the \dest{} date types are as follows:
}{Routine}{Data type}
    \apitablerow{shmem\_int8\_and\_to\_all}{Integer, with an element size of 8 bytes.}
    \apitablerow{shmem\_int4\_and\_to\_all}{Integer, with an element size of 4 bytes.}
    \apitablerow{shmem\_comp8\_max\_to\_all}{Complex, with an element size equal to two 8-byte real values.}
    \apitablerow{shmem\_int4\_max\_to\_all}{Integer, with an element size of 4 bytes.}
    \apitablerow{shmem\_int8\_max\_to\_all}{Integer, with an element size of 8 bytes.}
    \apitablerow{shmem\_real4\_max\_to\_all}{Real, with an element size of 4 bytes.}
    \apitablerow{shmem\_real16\_max\_to\_all}{Real, with an element size of 16 bytes.}
    \apitablerow{shmem\_int4\_min\_to\_all}{Integer, with an element size of 4 bytes.}
    \apitablerow{shmem\_int8\_min\_to\_all}{Integer, with an element size of 8 bytes.}
    \apitablerow{shmem\_real4\_min\_to\_all}{Real, with an element size of 4 bytes.}
    \apitablerow{shmem\_real8\_min\_to\_all}{Real, with an element size of 8 bytes.}
    \apitablerow{shmem\_real16\_min\_to\_all}{Real,with an element size of 16 bytes.}
    \apitablerow{shmem\_comp4\_sum\_to\_all}{Complex, with an element size equal to two 4-byte real values.}
    \apitablerow{shmem\_comp8\_sum\_to\_all}{Complex, with an element size equal to two 8-byte real values.}
    \apitablerow{shmem\_int4\_sum\_to\_all}{Integer, with an element size of 4 bytes.}
    \apitablerow{shmem\_int8\_sum\_to\_all}{Integer, with an element size of 8 bytes..}
    \apitablerow{shmem\_real4\_sum\_to\_all}{Real, with an element size of 4 bytes.}
    \apitablerow{shmem\_real8\_sum\_to\_all}{Real, with an element size of 8 bytes.}
    \apitablerow{shmem\_real16\_sum\_to\_all}{Real, with an element size of 16 bytes.}
    \apitablerow{shmem\_comp4\_prod\_to\_all}{ Complex, with an element size equal to two 4-byte real values. }		 
    \apitablerow{shmem\_comp8\_prod\_to\_all}{ Complex, with an element size equal to two 8-byte real values.}
    \apitablerow{shmem\_int4\_prod\_to\_all}{Integer, with an element size of 4 bytes.}
    \apitablerow{shmem\_int8\_prod\_to\_all}{Integer, with an element size of 8 bytes.}
    \apitablerow{shmem\_real4\_prod\_to\_all}{Real, with an element size of 4 bytes.}
    \apitablerow{shmem\_real8\_prod\_to\_all}{Real, with an element size of 8 bytes.}
    \apitablerow{shmem\_real16\_prod\_to\_all}{Real, with an element size of 16 bytes.}
    \apitablerow{shmem\_int8\_or\_to\_all}{Integer, with an element size of 8 bytes.}
    \apitablerow{shmem\_int4\_or\_to\_all}{Integer, with an element size of 4 bytes.}
    \apitablerow{shmem\_int8\_xor\_to\_all}{Integer, with an element size of 8 bytes.}
    \apitablerow{shmem\_int4\_xor\_to\_all}{Integer, with an element size of 4 bytes.}

\apireturnvalues{
    None.
}

\apinotes{  
    All \openshmem reduction routines reset the values in \VAR{pSync} before they
    return, so a particular \VAR{pSync} buffer need only be initialized the first
    time it is used. The user must ensure that the \VAR{pSync} array is not being updated on any \ac{PE}
    in the \activeset{} while any of the \acp{PE} participate in processing of an
    \openshmem reduction routine. Be careful to avoid the following situations: If
    the \VAR{pSync} array is initialized at run time, some type of synchronization
    is needed to ensure that all \acp{PE} in the working set have initialized
    \VAR{pSync} before any of them enter an \openshmem routine called with the
    \VAR{pSync} synchronization array. A \VAR{pSync} or \VAR{pWrk} array can be
    reused in a subsequent reduction routine call only if none of the \acp{PE} in
    the \activeset{} are still processing a prior reduction routine call that used
    the same \VAR{pSync} or \VAR{pWrk} arrays. In general, this can be assured only
    by doing some type of synchronization. 
}

\begin{apiexamples}

\apifexample
    {This \Fortran reduction example statically initializes the \VAR{pSync} array
    and finds the logical \OPR{AND} of the integer variable \VAR{FOO} across all
    even \acp{PE}.}
    {./example_code/shmem_and_example.f90}
    {}
    
\apifexample
    {This \Fortran example statically initializes the \VAR{pSync} array and finds
    the \OPR{maximum} value of real variable \VAR{FOO} across all even \acp{PE}.}
    {./example_code/shmem_max_example.f90}
    {}

\apifexample
    { This \Fortran example statically initializes the \VAR{pSync} array and finds
    the \OPR{minimum} value of real variable \VAR{FOO} across all the even
    \acp{PE}.}
    {./example_code/shmem_min_example.f90}
    {}

\apifexample
    {This \Fortran example statically initializes the \VAR{pSync} array and finds
    the \OPR{sum} of the real variable \VAR{FOO} across all even \acp{PE}.}
    {./example_code/shmem_sum_example.f90}
    {}

\apifexample
    {This \Fortran example statically initializes the \VAR{pSync} array and finds
    the \OPR{product} of the real variable \VAR{FOO} across all the even \acp{PE}.}
    {./example_code/shmem_prod_example.f90}
    {}

\apifexample
    {This \Fortran example statically initializes the \VAR{pSync} array and finds
    the logical \OPR{OR} of the integer variable \VAR{FOO} across all even
    \acp{PE}.}
    {./example_code/shmem_or_example.f90}
    {}

\apifexample
    {This \Fortran example statically initializes the \VAR{pSync} array and
    computes the exclusive \OPR{XOR} of variable \VAR{FOO} across all even
    \acp{PE}.}
    {./example_code/shmem_xor_example.f90}
    {} 

\end{apiexamples}

\end{apidefinition}





\subsection{Point-To-Point Synchronization Routines}\label{subsec:p2p_intro}
The following section discusses \openshmem \acp{API} that provides a mechanism
for synchronization between two \acp{PE} based on the value of a symmetric data
object.
The point-to-point synchronization routines can be used to portably ensure
that memory access operations observe remote updates in the order enforced by
the initiator \ac{PE} using the \FUNC{shmem\_fence} and \FUNC{shmem\_quiet}
routines.

Where appropriate compiler support is available, \openshmem provides
type-generic point-to-point synchronization interfaces via \Cstd[11] generic
selection. Such type-generic routines are supported for the
``point-to-point synchronization types'' identified in
Table~\ref{p2psynctypes}.

% Fix the C99 styling here with the macros from PRs #55 and #31, and
% osh_spec_next for \HEADER{}
The point-to-point synchronization types include some of the exact-width
integer types defined in \HEADER{stdint.h} by \Cstd[99]~\S7.18.1.1 and
\Cstd[11]~\S7.20.1.1. When the \Cstd translation environment
does not provide exact-width integer types with \HEADER{stdint.h}, an
\openshmem implemementation is not required to provide support for these types.

\begin{table}[h]
  \begin{center}
    \begin{tabular}{|l|l|}
      \hline
      \TYPE              & \TYPENAME  \\ \hline
      short              & short      \\ \hline
      int                & int        \\ \hline
      long               & long       \\ \hline
      long long          & longlong   \\ \hline
      unsigned short     & ushort     \\ \hline
      unsigned int       & uint       \\ \hline
      unsigned long      & ulong      \\ \hline
      unsigned long long & ulonglong  \\ \hline
      int32\_t           & int32      \\ \hline
      int64\_t           & int64      \\ \hline
      uint32\_t          & uint32     \\ \hline
      uint64\_t          & uint64     \\ \hline
      size\_t            & size       \\ \hline
      ptrdiff\_t         & ptrdiff    \\ \hline
    \end{tabular}
    \caption{Point-to-Point Synchronization Types and Names}
    \label{p2psynctypes}
  \end{center}
\end{table}

The point-to-point synchronization interface provides the enumerated type
\CTYPE{shmem\_cmp\_t}, whose enumerators specify the comparison operators used
by synchronization routines that take a \CTYPE{shmem\_cmp\_t} argument. The
enumerators of \CTYPE{shmem\_cmp\_t} and their associated operations are
presented in Table~\ref{p2p-consts}.  For Fortran, the constant names of
Table~\ref{p2p-consts} shall be identifiers for integer parameters of
default kind corresponding to the associated comparison operation.

\begin{table}[h]
  \begin{center}
    \begin{tabular}{ll}
      \hline
      Constant Name    & Comparison               \\ \hline
      SHMEM\_CMP\_EQ   & Equal                    \\
      SHMEM\_CMP\_NE   & Not equal                \\
      SHMEM\_CMP\_GT   & Greater than             \\
      SHMEM\_CMP\_GE   & Greater than or equal to \\
      SHMEM\_CMP\_LT   & Less than                \\
      SHMEM\_CMP\_LE   & Less than or equal to    \\ \hline
    \end{tabular}
    \caption{Point-to-Point Comparison Enumeration Constants}
    \label{p2p-consts}
  \end{center}
\end{table}


\subsubsection{\textbf{SHMEM\_WAIT\_UNTIL}}\label{subsec:shmem_wait_until}
\input{content/shmem_wait_until.tex}

\subsubsection{\textbf{SHMEM\_WAIT\_UNTIL\_ALL}}\label{subsec:shmem_wait_until_all}
\input{content/shmem_wait_until_all.tex}

\subsubsection{\textbf{SHMEM\_WAIT\_UNTIL\_ANY}}\label{subsec:shmem_wait_until_any}
\input{content/shmem_wait_until_any.tex}

\subsubsection{\textbf{SHMEM\_WAIT\_UNTIL\_SOME}}\label{subsec:shmem_wait_until_some}
\input{content/shmem_wait_until_some.tex}

\subsubsection{\textbf{SHMEM\_WAIT\_UNTIL\_ALL\_VECTOR}}\label{subsec:shmem_wait_until_all_vector}
\input{content/shmem_wait_until_all_vector.tex}

\subsubsection{\textbf{SHMEM\_WAIT\_UNTIL\_ANY\_VECTOR}}\label{subsec:shmem_wait_until_any_vector}
\input{content/shmem_wait_until_any_vector.tex}

\subsubsection{\textbf{SHMEM\_WAIT\_UNTIL\_SOME\_VECTOR}}\label{subsec:shmem_wait_until_some_vector}
\input{content/shmem_wait_until_some_vector.tex}

\subsubsection{\textbf{SHMEM\_TEST}}\label{subsec:shmem_test}
\input{content/shmem_test.tex}

\subsubsection{\textbf{SHMEM\_TEST\_ALL}}\label{subsec:shmem_test_all}
\input{content/shmem_test_all.tex}

\subsubsection{\textbf{SHMEM\_TEST\_ANY}}\label{subsec:shmem_test_any}
\input{content/shmem_test_any.tex}

\subsubsection{\textbf{SHMEM\_TEST\_SOME}}\label{subsec:shmem_test_some}
\input{content/shmem_test_some.tex}

\subsubsection{\textbf{SHMEM\_TEST\_ALL\_VECTOR}}\label{subsec:shmem_test_all_vector}
\input{content/shmem_test_all_vector.tex}

\subsubsection{\textbf{SHMEM\_TEST\_ANY\_VECTOR}}\label{subsec:shmem_test_any_vector}
\input{content/shmem_test_any_vector.tex}

\subsubsection{\textbf{SHMEM\_TEST\_SOME\_VECTOR}}\label{subsec:shmem_test_some_vector}
\input{content/shmem_test_some_vector.tex}

\subsubsection{\textbf{SHMEM\_SIGNAL\_WAIT\_UNTIL}}\label{subsec:shmem_signal_wait_until}
\input{content/shmem_signal_wait_until.tex}



\subsection{Memory Ordering Routines}\label{subsec:memory_order}
The following section discusses \openshmem \acp{API} that provide mechanisms to
ensure ordering and/or delivery of completion on memory store, blocking, and
nonblocking \openshmem routines. Table~\ref{mem-order} lists the operations
affected by \openshmem memory ordering routines.

\begin{longtable}{|l|c|c|}
 \hline
  Operations & Fence & Quiet \\ \hline
  Memory Store \footnote{Ordering and/or delivery of memory store operations are
  ensured only on contexts created with certain options. For details, refer the
  description of context options in Section~\ref{subsec:shmem_ctx_create}.}
                                     & X & X \\
  Blocking \OPR{Put}                 & X & X \\
  Blocking \OPR{Get}                 &   &   \\
  Blocking \OPR{\acs{AMO}}           & X & X \\
  Blocking \OPR{put-with-signal}     & X & X \\
  Nonblocking \OPR{Put}              & X & X \\
  Nonblocking \OPR{Get}              &   & X \\
  Nonblocking \OPR{\acs{AMO}}        & X\footnote{\openshmem fence routines does
  not guarantee order of delivery of values fetched by nonblocking \ac{AMO}
  routines.}
                                         & X \\
  Nonblocking \OPR{put-with-signal}  & X & X \\ \hline
\TableCaptionRef{List of operations affected by \openshmem Memory Ordering
routines}
\label{mem-order}
\end{longtable}



\subsubsection{\textbf{SHMEM\_FENCE}}\label{subsec:shmem_fence}
\input{content/shmem_fence.tex}

\subsubsection{\textbf{SHMEM\_QUIET}}\label{subsec:shmem_quiet}
\input{content/shmem_quiet.tex}

\subsubsection{Synchronization and Communication Ordering in OpenSHMEM}
\input{content/synchronization_model.tex}






\subsection{Distributed Locking Routines}
The following section discusses \openshmem locks as a mechanism to provide
mutual exclusion. Three routines are available for distributed locking,
\textit{set, test} and \textit{clear}.

\subsubsection{\textbf{SHMEM\_LOCK}}\label{subsec:shmem_lock}
\input{content/shmem_lock.tex}





\section{OpenSHMEM Profiling Interface}\label{sec:openshmem_profiling_interface}
\input{content/profiling_interface.tex}

\clearpage
\clearpage %%%%%%%%%%%%%%%%%%%%%%%%%%%%%%%%%%%%%%%%%%%%%%%%%%%%%%%%%%%%

\appendix

%defining pagestyle for annex
%\pagestyle{plain} \withlinenumbers
\pagestyle{fancy} \withlinenumbers
\fancyhf{}
\fancyhead[RE, LO]{\leftmark}
\fancyhead[RO, LE]{\thepage}
\fancyfoot[CE, CO]{\thepage}
\renewcommand{\headrulewidth}{0pt}




\chapter{Writing OpenSHMEM Programs}
\section*{Incorporating OpenSHMEM into Programs}\label{sec:writing_programs}

The following section describes how to write a ``Hello World" \openshmem program.
To write a ``Hello World" \openshmem program, the user must:

\begin{itemize}
\item Include the header file \HEADER{shmem.h} for \Cstd.
\item Add the initialization call \hyperref[subsec:shmem_init]{\FUNC{shmem\_init}}.
\item Use \openshmem calls to query the local \ac{PE} number
    (\hyperref[subsec:shmem_my_pe]{\FUNC{shmem\_my\_pe}}) and the total number
    of \acp{PE} (\hyperref[subsec:shmem_n_pes]{\FUNC{shmem\_n\_pes}}).
\item Add the finalization call \hyperref[subsec:shmem_finalize]{\FUNC{shmem\_finalize}}.
\end{itemize}

In \openshmem, the order in which lines appear in the output is not
deterministic because \acp{PE} execute asynchronously in parallel.

\begin{minipage}{\linewidth}
\vspace{0.1in}
\numberedlisting{caption={``Hello World'' example program in \Cstd},label=openshmem-hello,language=OSH2+C}
                {example_code/hello-openshmem.c}
\outputlisting{language=bash,caption={Possible ordering of expected output with 4 \acp{PE} from the program in Listing~\ref{openshmem-hello}}}
                {example_code/hello-openshmem-c.output}
\vspace{0.1in}
\end{minipage}

\clearpage %%%%%%%%%%%%%%%%%%%%%%%%%%%%%%%%%%%%%%%%%%%%%%%%%%%%%%%%%%%%

The example in Listing~\ref{openshmem-hello-symmetric} shows a more complex
\openshmem program that illustrates the use of symmetric data objects.
Note the declaration of the \VAR{static short dest} array and its use as the
remote destination in \hyperref[subsec:shmem_put]{\FUNC{shmem\_put}}.

The \KEYWORD{static} keyword makes the \VAR{dest} array symmetric on all \acp{PE}.
Each \ac{PE} is able to transfer data to a remote \dest{} array by simply
specifying to an OpenSHMEM routine such as \hyperref[subsec:shmem_put]{\FUNC{shmem\_put}}
the local address of the symmetric data object that will receive the data.
This local address resolution aids programmability because the address of the
\dest{} need not be exchanged with the active side (\ac{PE} \CONST{0}) prior to
the \acf{RMA} routine.

Conversely, the declaration of the \VAR{short source} array is asymmetric
(local only).
The \source{} object does not need to be symmetric because \PUT{} handles the
references to the \VAR{source} array only on the active (local) side.

\begin{minipage}{\linewidth}
\vspace{0.1in}
\numberedlisting{caption={Example program with symmetric data objects},label=openshmem-hello-symmetric,language=OSH2+C}
                {example_code/writing_shmem_example.c}
\outputlisting{language=bash,caption={Possible ordering of expected output with 4 \acp{PE} from the program in Listing~\ref{openshmem-hello-symmetric}}}
                {example_code/writing_shmem_example.output}
\vspace{0.1in}
\end{minipage}




\chapter{Compiling and Running Programs}\label{sec:compiling}
The \openshmem Specification does not specify how
\openshmem programs are compiled, linked, and run. This section shows some
examples of how wrapper programs are utilized in the \openshmem Reference
Implementation to compile and launch programs.

\section{Compilation}
\subsection*{Programs written in \Cstd}

The \openshmem Reference Implementation provides a wrapper program, named
\textbf{oshcc}, to aid in the compilation of \Cstd programs.
The wrapper may be called as follows:

\begin{lstlisting}[language=bash]
oshcc <compiler options> -o myprogram myprogram.c
\end{lstlisting}
Where the $\langle\mbox{compiler options}\rangle$ are options understood by the
underlying \Cstd compiler called by \textbf{oshcc}.


\subsection*{Programs written in \Cpp}

The \openshmem Reference Implementation provides a wrapper program, named
\textbf{oshc++}, to aid in the compilation of \Cpp programs.
The wrapper may be called as follows:

\begin{lstlisting}[language=bash]
oshc++ <compiler options> -o myprogram myprogram.cpp
\end{lstlisting}
Where the $\langle\mbox{compiler options}\rangle$ are options understood by the
underlying \Cpp compiler called by \textbf{oshc++}.


\section{Running Programs}

The \openshmem Reference Implementation provides a wrapper program, named
\textbf{oshrun}, to launch \openshmem programs.
The wrapper may be called as follows:

\begin{lstlisting}[language=bash]
oshrun <runner options> -np <#> <program> <program arguments>
\end{lstlisting}
The arguments for \textbf{oshrun} are:

\begin{tabular}{p{0.3\textwidth}p{0.6\textwidth}}
$\langle\mbox{runner options}\rangle$ & {Options passed to the underlying launcher.}\tabularnewline
-np $\langle\mbox{\#}\rangle$ & {The number of \acp{PE} to be used in the execution.}\tabularnewline
$\langle\mbox{program}\rangle$ & {The program executable to be launched.}\tabularnewline
$\langle\mbox{program arguments}\rangle$ & {Flags and other parameters to pass to the program.}\tabularnewline
\end{tabular}




\chapter{Undefined Behavior in OpenSHMEM}\label{sec:undefined}

The \openshmem Specification formalizes the expected behavior of
its library routines.  In cases where routines are improperly used
or the input is not in accordance with the Specification, the behavior
is undefined.

\begin{longtable}{|>{\raggedright}p{0.3\textwidth}|>{\raggedright}p{0.6\textwidth}|}
\hline
\textbf{Inappropriate Usage} & \textbf{Undefined Behavior}\tabularnewline
\hline
\endhead
Uninitialized library & If the \openshmem library is not initialized,
calls to non-initializing \openshmem routines have undefined
behavior.  For example, an implementation may try to continue or may abort
immediately upon an \openshmem call into the uninitialized library.
\tabularnewline
\hline
Multiple calls to initialization routines & In an \openshmem program where
the initialization routines \FUNC{shmem\_init} or \FUNC{shmem\_init\_thread}
have already been called, any subsequent calls to these initialization routines
result in undefined behavior.
\tabularnewline
\hline
Accessing non-existent \acp{PE} & If a communications routine accesses a
non-existent \ac{PE}, then the \openshmem library may handle this
situation in an implementation-defined way.  For example, the library may report
an error message saying that the \ac{PE} accessed is outside the range of
accessible \acp{PE}, or may exit without a warning.\tabularnewline
\hline
Use of non-symmetric variables & Some routines require remotely accessible
variables to perform their function.  For example, a \PUT{} to a non-symmetric variable may
be trapped where possible and the library may abort the program.  Another
implementation may choose to continue execution with or without a warning.
\tabularnewline
\hline
Non-symmetric allocation of symmetric memory & The symmetric memory management routines are
collectives. For example, all \acp{PE} in the program must call
\FUNC{shmem\_malloc} with the same \VAR{size} argument.  Program behavior after a
mismatched \FUNC{shmem\_malloc} call is undefined.\tabularnewline
\hline
Use of null pointers with non-zero \VAR{len} specified & In any \openshmem routine
that takes a pointer and \VAR{len} describing the number of elements in that
pointer, a null pointer may not be given unless the corresponding \VAR{len} is also
specified as zero. Otherwise, the resulting behavior is undefined.
The following cases summarize this behavior:
\begin{itemize}
    \item \VAR{len} is 0, pointer is null: supported.
    \item \VAR{len} is not 0, pointer is null: undefined behavior.
    \item \VAR{len} is 0, pointer is non-null: supported.
    \item \VAR{len} is not 0, pointer is non-null: supported.
\end{itemize}
\tabularnewline
\hline
\end{longtable}


\chapter{Interoperability with Other Programming Models}\label{sec:interoperability}

OpenSHMEM routines may be used in conjunction with the routines of other
communication libraries or parallel languages in the same program. This section
describes the interoperability with other programming models, including
clarification of undefined behaviors caused by mixed use of different models,
and advice to \openshmem library users and developers that may improve the portability
and performance of hybrid programs.

\section{Subprocesses}

In some cases, an \openshmem application may be used to create or
orchestrate other processes, which can be created through a number of
system-level interfaces.  In these instances, such subprocesses are
subject to the following interoperability constraints.

On platforms that provide the referenced POSIX\footnotemark[1] \acp{API}:

\footnotetext[1]{POSIX, the Portable Operating System Interface, is
  formally specified in IEEE Std 1003.1-2017 and The Open Group
  Technical Standard Base Specifications, Issue 7.}

\begin{itemize}
\item When \FUNC{fork} is invoked before the \openshmem library is
  initialized, only one of either the parent or child processes may
  initialize the \openshmem library.
\item When \FUNC{fork} is invoked within the \openshmem portion of the
  program or after the \openshmem library has been finalized, the
  newly created child process shall not call any \openshmem routines;
  otherwise, the behavior is undefined.
\item Not all \openshmem implementations may support the use of
  \FUNC{fork} during or after the \openshmem portion of a
  program. When subprocess creation is needed in these instances, the
  application is encouraged to make use of the \FUNC{posix\_spawn} and
  \FUNC{posix\_spawnp} \acp{API}.
\end{itemize}

\parimpnotes{
  All \openshmem implementations should ensure interoperability with
  the \FUNC{posix\_spawn} and \FUNC{posix\_spawnp} \acp{API}.
}


\section{MPI Interoperability}

\openshmem and \ac{MPI} are two commonly used parallel programming models for
distributed-memory systems. The user can choose to utilize both models in the same program
to efficiently and easily support various communication patterns.

A vendor may implement the \openshmem and \ac{MPI} libraries in different ways. For
instance, one may implement both \openshmem and \ac{MPI} as standalone libraries,
each of which allocates and initializes fully isolated communication
resources.
Another approach
is to implement both \openshmem and \ac{MPI} interfaces within the
same software system in order to share a communication resource when possible.

To improve interoperability and portability in \openshmem + \ac{MPI} hybrid
programming, we clarify the relevant semantics in the following subsections.


\subsection{Initialization}
In order to ensure that a hybrid program can be portably performed with different vendor
implementations, the \openshmem environment of the program must be initialized by
a call to \FUNC{shmem\_init} or \FUNC{shmem\_init\_thread} and be finalized by
a call to \FUNC{shmem\_finalize}; the \ac{MPI} environment of the program must be initialized
by a call to \FUNC{MPI\_Init} or \FUNC{MPI\_Init\_thread} and be finalized by a
call to \FUNC{MPI\_Finalize}.

\parimpnotes{
  Portable implementations of OpenSHMEM and \ac{MPI} must ensure that the initialization
  calls can be made in an arbitrary order within a program; the same rule also
  applies to the finalization calls. A software runtime that utilizes a shared
  communication resource for \openshmem and \ac{MPI} communication may maintain an
  internal reference counter in order to ensure that the shared resource is
  initialized only once and thus no shared resource is released until the last
  finalization call is made.
}


\subsection{Dynamic Process Creation}
\label{subsec:interoperability:mpmd}

\ac{MPI} defines a dynamic process model that allows creation of processes after
an \ac{MPI} application has started (e.g., by calling \FUNC{MPI\_Comm\_spawn}) and
connection to independent processes (e.g., through \FUNC{MPI\_Comm\_accept}
and \FUNC{MPI\_Comm\_connect}).
It provides a mechanism to establish communication
between the newly created processes and the existing \ac{MPI} application (see
\ac{MPI} standard version 3.1, Chapter 10).
Unlike \ac{MPI}, \openshmem starts all processes at once and requires all \acp{PE} to
collectively allocate and initialize resources (e.g., symmetric heap) used by
the \openshmem library before any other \openshmem routine may
be called. \openshmem does not support communication with dynamically created
or connected processes. In such a scenario, \ac{MPI} can be used to communicate
with these processes.


\subsection{Thread Safety}
\label{subsec:interoperability:thread}
Both \openshmem and \ac{MPI} define the interaction with user threads in a program
with routines that can be used for initializing and querying the thread
environment. A hybrid program may request different thread levels
at the initialization calls of \openshmem and \ac{MPI} environments; however, the
returned support level provided by the \openshmem or \ac{MPI} library might be different
from that returned in an non-hybrid program. For instance, the former
initialization call in a hybrid program may initialize a resource with the
requested thread level, but the supported level cannot be updated by a subsequent
initialization call if the underlying software runtime of \openshmem and \ac{MPI}
share the same internal communication resource.
The program should always check the \VAR{provided} thread level returned
at the corresponding initialization call or query the level of thread support
after initialization to portably ensure thread support in each communication
environment.

Both \openshmem and \ac{MPI} define similar thread levels, namely, \VAR{THREAD\_SINGLE},
\VAR{THREAD\_FUNNELED}, \VAR{THREAD\_SERIALIZED}, and \VAR{THREAD\_MULTIPLE}.
When requesting threading support in a hybrid program, however,
the following additional rules are applied if the implementations of \openshmem
and \ac{MPI} share the same internal communication resource.
It is strongly recommended to always follow these rules to ensure program
portability.

\begin{itemize}
    \item The \VAR{THREAD\_SINGLE} thread level requires a single-threaded program.
    Hence, a hybrid program should not request \VAR{THREAD\_SINGLE} at the initialization
    call of either \openshmem or \ac{MPI} but request a different thread level at the
    initialization call of the other model.

    \item The \VAR{THREAD\_FUNNELED} thread level allows only the main thread to
    make communication calls. A hybrid program using the \VAR{THREAD\_FUNNELED}
    thread level in both \openshmem and \ac{MPI} should ensure that the same main thread
    is used in both communication environments.

    \item The \VAR{THREAD\_SERIALIZED} thread level requires the program to ensure
    that communication calls are not made concurrently by multiple threads. If a
    hybrid program uses \VAR{THREAD\_SERIALIZED} in one communication environment
    and \VAR{THREAD\_SERIALIZED} or \VAR{THREAD\_FUNNELED} in the other one, it
    should also guarantee that the \openshmem and \ac{MPI} calls are not made concurrently
    from two distinct threads.
\end{itemize}

\subsection{Mapping Process Identification Numbers}
\label{subsec:interoperability:id}

Similar to the \ac{PE} number in \openshmem, \ac{MPI} defines rank as the
identification number of a process in a communicator. Both the \openshmem \ac{PE}
and the \ac{MPI} rank are unique integers assigned from zero to one less than the total
number of processes. In a hybrid program, the \openshmem
\ac{PE} number in \LibHandleRef{SHMEM\_TEAM\_WORLD}
and the \ac{MPI} rank in \VAR{MPI\_COMM\_WORLD} of a process can be equal.
This feature, however, may be provided by only some of the \openshmem and \ac{MPI}
implementations (e.g., if both environments share the same underlying process
manager) and is not portably guaranteed. A portable program should always
use the standard functions in each model, namely, \FUNC{shmem\_team\_my\_pe} in \openshmem
and \FUNC{MPI\_Comm\_rank} in \ac{MPI}, to query the process identification numbers
in each communication environment and manage the mapping of identifiers in the
program when necessary.

\subsubsection*{Examples}
\label{subsubsec:interoperability:id:example}

\SourceExample{example_code/hybrid_mpi_mapping_id.c}{
  The following example demonstrates how to manage the mapping between
  \openshmem \ac{PE} numbers and \ac{MPI} ranks in
  \VAR{MPI\_COMM\_WORLD} in a hybrid \openshmem and \ac{MPI} program.
}


\SourceExample{example_code/hybrid_mpi_mapping_id_shmem_comm.c}{
  The following example demonstrates an alternative approach for
  managing the mapping of process identification numbers in a hybrid
  program. The program creates a new MPI communicator, named
  \VAR{shmem\_comm}, that contains all processes in
  \VAR{MPI\_COMM\_WORLD} and each process has the same \ac{MPI} rank
  number as its \openshmem \ac{PE} number.
}

\subsection{RMA Programming Models}
\label{subsec:interoperability:rma}

\openshmem and \ac{MPI} each define similar one-sided communication models;
however, a portable program should not assume interoperability between these
models.
For instance, \openshmem guarantees the atomicity only of concurrent \openshmem \ac{AMO} operations
that operate on symmetric data with the same datatype. Access to the same symmetric
object with \ac{MPI} atomic operations, such as an \FUNC{MPI\_Fetch\_and\_op}, may
result in an undefined result. A hybrid program should avoid situations where \ac{MPI} and
\openshmem one-sided operations perform concurrent accesses to the same memory
location; otherwise, the behavior is undefined.

\subsection{Communication Progress}
\label{subsec:interoperability:progress}

\openshmem promises the progression of communication both with and without
\openshmem calls and requires the software progress mechanism in the implementation
(e.g., a progress thread) when the hardware does not provide asynchronous communication
capabilities (see Section \ref{subsec:progress}).
In \ac{MPI}, however, a weak progress semantics is applied. That is,
an \ac{MPI} communication call is guaranteed only to complete in finite time. For
instance, an \FUNC{MPI\_Put} may be completed only when the remote process makes an \ac{MPI}
call that internally triggers the progress of \ac{MPI}, if the underlying hardware
does not support asynchronous communication. A hybrid program
should not assume that the \openshmem library also makes progress for \ac{MPI}.
It can explicitly manage the asynchronous communication of \ac{MPI} in
order to prevent any deadlock or performance degradation.


\chapter{History of OpenSHMEM}\label{sec:openshmem_history}

SHMEM has a long history as a parallel-programming model and has been
extensively used on a number of products since 1993, including the Cray T3D,
Cray X1E, Cray XT3 and XT4, \ac{SGI} Origin, \ac{SGI} Altix, Quadrics-based
clusters, and InfiniBand-based clusters.

\begin{itemize}
\item SHMEM Timeline
  \begin{itemize}
  \item Cray SHMEM
    \begin{itemize}
    \item SHMEM first introduced by Cray Research, Inc.\ in 1993 for Cray T3D
    \item Cray was acquired by \ac{SGI} in 1996
    \item Cray was acquired by Tera in 2000 (MTA)
    \item Platforms: Cray T3D, T3E, C90, J90, SV1, SV2, X1, X2, XE, XMT, XT
    \end{itemize}
  \item \ac{SGI} SHMEM
    \begin{itemize}
    \item \ac{SGI} acquired Cray Research, Inc.\ and SHMEM was integrated into
      \ac{SGI}'s Message Passing Toolkit (MPT)
    \item \ac{SGI} currently owns the rights to SHMEM and \openshmem
    \item Platforms: Origin, Altix 4700, Altix XE, ICE, UV
    \item \ac{SGI} was acquired by Rackable Systems in 2009
    \item \ac{SGI} and \ac{OSSS} signed a
      SHMEM trademark licensing agreement in 2010
    \item \ac{HPE} acquired {SGI} in 2016
    \end{itemize}
  \end{itemize}
\end{itemize}

A listing of \openshmem implementations can be found on
\url{http://www.openshmem.org/}.








\chapter{OpenSHMEM Specification and Deprecated API}\label{sec:dep_api}

\section{Overview}\label{subsec:dep_overview}
\TableIndex{Deprecated API}
For the \openshmem Specification, deprecation is the process of identifying
API that is supported but no longer recommended for use by users.
The deprecated API \textbf{must} be supported until clearly
indicated as otherwise by the Specification.
This chapter records the API or functionality that have been deprecated, the
version of the \openshmem Specification that effected the deprecation, and the
most recent version of the \openshmem Specification in which the feature was
supported before removal.

\begin{center}
\scriptsize
\begin{longtable}{|l|c|c|l|}
    \hline
    \textbf{Deprecated API}
    & \textbf{Deprecated Since}
    & \textbf{Last Version Supported}
    & \textbf{Replaced By} \\
    \hline
    \endhead
    Header Directory: \hyperref[subsec:dep_rationale:mpp]{\HEADER{mpp}} & 1.1 & Current & (none) \\ \hline
    \CorCpp: \hyperref[subsec:start_pes]{\FuncRef{start\_pes}} & 1.2 & Current & \hyperref[subsec:shmem_init]{\FUNC{shmem\_init}} \\ \hline
    \Fortran: \hyperref[subsec:start_pes]{\FuncRef{START\_PES}} & 1.2 & Current & \hyperref[subsec:shmem_init]{\FUNC{SHMEM\_INIT}} \\ \hline
    \hyperref[subsec:start_pes]{Implicit finalization} & 1.2 & Current & \hyperref[subsec:shmem_finalize]{\FUNC{shmem\_finalize}} \\ \hline
    \CorCpp: \FuncRef{\_my\_pe} & 1.2 & Current & \hyperref[subsec:shmem_my_pe]{\FUNC{shmem\_my\_pe}} \\ \hline
    \CorCpp: \FuncRef{\_num\_pes} & 1.2 & Current & \hyperref[subsec:shmem_n_pes]{\FUNC{shmem\_n\_pes}} \\ \hline
    \Fortran: \FuncRef{MY\_PE} & 1.2 & Current & \hyperref[subsec:shmem_my_pe]{\FUNC{SHMEM\_MY\_PE}} \\ \hline
    \Fortran: \FuncRef{NUM\_PES} & 1.2 & Current & \hyperref[subsec:shmem_n_pes]{\FUNC{SHMEM\_N\_PES}} \\ \hline
    \CorCpp: \FuncRef{shmalloc} & 1.2 & Current & \hyperref[subsec:shfree]{\FUNC{shmem\_malloc}} \\ \hline
    \CorCpp: \FuncRef{shfree} & 1.2 & Current & \hyperref[subsec:shfree]{\FUNC{shmem\_free}} \\ \hline
    \CorCpp: \FuncRef{shrealloc} & 1.2 & Current & \hyperref[subsec:shfree]{\FUNC{shmem\_realloc}} \\ \hline
    \CorCpp: \FuncRef{shmemalign} & 1.2 & Current & \hyperref[subsec:shfree]{\FUNC{shmem\_align}} \\ \hline
    \Fortran: \FuncRef{SHMEM\_PUT} & 1.2 & Current & \hyperref[subsec:shmem_put]{\FUNC{SHMEM\_PUT8} or \FUNC{SHMEM\_PUT64}} \\ \hline
    \minitab{\CorCpp: \hyperref[subsec:shmem_cache]{\FuncRef{shmem\_clear\_cache\_inv}}
        \\ \Fortran: \hyperref[subsec:shmem_cache]{\FuncRef{SHMEM\_CLEAR\_CACHE\_INV}}}
        & 1.3 & Current & (none) \\ \hline
    \CorCpp: \hyperref[subsec:shmem_cache]{\FuncRef{shmem\_clear\_cache\_line\_inv}} & 1.3 & Current & (none) \\ \hline
    \minitab{\CorCpp: \hyperref[subsec:shmem_cache]{\FuncRef{shmem\_set\_cache\_inv}}
        \\ \Fortran: \hyperref[subsec:shmem_cache]{\FuncRef{SHMEM\_SET\_CACHE\_INV}}}
        & 1.3 & Current & (none) \\ \hline
    \minitab{\CorCpp: \hyperref[subsec:shmem_cache]{\FuncRef{shmem\_set\_cache\_line\_inv}}
        \\ \Fortran: \hyperref[subsec:shmem_cache]{\FuncRef{SHMEM\_SET\_CACHE\_LINE\_INV}}}
        & 1.3 & Current & (none) \\ \hline
    \minitab{\CorCpp: \hyperref[subsec:shmem_cache]{\FuncRef{shmem\_udcflush}}
        \\ \Fortran: \hyperref[subsec:shmem_cache]{\FuncRef{SHMEM\_UDCFLUSH}}}
        & 1.3 & Current & (none) \\ \hline
    \minitab{\CorCpp: \hyperref[subsec:shmem_cache]{\FuncRef{shmem\_udcflush\_line}}
        \\ \Fortran: \hyperref[subsec:shmem_cache]{\FuncRef{SHMEM\_UDCFLUSH\_LINE}}}
        & 1.3 & Current & (none) \\ \hline
    \LibConstRef{\_SHMEM\_SYNC\_VALUE}         & 1.3 & Current & \hyperref[subsec:library_constants]{\CONST{SHMEM\_SYNC\_VALUE}} \\ \hline
    \LibConstRef{\_SHMEM\_BARRIER\_SYNC\_SIZE} & 1.3 & Current & \hyperref[subsec:library_constants]{\CONST{SHMEM\_BARRIER\_SYNC\_SIZE}} \\ \hline
    \LibConstRef{\_SHMEM\_BCAST\_SYNC\_SIZE}   & 1.3 & Current & \hyperref[subsec:library_constants]{\CONST{SHMEM\_BCAST\_SYNC\_SIZE}} \\ \hline
    \LibConstRef{\_SHMEM\_COLLECT\_SYNC\_SIZE} & 1.3 & Current & \hyperref[subsec:library_constants]{\CONST{SHMEM\_COLLECT\_SYNC\_SIZE}} \\ \hline
    \LibConstRef{\_SHMEM\_REDUCE\_SYNC\_SIZE}  & 1.3 & Current & \hyperref[subsec:library_constants]{\CONST{SHMEM\_REDUCE\_SYNC\_SIZE}} \\ \hline
    \LibConstRef{\_SHMEM\_REDUCE\_MIN\_WRKDATA\_SIZE} & 1.3 & Current & \hyperref[subsec:library_constants]{\CONST{SHMEM\_REDUCE\_MIN\_WRKDATA\_SIZE}} \\ \hline
    \LibConstRef{\_SHMEM\_MAJOR\_VERSION} & 1.3 & Current & \hyperref[subsec:library_constants]{\CONST{SHMEM\_MAJOR\_VERSION}} \\ \hline
    \LibConstRef{\_SHMEM\_MINOR\_VERSION} & 1.3 & Current & \hyperref[subsec:library_constants]{\CONST{SHMEM\_MINOR\_VERSION}} \\ \hline
    \LibConstRef{\_SHMEM\_MAX\_NAME\_LEN} & 1.3 & Current & \hyperref[subsec:library_constants]{\CONST{SHMEM\_MAX\_NAME\_LEN}} \\ \hline
    \LibConstRef{\_SHMEM\_VENDOR\_STRING} & 1.3 & Current & \hyperref[subsec:library_constants]{\CONST{SHMEM\_VENDOR\_STRING}} \\ \hline
    \LibConstRef{\_SHMEM\_CMP\_EQ} & 1.3 & Current & \hyperref[subsec:library_constants]{\CONST{SHMEM\_CMP\_EQ}} \\ \hline
    \LibConstRef{\_SHMEM\_CMP\_NE} & 1.3 & Current & \hyperref[subsec:library_constants]{\CONST{SHMEM\_CMP\_NE}} \\ \hline
    \LibConstRef{\_SHMEM\_CMP\_LT} & 1.3 & Current & \hyperref[subsec:library_constants]{\CONST{SHMEM\_CMP\_LT}} \\ \hline
    \LibConstRef{\_SHMEM\_CMP\_LE} & 1.3 & Current & \hyperref[subsec:library_constants]{\CONST{SHMEM\_CMP\_LE}} \\ \hline
    \LibConstRef{\_SHMEM\_CMP\_GT} & 1.3 & Current & \hyperref[subsec:library_constants]{\CONST{SHMEM\_CMP\_GT}} \\ \hline
    \LibConstRef{\_SHMEM\_CMP\_GE} & 1.3 & Current & \hyperref[subsec:library_constants]{\CONST{SHMEM\_CMP\_GE}} \\ \hline
    \EnvVarRef{SMA\_VERSION}         & 1.4 & Current & \hyperref[subsec:environment_variables]{\ENVVAR{SHMEM\_VERSION}} \\ \hline
    \EnvVarRef{SMA\_INFO}            & 1.4 & Current & \hyperref[subsec:environment_variables]{\ENVVAR{SHMEM\_INFO}} \\ \hline
    \EnvVarRef{SMA\_SYMMETRIC\_SIZE} & 1.4 & Current & \hyperref[subsec:environment_variables]{\ENVVAR{SHMEM\_SYMMETRIC\_SIZE}} \\ \hline
    \EnvVarRef{SMA\_DEBUG}           & 1.4 & Current & \hyperref[subsec:environment_variables]{\ENVVAR{SHMEM\_DEBUG}} \\ \hline
    \minitab{\CorCpp: \FuncRef{shmem\_wait}
        \\ \CorCpp: \FuncRef{shmem\_\FuncParam{TYPENAME}\_wait}}
        & 1.4 & Current & See \textbf{Notes} for \hyperref[subsec:shmem_wait_until]{\FUNC{shmem\_wait\_until}} \\ \hline
    \CorCpp: \FuncRef{shmem\_wait\_until} & 1.4 & Current
        & \Cstd[11]: \hyperref[subsec:shmem_wait_until]{\FUNC{shmem\_wait\_until}}, \CorCpp: \hyperref[subsec:shmem_wait_until]{\FUNC{shmem\_long\_wait\_until}} \\ \hline
    \minitab{\Cstd[11]: \FuncRef{shmem\_fetch}
        \\ \CorCpp: \FuncRef{shmem\_\FuncParam{TYPENAME}\_fetch}}
        & 1.4 & Current & \hyperref[subsec:shmem_atomic_fetch]{\FUNC{shmem\_atomic\_fetch}} \\ \hline
    \minitab{\Cstd[11]: \FuncRef{shmem\_set}
        \\ \CorCpp: \FuncRef{shmem\_\FuncParam{TYPENAME}\_set}}
        & 1.4 & Current & \hyperref[subsec:shmem_atomic_set]{\FUNC{shmem\_atomic\_set}} \\ \hline
    \minitab{\Cstd[11]: \FuncRef{shmem\_cswap}
        \\ \CorCpp: \FuncRef{shmem\_\FuncParam{TYPENAME}\_cswap}}
        & 1.4 & Current & \hyperref[subsec:shmem_atomic_compare_swap]{\FUNC{shmem\_atomic\_compare\_swap}} \\ \hline
    \minitab{\Cstd[11]: \FuncRef{shmem\_swap}
        \\ \CorCpp: \FuncRef{shmem\_\FuncParam{TYPENAME}\_swap}}
        & 1.4 & Current & \hyperref[subsec:shmem_atomic_swap]{\FUNC{shmem\_atomic\_swap}} \\ \hline
    \minitab{\Cstd[11]: \FuncRef{shmem\_finc}
        \\ \CorCpp: \FuncRef{shmem\_\FuncParam{TYPENAME}\_finc}}
        & 1.4 & Current & \hyperref[subsec:shmem_atomic_fetch_inc]{\FUNC{shmem\_atomic\_fetch\_inc}} \\ \hline
    \minitab{\Cstd[11]: \FuncRef{shmem\_inc}
        \\ \CorCpp: \FuncRef{shmem\_\FuncParam{TYPENAME}\_inc}}
        & 1.4 & Current & \hyperref[subsec:shmem_atomic_inc]{\FUNC{shmem\_atomic\_inc}} \\ \hline
    \minitab{\Cstd[11]: \FuncRef{shmem\_fadd}
        \\ \CorCpp: \FuncRef{shmem\_\FuncParam{TYPENAME}\_fadd}}
        & 1.4 & Current & \hyperref[subsec:shmem_atomic_fetch_add]{\FUNC{shmem\_atomic\_fetch\_add}} \\ \hline
    \minitab{\Cstd[11]: \FuncRef{shmem\_add}
        \\ \CorCpp: \FuncRef{shmem\_\FuncParam{TYPENAME}\_add}}
        & 1.4 & Current & \hyperref[subsec:shmem_atomic_add]{\FUNC{shmem\_atomic\_add}} \\ \hline
    Entire \Fortran API & 1.4 & Current & (none) \\ \hline
    \CorCpp: \FuncRef{shmem\_barrier} & 1.5 & Current &
    \hyperref[subsec:shmem_quiet]{\FUNC{shmem\_quiet}}; \hyperref[subsec:shmem_sync]{\FUNC{shmem\_sync}} \\ \hline
    \CorCpp: Active set based \FuncRef{shmem\_sync} & 1.5 & Current &
    Team based \hyperref[subsec:shmem_sync]{\FUNC{shmem\_sync}} \\ \hline
    \CorCpp: \FuncRef{shmem\_broadcast[32,64]} & 1.5 & Current &
    \hyperref[subsec:shmem_broadcast]{\FUNC{shmem\_broadcast}} \\ \hline
    \CorCpp: \FuncRef{shmem\_collect[32,64]} & 1.5 & Current &
    \hyperref[subsec:shmem_collect]{\FUNC{shmem\_collect}} \\ \hline
    \CorCpp: \FuncRef{shmem\_fcollect[32,64]} & 1.5 & Current &
    \hyperref[subsec:shmem_collect]{\FUNC{shmem\_fcollect}} \\ \hline
    \CorCpp: \FuncRef{shmem\_\TYPENAME\_OP\_to\_all} & 1.5 & Current &
    \hyperref[subsec:shmem_collect]{\FUNC{shmem\_\TYPENAME\_OP\_reduce}} \\ \hline
    \CorCpp: \FuncRef{shmem\_alltoall[32,64]} & 1.5 & Current &
    \hyperref[subsec:shmem_alltoall]{\FUNC{shmem\_alltoall}} \\ \hline
    \CorCpp: \FuncRef{shmem\_alltoalls[32,64]} & 1.5 & Current &
    \hyperref[subsec:shmem_alltoalls]{\FUNC{shmem\_alltoalls}} \\ \hline
    \end{longtable}
\end{center}

\section{Deprecation Rationale}\label{subsec:dep_rationale}

\subsection{Header Directory: \HEADER{mpp}}
\label{subsec:dep_rationale:mpp}
In addition to the default system header paths, \openshmem implementations
must provide all \openshmem-specified header files from the \HEADER{mpp}
header directory such that these headers can be referenced in \CorCpp as
\begin{lstlisting}[language=]
#include <mpp/shmem.h>
#include <mpp/shmemx.h>
\end{lstlisting}
and in \Fortran as
\begin{lstlisting}[language=]
include 'mpp/shmem.fh'
include 'mpp/shmemx.fh'
\end{lstlisting}
for backwards compatibility with \ac{SGI} SHMEM.

\subsection{\CorCpp: \FUNC{start\_pes}}
The \CorCpp routine \FUNC{start\_pes} includes an unnecessary initialization
argument that is remnant of historical \emph{SHMEM} implementations and no
longer reflects the requirements of modern \openshmem implementations.
Furthermore, the naming of \FUNC{start\_pes} does not include the standardized
\shmemprefixLC{} naming prefix. This routine has been deprecated and
\openshmem users are encouraged to use \FUNC{shmem\_init} instead.

\subsection{Implicit Finalization}
Implicit finalization was deprecated and replaced with explicit finalization using the
\FUNC{shmem\_finalize} routine.  Explicit finalization improves portability and
also improves interoperability with profiling and debugging tools.

\subsection{\CorCpp: \FUNC{\_my\_pe}, \FUNC{\_num\_pes}, \FUNC{shmalloc},
    \FUNC{shfree}, \FUNC{shrealloc}, \FUNC{shmemalign}}
The \CorCpp routines \FUNC{\_my\_pe}, \FUNC{\_num\_pes}, \FUNC{shmalloc},
\FUNC{shfree}, \FUNC{shrealloc}, and \FUNC{shmemalign} were deprecated in order
to normalize the \openshmem \ac{API} to use \shmemprefixLC{} as the standard
prefix for all routines.

\subsection{\textit{Fortran}: \FUNC{START\_PES}, \FUNC{MY\_PE}, \FUNC{NUM\_PES}} %% WARNING: Issue #66.
The \Fortran routines \FUNC{START\_PES}, \FUNC{MY\_PE}, and \FUNC{NUM\_PES}
were deprecated in order to minimize the API differences from the deprecation
of \CorCpp routines \FUNC{start\_pes}, \FUNC{\_my\_pe}, and \FUNC{\_num\_pes}.

\subsection{\textit{Fortran}: \FUNC{SHMEM\_PUT}} %% WARNING: Issue #66.
The \Fortran routine \FUNC{SHMEM\_PUT} is defined only for the \Fortran
\ac{API} and is semantically identical to \Fortran routines
\FUNC{SHMEM\_PUT8} and \FUNC{SHMEM\_PUT64}.  Since \FUNC{SHMEM\_PUT8} and
\FUNC{SHMEM\_PUT64} have defined equivalents in the \CorCpp interface,
\FUNC{SHMEM\_PUT} is ambiguous and has been deprecated.

\subsection{SHMEM\_CACHE}
The \FUNC{SHMEM\_CACHE} \ac{API}
\begin{center}
\begin{tabular}{ll}
    \CorCpp: & \Fortran: \\
    \FUNC{shmem\_clear\_cache\_inv}     & \FUNC{SHMEM\_CLEAR\_CACHE\_INV} \\
    \FUNC{shmem\_set\_cache\_inv}       & \FUNC{SHMEM\_SET\_CACHE\_INV} \\
    \FUNC{shmem\_set\_cache\_line\_inv} & \FUNC{SHMEM\_SET\_CACHE\_LINE\_INV} \\
    \FUNC{shmem\_udcflush}              & \FUNC{SHMEM\_UDCFLUSH} \\
    \FUNC{shmem\_udcflush\_line}        & \FUNC{SHMEM\_UDCFLUSH\_LINE} \\
    \FUNC{shmem\_clear\_cache\_line\_inv} \\
\end{tabular}
\end{center}
was originally implemented for systems with cache-management instructions.
This API has largely gone unused on cache-coherent system architectures.
\FUNC{SHMEM\_CACHE} has been deprecated.

\subsection{\CONST{\_SHMEM\_*} Library Constants}
The library constants
\begin{center}
\begin{tabular}{ll}
    \CONST{\_SHMEM\_SYNC\_VALUE}         & \CONST{\_SHMEM\_MAX\_NAME\_LEN} \\
    \CONST{\_SHMEM\_BARRIER\_SYNC\_SIZE} & \CONST{\_SHMEM\_VENDOR\_STRING} \\
    \CONST{\_SHMEM\_BCAST\_SYNC\_SIZE}   & \CONST{\_SHMEM\_CMP\_EQ} \\
    \CONST{\_SHMEM\_COLLECT\_SYNC\_SIZE} & \CONST{\_SHMEM\_CMP\_NE} \\
    \CONST{\_SHMEM\_REDUCE\_SYNC\_SIZE}  & \CONST{\_SHMEM\_CMP\_LT} \\
    \CONST{\_SHMEM\_REDUCE\_MIN\_WRKDATA\_SIZE} & \CONST{\_SHMEM\_CMP\_LE} \\
    \CONST{\_SHMEM\_MAJOR\_VERSION}      & \CONST{\_SHMEM\_CMP\_GT} \\
    \CONST{\_SHMEM\_MINOR\_VERSION}      & \CONST{\_SHMEM\_CMP\_GE} \\
\end{tabular}
\end{center}
do not adhere to the \Cstd standard's reserved identifiers and the \Cpp
standard's reserved names.  These constants were deprecated and replaced
with corresponding constants of prefix \shmemprefix{} that adhere to \CorCpp{}
and \Fortran naming conventions.

\subsection{\ENVVAR{SMA\_*} Environment Variables}\label{subsec:deprecate-sma-env}
The environment variables \ENVVAR{SMA\_VERSION}, \ENVVAR{SMA\_INFO},
\ENVVAR{SMA\_SYMMETRIC\_SIZE}, and \ENVVAR{SMA\_DEBUG}
were deprecated in order to normalize the \openshmem \ac{API} to use
\shmemprefix{} as the standard prefix for all environment variables.

\subsection{\CorCpp: \FUNC{shmem\_wait}}
The \CorCpp interface for \FUNC{shmem\_wait} and \FUNC{shmem\_\FuncParam{TYPENAME}\_wait}
was identified as unintuitive with respect to
the comparison operation it performed.  As \FUNC{shmem\_wait} can be trivially
replaced by \FUNC{shmem\_wait\_until} where \VAR{cmp} is
\CONST{SHMEM\_CMP\_NE}, the \FUNC{shmem\_wait} interface was deprecated in
favor of \FUNC{shmem\_wait\_until}, which makes the comparison operation
explicit and better communicates the developer's intent.

\subsection{\CorCpp: \FUNC{shmem\_wait\_until}}
The \CTYPE{long}-typed \CorCpp routine \FUNC{shmem\_wait\_until} was deprecated
in favor of the \Cstd[11] type-generic interface of the same name or the
explicitly typed \CorCpp routine \FUNC{shmem\_long\_wait\_until}.

\subsection{\textit{C11} and \CorCpp: \FUNC{shmem\_fetch}, \FUNC{shmem\_set}, %% Issue #66.
    \FUNC{shmem\_cswap}, \FUNC{shmem\_swap}, \FUNC{shmem\_finc},
    \FUNC{shmem\_inc}, \FUNC{shmem\_fadd}, \FUNC{shmem\_add}}
The \Cstd[11] and \CorCpp interfaces for
\begin{center}
\begin{tabular}{ll}
    \Cstd[11]: & \CorCpp: \\
    \FUNC{shmem\_fetch} & \FUNC{shmem\_\FuncParam{TYPENAME}\_fetch} \\
    \FUNC{shmem\_set}   & \FUNC{shmem\_\FuncParam{TYPENAME}\_set}   \\
    \FUNC{shmem\_cswap} & \FUNC{shmem\_\FuncParam{TYPENAME}\_cswap} \\
    \FUNC{shmem\_swap}  & \FUNC{shmem\_\FuncParam{TYPENAME}\_swap}  \\
    \FUNC{shmem\_finc}  & \FUNC{shmem\_\FuncParam{TYPENAME}\_finc}  \\
    \FUNC{shmem\_inc}   & \FUNC{shmem\_\FuncParam{TYPENAME}\_inc}   \\
    \FUNC{shmem\_fadd}  & \FUNC{shmem\_\FuncParam{TYPENAME}\_fadd}  \\
    \FUNC{shmem\_add}   & \FUNC{shmem\_\FuncParam{TYPENAME}\_add}   \\
\end{tabular}
\end{center}
were deprecated and replaced with
similarly named interfaces within the \FUNC{shmem\_atomic\_*} namespace
in order to more clearly identify these calls as performing atomic operations.
In addition, the abbreviated names ``cswap'', ``finc'', and ``fadd'' were
expanded for clarity to ``compare\_swap'', ``fetch\_inc'', and ``fetch\_add''.

\subsection{\textit{Fortran} API}\label{subsec:deprecate-fortran} %% WARNING: Issue #66.
The entire \openshmem \Fortran API was deprecated in \openshmem[1.4] and
removed in \openshmem[1.5] because of a general lack of
use and a lack of conformance with legacy \Fortran standards. In lieu of an
extensive update of the \Fortran API, \Fortran users are encouraged to
leverage the \openshmem Specification's \Cstd API through the
\Fortran--\Cstd interoperability initially standardized by \Fortran[2003]%
\footnote{Formally, \Fortran[2003] is known as ISO/IEC~1539-1:2004(E).}.


\subsection{Active-set-based collective routines}
With the addition of \openshmem teams, the previous methods for performing collective
operations has been superseded by a more readable, flexible method for
organizing and communicating between groups of \acp{PE}. All collective routines
which previously indicated subgroups of \acp{PE} with a list of
parameters to describe the subgroup composition should be phased
out in favor of using collective operations with a team parameter.

When moving from active set routines to teams based routines, the fixed-size
versions of the routines, e.g. \FUNC{shmem\_broadcast32}, were not
carried forward. Instead, all teams based collective routines use standard
\Cstd types with the option to use generic \textit{C11} functions for more portable
and maintainable implementations.

\subsection{\CorCpp: \FUNC{shmem\_barrier}}
Each \openshmem team might
be associated with some number of communication contexts. The \FUNC{shmem\_barrier}
functions imply that the default context is quiesced after synchronizing
some set of \acp{PE}. Since teams may have some number of contexts associated
with the team, it becomes less clear which context would be the ``default'' context
for that particular team. Rather than continue to support \FUNC{shmem\_barrier}
for active-sets or teams, programs should use a call to \FUNC{shmem\_quiet}
followed by a call to \FUNC{shmem\_sync} in order to explicitly
indicate which context to quiesce.

\chapter{Changes to this Document}\label{sec:changelog}

\section{Version 1.5}
Major changes in \openshmem[1.5] include \dots

The following list describes the specific changes in \openshmem[1.5]:
\begin{itemize}
%
\item Added \FUNC{shmem\_malloc\_with\_hints} interface and corresponding hints
\CONST{SHMEM\_MALLOC\_ATOMICS\_REMOTE} and \CONST{SHMEM\_MALLOC\_SIGNAL\_REMOTE}.
\\ See Section \ref{subsec:shmmallochint} and \ref{subsec:library_constants}.
%
\item Added support for nonblocking \ac{AMO} functions.
\\ See Section \ref{sec:amo-nbi}.
%
\item Added support for blocking \OPR{put-with-signal} functions.
\\ See Section \ref{subsec:shmem_put_signal}.
%
\item Added support for nonblocking \OPR{put-with-signal} functions.
\\ See Section \ref{subsec:shmem_put_signal_nbi}.
%
\item Clarified that point-to-point synchronization routines preserve the
  atomicity of OpenSHMEM \acp{AMO}.
\\ See Section~\ref{subsec:amo_guarantees}.
%
\item Clarified that symmetric variables used as \VAR{ivar} arguments to
  point-to-point synchronization routines must be updated using OpenSHMEM
  \acp{AMO}.
\\ See Section~\ref{subsec:p2p_intro}.
%
\item Removed the entire \openshmem \Fortran API. 
%
\item Added support for multipliers in \VAR{SHMEM\_SYMMETRIC\_SIZE}
environment variables.
\\ See Section \ref{subsec:environment_variables}.
%
\item Added support for a multiple-element point-to-point synchronization API with
  the functions: \FUNC{shmem\_wait\_until\_all}, \FUNC{shmem\_wait\_until\_any},
  \FUNC{shmem\_wait\_until\_some}, \FUNC{shmem\_test\_all},
  \FUNC{shmem\_test\_any}, and \FUNC{shmem\_test\_some}.
  \\See Sections \ref{subsec:shmem_wait_until_all},
  \ref{subsec:shmem_wait_until_any}, \ref{subsec:shmem_wait_until_some},
  \ref{subsec:shmem_test_all}, \ref{subsec:shmem_test_any}, and
  \ref{subsec:shmem_test_some}.
%
\item Added support for vectorized comparison values in the multiple-element
  point-to-point synchronization API with the functions:
  \FUNC{shmem\_wait\_until\_all\_vector}, \FUNC{shmem\_wait\_until\_any\_vector},
  \FUNC{shmem\_wait\_until\_some\_vector}, \\
  \FUNC{shmem\_test\_all\_vector}, \FUNC{shmem\_test\_any\_vector}, and
  \FUNC{shmem\_test\_some\_vector}.
  \\See Sections \ref{subsec:shmem_wait_until_all_vector},
  \ref{subsec:shmem_wait_until_any_vector}, \ref{subsec:shmem_wait_until_some_vector},
  \ref{subsec:shmem_test_all_vector}, \ref{subsec:shmem_test_any_vector}, and
  \ref{subsec:shmem_test_some_vector}.
%
\item Added \openshmem profiling interface.
  \\ See Section~\ref{sec:openshmem_profiling_interface}.
%
\item Specified the validity of communication contexts, added the constant
  \CONST{SHMEM\_CTX\_INVALID}, and clarified the behavior of
  \FUNC{shmem\_ctx\_*} routines on invalid contexts.
  \\ See Section~\ref{sec:ctx}.
%
\item Clarified \ac{PE} active set requirements.
    \\See Section~\ref{subsec:coll}.
%
\item Clarified that when the \VAR{size} argument is zero, symmetric heap
    allocation routines perform no action and return a null pointer; that
    symmetric heap management routines that perform no action do not perform a
    barrier; and that the \VAR{alignment} argument to \FUNC{shmem\_align} must
    be power of two multiple of \CONST{sizeof(void*)}.
    \\See Section~\ref{subsec:shfree}.
%
\item Clarified that the \openshmem lock API provides a non-reentrant mutex and
    that \FUNC{shmem\_clear\_lock} performs a quiet operation on the default
    context.
    \\See Section~\ref{subsec:shmem_lock}
%
\item Clarified the atomicity guarantees of the \openshmem memory model.
    \\See Section~\ref{subsec:amo_guarantees}.
%
\end{itemize}

\section{Version 1.4}
Major changes in \openshmem[1.4] include
multithreading support,
\emph{contexts} for communication management,
\FUNC{shmem\_sync},
\FUNC{shmem\_calloc},
expanded type support,
a new namespace for atomic operations,
atomic bitwise operations,
\FUNC{shmem\_test} for nonblocking point-to-point synchronization,
and \Cstd[11] type-generic interfaces for point-to-point synchronization.

The following list describes the specific changes in \openshmem[1.4]:
\begin{itemize}
%
\item New communication management API, including \FUNC{shmem\_ctx\_create};
    \FUNC{shmem\_ctx\_destroy}; and additional RMA, AMO, and memory ordering
    routines that accept \CTYPE{shmem\_ctx\_t} arguments.
\\See Section \ref{sec:ctx}.
%
\item New API \FUNC{shmem\_sync\_all} and \FUNC{shmem\_sync} to provide \ac{PE}
    synchronization without completing pending communication operations.
    \\See Sections \ref{subsec:shmem_sync_all} and \ref{subsec:shmem_sync}.
%
\item Clarified that the \openshmem extensions header files are required, even when empty.
\\See Section~\ref{subsec:bindings}.
%
\item Clarified that the \FUNC{SHMEM\_GET64} and \FUNC{SHMEM\_GET64\_NBI}
    routines are included in the \Fortran language bindings.\\
    See Sections \ref{subsec:shmem_get} and \ref{subsec:shmem_get_nbi}.
%
\item Clarified that \FUNC{shmem\_init} must be matched with a call to
    \FUNC{shmem\_finalize}.
\\See Sections \ref{subsec:shmem_init} and \ref{subsec:shmem_finalize}.
%
\item Added the \CONST{SHMEM\_SYNC\_SIZE} constant.
\\See Section \ref{subsec:library_constants}.
%
\item Added type-generic interfaces for \FUNC{shmem\_wait\_until}.
\\ See Section \ref{subsec:shmem_wait_until}.
%
\item Removed the \VAR{volatile} qualifiers from the \VAR{ivar} arguments to
\FUNC{shmem\_wait} routines and the \VAR{lock} arguments in the lock API.
\emph{Rationale: Volatile qualifiers were added to several API routines in
\openshmem[1.3]; however, they were later found to be unnecessary.}
\\ See Sections \ref{subsec:shmem_wait_until} and \ref{subsec:shmem_lock}.
%
\item Deprecated the \VAR{SMA\_}* environment variables and added equivalent
\VAR{SHMEM\_}* environment variables.
\\ See Section \ref{subsec:environment_variables}.
%
\item Added the \Cstd[11] \CTYPE{\_Noreturn} function specifier to
\FUNC{shmem\_global\_exit}.
\\ See Section \ref{subsec:shmem_global_exit}.
%
\item Clarified ordering semantics of memory ordering, point-to-point synchronization, and collective
synchronization routines.
%
\item Clarified deprecation overview and added deprecation rationale in Annex F.
\\See Section \ref{sec:dep_api}.
%
\item Deprecated header directory \HEADER{mpp}.
\\See Section \ref{sec:dep_api}.
%
\item Deprecated the \FUNC{shmem\_wait} functions and the \CTYPE{long}-typed \CorCpp \FUNC{shmem\_wait\_until} function.
\\ See Section \ref{subsec:p2p_intro}.
%
\item Added the \FUNC{shmem\_test} functions.
\\ See Section \ref{subsec:p2p_intro}.
%
\item Added the \FUNC{shmem\_calloc} function.
\\ See Section \ref{subsec:shmem_calloc}.
%
\item Introduced the thread safe semantics that define the interaction between
    \openshmem routines and user threads.
\\See Section \ref{subsec:thread_support}.
%
\item Added the new routine \FUNC{shmem\_init\_thread} to initialize the
    \openshmem library with one of the defined thread levels.
\\See Section \ref{subsec:shmem_init_thread}.
%
\item Added the new routine \FUNC{shmem\_query\_thread} to query the thread
    level provided by the \openshmem implementation.
\\See Section \ref{subsec:shmem_query_thread}.
%
\item Clarified the semantics of \FUNC{shmem\_quiet} for a multithreaded
    \openshmem \ac{PE}.
\\See Section \ref{subsec:shmem_quiet}
%
\item Revised the description of \FUNC{shmem\_barrier\_all} for a multithreaded
    \openshmem \ac{PE}.
\\See Section \ref{subsec:shmem_barrier_all}
%
\item Revised the description of \FUNC{shmem\_wait} for a multithreaded
    \openshmem \ac{PE}.
\\See Section \ref{subsec:shmem_wait_until}
%
\item Clarified description for \CONST{SHMEM\_VENDOR\_STRING}.
\\See Section \ref{subsec:library_constants}.
%
\item Clarified description for \CONST{SHMEM\_MAX\_NAME\_LEN}.
\\See Section \ref{subsec:library_constants}.
%
\item Clarified API description for \FUNC{shmem\_info\_get\_name}.
\\See Section \ref{subsec:shmem_info_get_name}.
%
\item Expanded the type support for RMA, AMO, and point-to-point
    synchronization operations.
\\ See Tables \ref{stdrmatypes}, \ref{stdamotypes}, \ref{extamotypes}, and
    \ref{p2psynctypes}
%
\item Renamed AMO operations to use \FUNC{shmem\_atomic\_*} prefix and
      deprecated old AMO routines.
\\ See Section \ref{sec:amo}.
%
\item Added fetching and non-fetching bitwise AND, OR, and XOR atomic
      operations.
\\ See Section \ref{sec:amo}.
%
\item Deprecated the entire \Fortran API.
%
\item Replaced the \CTYPE{complex} macro in complex-typed reductions with the
      \Cstd[99] (and later) type specifier \CTYPE{\_Complex} to remove an
      implicit dependence on \HEADER{complex.h}.
\\ See Section \ref{subsec:shmem_reductions}.
%
\item Clarified that complex-typed reductions in C are optionally supported.
\\ See Section \ref{subsec:shmem_reductions}.
%
\end{itemize}




\section{Version 1.3}
Major changes in \openshmem[1.3] include the addition of
nonblocking \ac{RMA} operations,
atomic \PUT{} and \GET{} operations,
all-to-all collectives,
and \Cstd[11] type-generic interfaces for \ac{RMA} and \ac{AMO} operations.

The following list describes the specific changes in \openshmem[1.3]:
\begin{itemize}
%
\item Clarified implementation of \acp{PE} as threads.
%
\item Added \CTYPE{const} to every read-only pointer argument.
%
\item Clarified definition of \OPR{Fence}.
\\See Section \ref{subsec:programming_model}.
%
\item Clarified implementation of symmetric memory allocation.
\\See Section \ref{subsec:memory_model}.
%
\item Restricted atomic operation guarantees to other atomic operations with the same datatype.
\\See Section \ref{subsec:amo_guarantees}.
%
\item Deprecation of all constants that start with \CONST{\_SHMEM\_*}.
\\See Section \ref{subsec:library_constants}.
%
\item Added a type-generic interface to \openshmem \ac{RMA} and \ac{AMO}
    operations based on \Cstd[11] Generics.
\\See Sections \ref{sec:rma}, \ref{sec:rma_nbi} and \ref{sec:amo}.
%
\item New nonblocking variants of remote memory access, \FUNC{SHMEM\_PUT\_NBI}
    and \FUNC{SHMEM\_GET\_NBI}.
\\See Sections \ref{subsec:shmem_put_nbi} and \ref{subsec:shmem_get_nbi}.
%
\item New atomic elemental read and write operations, \FUNC{SHMEM\_FETCH} and
    \FUNC{SHMEM\_SET}.
\\See Sections \ref{subsec:shmem_atomic_fetch} and \ref{subsec:shmem_atomic_set}
%
\item New alltoall data exchange operations, \FUNC{SHMEM\_ALLTOALL}
    and \FUNC{SHMEM\_ALLTOALLS}.
\\See Sections \ref{subsec:shmem_alltoall} and \ref{subsec:shmem_alltoalls}.
%
\item Added \CTYPE{volatile} to remotely accessible pointer argument in
    \FUNC{SHMEM\_WAIT} and \FUNC{SHMEM\_LOCK}.
\\See Sections \ref{subsec:shmem_wait_until} and \ref{subsec:shmem_lock}.
%
\item Deprecation of \FUNC{SHMEM\_CACHE}.
\\See Section \ref{subsec:shmem_cache}.
%
\end{itemize}




\section{Version 1.2}
Major changes in \openshmem[1.2] include
a new initialization routine (\FUNC{shmem\_init}),
improvements to the execution model with an explicit
library-finalization routine (\FUNC{shmem\_finalize}),
an early-exit routine (\FUNC{shmem\_global\_exit}),
namespace standardization,
and clarifications to several API descriptions.

The following list describes the specific changes in \openshmem[1.2]:
\begin{itemize}
%
\item Added specification of \VAR{pSync} initialization for all routines that use it.
%
\item Replaced all placeholder variable names \VAR{target} with \VAR{dest} to
      avoid confusion with \Fortran's \KEYWORD{target} keyword.
%
\item New Execution Model for exiting/finishing \openshmem programs.
\\See Section  \ref{subsec:execution_model}.
%
\item New library constants to support API that query version and name information.
\\See Section \ref{subsec:library_constants}.
%
\item New API \FUNC{shmem\_init} to provide mechanism to start an \openshmem
      program and replace deprecated \FUNC{start\_pes}.
\\See Section \ref{subsec:shmem_init}.
%
\item Deprecation of \FUNC{\_my\_pe} and \FUNC{\_num\_pes} routines.
\\See Sections \ref{subsec:shmem_my_pe} and \ref{subsec:shmem_n_pes}.
%
\item New API \FUNC{shmem\_finalize} to provide collective mechanism to cleanly
      exit an \openshmem program and release resources.
\\See Section \ref{subsec:shmem_finalize}.
%
\item New API \FUNC{shmem\_global\_exit} to provide mechanism to exit an
    \openshmem program.
\\See Section \ref{subsec:shmem_global_exit}.
%
\item Clarification related to the address of the referenced object in
    \FUNC{shmem\_ptr}.
\\See Section \ref{subsec:shmem_ptr}.
%
\item New API to query the version and name information.
\\See Section \ref{subsec:shmem_info_get_version} and \ref{subsec:shmem_info_get_name}.
%
\item \openshmem library API normalization. All \Cstd symmetric memory management
      API begins with  \FUNC{shmem\_}.
\\See Section \ref{subsec:shfree}.
%
\item Notes and clarifications added to \FUNC{shmem\_malloc}.
\\See Section \ref{subsec:shfree}.
%
\item Deprecation of \Fortran API routine \FUNC{SHMEM\_PUT}.
\\See Section \ref{subsec:shmem_put}.
%
\item Clarification related to \FUNC{shmem\_wait}.
\\See Section \ref{subsec:shmem_wait_until}.
%
\item Undefined behavior for null pointers without zero counts added.
\\See Annex \ref{sec:undefined}
%
\item Addition of new Annex for clearly specifying deprecated API and its
      support across versions of the \openshmem Specification.
\\See Annex \ref{sec:dep_api}.
%
\end{itemize}




\section{Version 1.1}
Major changes from \openshmem[1.0] to \openshmem[1.1] include
the introduction of the \HEADER{shmemx.h} header file for non-standard API
extensions,
clarifications to completion semantics and API descriptions in agreement with
the \ac{SGI} SHMEM specification,
and general readabilty and usability improvements to the document structure.

The following list describes the specific changes in \openshmem[1.1]:
\begin{itemize}
%
\item Clarifications of the completion semantics of memory synchronization
      interfaces.
\\See Section \ref{subsec:memory_order}.
%
\item Clarification of the completion semantics of memory load and store
      operations in context of \FUNC{shmem\_barrier\_all} and \FUNC{shmem\_barrier}
      routines.
\\See Section \ref{subsec:shmem_barrier_all} and \ref{subsec:shmem_barrier}.
%
\item Clarification of the completion and ordering semantics of
      \FUNC{shmem\_quiet} and \FUNC{shmem\_fence}.
\\See Section \ref{subsec:shmem_quiet} and \ref{subsec:shmem_fence}.
%
\item Clarifications of the completion semantics of \ac{RMA} and \ac{AMO}
      routines.
\\See Sections \ref{sec:rma} and \ref{sec:amo}
%
\item Clarifications of the memory model and the memory alignment requirements
      for symmetric data objects.
\\See Section \ref{subsec:memory_model}.
%
\item Clarification of the execution model and the definition of a \ac{PE}.
\\See Section \ref{subsec:execution_model}
%
\item Clarifications of the semantics of \FUNC{shmem\_pe\_accessible} and
      \FUNC{shmem\_addr\_accessible}.
\\See Section \ref{subsec:shmem_pe_accessible} and \ref{subsec:shmem_addr_accessible}.
%
\item Added an annex on interoperability with \ac{MPI}.
\\See Annex D.
%
\item Added examples to the different interfaces.
%
\item Clarification of the naming conventions for constant in \Cstd and
      \Fortran.
\\See Section \ref{subsec:library_constants} and \ref{subsec:shmem_wait_until}.
%
\item Added \ac{API} calls: \FUNC{shmem\_char\_p}, \FUNC{shmem\_char\_g}.
\\See Sections \ref{subsec:shmem_p} and \ref{subsec:shmem_g}.
%
\item Removed \ac{API} calls: \FUNC{shmem\_char\_put},
      \FUNC{shmem\_char\_get}.
\\See Sections \ref{subsec:shmem_put} and \ref{subsec:shmem_get}.
%
\item The usage of \CTYPE{ptrdiff\_t}, \CTYPE{size\_t}, and \CTYPE{int} in the
      interface signature was made consistent with the description.
\\See Sections \ref{subsec:coll}, \ref{subsec:shmem_iput}, and \ref{subsec:shmem_iget}.
%
\item Revised \FUNC{shmem\_barrier} example.
\\See Section \ref{subsec:shmem_barrier}.
%
\item Clarification of the initial value of \VAR{pSync} work arrays for
\FUNC{shmem\_barrier}.\\ See Section \ref{subsec:shmem_barrier}.
%
\item Clarification of the expected behavior when multiple \FUNC{start\_pes}
calls are encountered.
\\See Section \ref{subsec:start_pes}.
%
\item Corrected the definition of atomic increment operation.
\\See Section \ref{subsec:shmem_atomic_inc}.
%
\item Clarification of the size of the symmetric heap and when it is set.
\\See Section \ref{subsec:shfree}.
%
\item Clarification of the integer and real sizes for \Fortran \ac{API}.
\\See Sections \ref{subsec:shmem_atomic_add},
      \ref{subsec:shmem_atomic_compare_swap},
      \ref{subsec:shmem_atomic_swap},
      \ref{subsec:shmem_atomic_fetch_inc},
      \ref{subsec:shmem_atomic_inc}, and
      \ref{subsec:shmem_atomic_fetch_add}.
%
\item Clarification of the expected behavior on program \OPR{exit}.
\\See Section \ref{subsec:execution_model}, Execution Model.
%
\item More detailed description for the progress of \openshmem operations
provided.
\\See Section \ref{subsec:progress}.
%
\item Clarification of naming convention for non-standard interfaces and their
inclusion in \HEADER{shmemx.h}.
\\See Section \ref{subsec:bindings}.
%
\item Various fixes to \openshmem code examples across the Specification to
include appropriate header files.
%
\item Removing requirement that implementations should detect size mismatch and
return error information for \FUNC{shmalloc} and ensuring consistent
language.
\\See Sections \ref{subsec:shfree} and Annex \ref{sec:undefined}.
%
\item \Fortran programming fixes for examples.\\ See Sections
\ref{subsec:shmem_reductions} and \ref{subsec:shmem_wait_until}.
%
\item Clarifications of the reuse \VAR{pSync} and \VAR{pWork} across
collectives.
\\See Sections \ref{subsec:coll}, \ref{subsec:shmem_broadcast},
      \ref{subsec:shmem_collect} and \ref{subsec:shmem_reductions}.
%
\item Name changes for UV and ICE for \ac{SGI} systems.
\\See Annex \ref{sec:openshmem_history}.
%
\end{itemize}

\chapter{Glossary}\label{sec:Glossary}

$\;$

$ $%
\begin{tabular}{|>{\raggedright}p{0.3\textwidth}|>{\raggedright}p{0.6\textwidth}|}
\hline 
\textbf{Terms} & \textbf{Definitions}\tabularnewline
\hline 
\hline 
Blocking & \tabularnewline
\hline 
Non-blocking & \tabularnewline
\hline 
Local Completion & \tabularnewline
\hline 
Remote Completion & \tabularnewline
\hline 

\end{tabular}

%end of setlength command that was started in frontmatter.tex


\chapter*{Glossary}
\addcontentsline{toc}{chapter}{Glossary}
\input{content/glossary.tex}

\clearpage
\phantomsection
\addcontentsline{toc}{chapter}{Index}
\printindex

\end{document}
